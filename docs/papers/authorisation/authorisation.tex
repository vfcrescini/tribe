% Vino Crescini  <jcrescin@cit.uws.edu.au>

\documentclass[a4paper]{article}
\usepackage[light]{draftcopy}
\author{Vino Fernando Crescini\\jcrescin@cit.uws.edu.au\\\\Intelligent Systems Laboratory\\School of Computing and Information Technology\\University of Western Sydney}
\title{Implementing a High-Level Description Language for a Sequence of Transformations of Authorisation Policies}
\date{17/03/2003}
\begin{document}
  \maketitle
  \section{Introduction}

    Authorisation policies are a collection of rules and constraints that 
    limits accesses to different resources within a system. Such policies may 
    be as simple as assigning read, write or execute permissions on specific 
    files, but some policies can be as high-level and abstract as to reflect 
    the organizational hierarchy of the system. For example, high-level 
    authorisation policies might include the following:

    \begin{itemize}
      \item
        Regional Managers have read and write access to the regional database.
      \item
        Branch Managers have no read and write access to the regional database.
      \item 
        Branch Managers have read and write access to their branch database.
      \item 
         If the branch manager is also a regional manager then he has read
         but not write access to the regional database.
    \end{itemize}

    \subsection{Background}

      Typically, in an access control system, authorisation policies are stored
      into a policy database or policy base. Another entity, the authorisation
      agent (or enforcer agent) ensures that the policies in the policy base
      are enforced. 

      At its simplest form, a policy base may be implemented as an access
      control matrix where the rows represent the subjects, the columns
      represent the objects or resources and each cell contains the access
      permissions. However, the access control matrix approach is limited in
      such a way that it lacks scalability and flexibility. The size of the
      access control matrix is directly proportional to the size of the objects
      times the number of subjects. It is not difficult to see that as the
      system gets larger, any authorisation policy mechanism that uses the
      access control matrix will get progressively slower and less efficient.

      A much better approach is the logic-based system where instead of
      storing each individual rules into a database, a generalized set of
      facts and constraints are stored into the policy base. There have been
      quite a few developments in this field. In their paper, Woo and Lam
      \cite{WL} describes a logic-based specification for policy bases. Bai and
      Varadharajan \cite{BV1} went a few steps further by developing a
      high-level language for the transformation of the policy base from one
      state to another. A transformation, in this sense, is the process of
      adding, removing or modifying a rule in the policy base. Their approach
      also handles conflict resolution within a transformation. In another
      paper, Bai and Varadharajan \cite{BV2} developed another language
      specification to handle sequences of transformations.

    \subsection{Problem}

    \subsection{Plan}

    \subsection{Organisation of this Paper}

    \pagebreak

  \section{A Formal Authorisation Language}

    \subsection{Properties}

      The proposed logic-based authorisation language has one main purpose:
      to provide a simple but expressive interface between the enforcer
      agent and the knowledge base. To achieve this purpose, the authorisation
      language must possess the following properties:

      \begin{itemize}
        \item
          Standardised Syntax

          The language provides high-level semantics to allow logic-based
          operations and at the same time syntactically concise to allow
          parser-level implementation.

        \item
          Provision for Knowledge Base Initialisation

          The language includes provisions for setting logical facts into
          the knowledge base before any operation is performed.

        \item
          Mechanism for Transformations

          The language has the means to modify the preset facts in the
          knowledge base. Such operations or transformations may be to add 
          new facts, or if the fact happens to be contrary to an existing
          fact, to delete existing ones.

        \item
          Ability to Express Default Propositions

          In addition to the ability to transform the knowledge base, the 
          language also possess the ability to express constraints or default
          propositions. Constraints are logical rules or facts that must hold 
          even after transformations.

        \item
          Ability to Express both Classical Negation and Negation as Failure

          Classical negation \emph{$\lnot$} means that a proposition is
          false, explicitly. Negation as failure \emph{not}, on the other 
          hand, simply means that a proposition is assumed to be false if
          there is no evidence that the proposition is true. \cite{GV}

          The language uses \emph{not}, or symbolically, \emph{!} as classical
          negation. Negation as failure is achieved by the \emph{with absence}
          clause of the constraint statement.

        \item
          Query Functionality

          The language of course allows the knowledge base to be queried by
          providing the ability to check whether a given proposition or fact
          holds in the knowledge base.
          
      \end{itemize}

    \subsection{Defintions}

      Before the language may be completely defined, the following components 
      must first be formalised:
 
      \subsubsection{Identifiers}

        An identifier is an alphanumeric string whose first character is an
        alpha character. The formal regular expression specifications of an
        identifier is:

\begin{verbatim}
[a-zA-Z][a-zA-Z0-9_]*
\end{verbatim}

        Identifiers are literal representations of any of the three possible
        conceptual disjoint types:

        \begin{itemize}
          \item Subject (e.g. user1, alice, bob)
          \item Access (e.g. read, write, execute)
          \item Object (e.g. file1, database4)
        \end{itemize}

        In addition to the above types, identifiers may also be used to
        represent a group of identifiers of the same type.

        \begin{itemize}
          \item Subject Group
          \item Access Group
          \item Object Group
        \end{itemize}

      \subsubsection{Variables}

        Variables are identifier placeholders that may be used to represent a
        specific literal or all literals of a specific type. Like an identifier,
        the regular expression of a variable is:

\begin{verbatim}
[a-zA-Z][a-zA-Z0-9_]*
\end{verbatim}

      \subsubsection{Atoms}

        Atoms are composed of a relation and two or more literals to represent
        a logical fact or predicate. The three relations, and therefore, three
        types of atoms are as follows:

        \begin{itemize}

          \item holds(S, A, O)

            An atom of this type states that the subject S holds the
            access-right A for object O.

          \item memb(E, G)

          This atom states that the non-group literal E is an element of the
          group literal G, where E and G are of the same literal type.
          E $\in$ G.

          \item subst(G$_{1}$, G$_{2}$)

          This atom states that the group literal G$_{1}$ is a proper
          subset of the group literal G$_{2}$, where G$_{1}$ and G$_{2}$
          are of the same literal type. G$_{1}$ $\subseteq$ G$_{2}$.

        \end{itemize}

        An atom may also be negated by the \emph{not} ! operator. (e.g.
        !holds(S, A, O))

        In addition to the three types of atoms, two special constant atom are
        also defined in this language: \emph{true} and \emph{false}.

      \subsubsection{Expressions}

        An expression is simply a conjuntion of atoms. In this language,
        disjunctions are not allowed. The conjunction \emph{and} is 
        represented by the symbol \&\&. For example, holds(S, A, O) \&\& 
        !memb(E, G) \&\& subst(G$_{1}$, G$_{2}$) is interpreted as
        holds(S, A, O) $\land$ (not E $\in$ G) $\land$ (G$_{1}$
        $\subseteq$ G$_{2}$).

        Since negated expressions yield a disjunction of atoms, only
        atoms are allowed to be negated. To illustrate: De Morgan's Law
        states that \emph{not} (holds(S$_{1}$, A$_{1}$, O$_{1}$) $\land$
        holds(S$_{2}$, A$_{2}$, O$_{2}$)) is equivalent to (\emph{not} 
        holds(S$_{1}$, A$_{1}$, O$_{1}$)) $\lor$ (\emph{not} holds(S$_{2}$, 
        A$_{2}$, O$_{2}$)). As a consequence, parentheses are allowed only
        to enclose literals within atoms but not to group atoms together. 

        Because the constants \emph{true} and \emph{false} are atoms, they may 
        appear within expressions as well. However, the constants are more
        useful without any other atoms within the same exression, as an
        expression added with the constant \emph{true} will have no effect
        and an expression added with the constant \emph{false} will set the
        entire expression to false.

    \subsection{Syntax and Semantics}

      The language is structured into 5 different ordered sections where each
      section is a set of statements native to that section.  Each statement
      will be terminated by a semicolon ; to mark the end of that statement.

      \subsubsection{Identifier Declarations}

        All identifiers, except transformation identifiers, must first be 
        declared in this section before any proposition, constraint, 
        transformation or query statements are declared.

        This section must contain one or more identifier declaration
        statements. However, more than one identifier of one type may be 
        declared in each identifier declaration statement.

\begin{verbatim}
ident <type> <identifier>[,<identifier2>[,...]]
\end{verbatim}

        where type is one of the following: \emph{sub}, \emph{acc}, \emph{obj},
        \emph{sub-grp}, \emph{acc-grp} or \emph{obj-grp}.

        Example:

\begin{verbatim}
ident sub s1, s2;
ident acc read, write, execute;
ident obj file1, file2;
\end{verbatim}

      \subsubsection{Initial State Definition}

        Before transformatons or constraints can be applied or even defined,
        the initial state must first be defined. This section will contain
        one or more definition statements of the form:

\begin{verbatim}
initially <expression>
\end{verbatim}

      \subsubsection{Constraint Statements}

        This section allows constrains or default propositions to be defined.
        Naturally, constraints will have to be defined first before
        transformations.

        The most general form is:

\begin{verbatim}
<expression1> implies <expression2> with absence <expression3>
\end{verbatim}

        The statement above says that if expression1 is true in the current
        state, and that it cannot be proven that expression3 is true in
        the current state, then expression1 is inferred to be true.

        If the "with absence" clause is ommitted, then expression2 is applied
        if and only if expression1 is true:

\begin{verbatim}
<expression1> implies <expression2>
\end{verbatim}

        A special case of the above statement is when expression1 is the
        constant \emph{true}. In this case, expression2 is always applied
        to the current state. The statement below:

\begin{verbatim}
always <expression2>
\end{verbatim}

        is equivalent to:

\begin{verbatim}
true implies <expression2>
\end{verbatim}

      \subsubsection{Transformation Statements}

        A transformation definition statement must be defined in the following
        syntax:

\begin{verbatim}
<trans-identifier>([<var1>[,<var2>][,...]]])
causes <post-condition-expression>
if <pre-condition-expression>
\end{verbatim}

        where:

        \begin{itemize}
          \item
            trans-identifier is the name of this transformation.
          \item
            post-condition-expression an expression that will be true in the
            resulting state after the transformation is applied.
          \item
            pre-condition-expression an expression that must be true in the
            current state before the transformation may be applied.
        \end{itemize}


        Transformation statements may also contain variables:

\begin{verbatim}
delete_write(SUBJECT, OBJECT)
causes !holds(SUBJECT, a_write, OBJECT)
if holds(SUBJECT, a_write, vOBJECT)
\end{verbatim}
        this means

\begin{verbatim}
delete_write(subject1, object1)
\end{verbatim}

        and

\begin{verbatim}
delete_write(subject2, object2)
\end{verbatim}

        will cause

\begin{verbatim}
!holds(subject1, a_write, object1)
\end{verbatim}

        and

\begin{verbatim}
!holds(subject2, a_write, object2)
\end{verbatim}

        if subject1 and subject2 had write access rights to object1 and
        object2, respectively. The case where the pre-condition is an
        empty set (i.e. the transformation has no pre-condition) does not 
        have to be a special case:

\begin{verbatim}
<trans-identifier>()
causes <post-condition-expression>
\end{verbatim}

        is equivalent to:

\begin{verbatim}
<trans-identifier>()
causes <post-condition-expression>
if true
\end{verbatim}

      \subsubsection{Query Statements}

        The general form is:

\begin{verbatim}
is <policy-expression> [after <trans-identifier1>([identifier1[,...]])[,...]]
\end{verbatim}

        where:

        \begin{itemize}
          \item
            policy-expression is the expression that will be tested.
          \item
            trans-identifiers are the names of the transformations that will
            be applied in series before the query is performed.
          \item
            identifiers are the names of the literals that will be used in
            place of the variables used in the tranformation declarations.
        \end{itemize}

    The statement will return one of the following: \emph{true}, \emph{false}
    or \emph{?}.

    \pagebreak

  \section{Translation into a Logic Program}

    \subsection{Convertion to Extended Logic Program: Flattening the States}

      Extended logic programs are \emph{flat} or stateless. The proposed
      language, however, is composed of logic facts that have a states as
      a result of the transformation property. Formally, a state represents
      a transition of facts as a result of the application of a transformation.
      As a consequence, the number of states is equal to the number of
      transformations plus 1 (the initial state, before any transformations
      are applied).

      Translating the proposed language into an extended logic program requires
      a stateless set of axioms and propositions. The first step in the 
      translation is to reduce every atom, which is composed of two to three
      parameters into a single proposition. This procedure, although does not
      seem necessary is to allow us to have a simplified representation of
      propositions. For example, the atoms holds(s1, a1, o$_{1}$) and memb(e1, g1)
      can be reduced to propositions A and B, respectively.

      \subsubsection{Adding a State Parameter to the Atoms}

        The next step is to add a state parameter into the definiton of an
        atom. This procedure will allow us to differentiate between atoms of
        the same value but on different states. For example, atom A in 
        the initial state S$_0$ can be represented as holds(A, S$_0$),
        likewise, the same atom A in state S$_1$ will be represented as
        holds(A, S$_1$).

      \subsubsection{Handling States in Transformations}

        To express transformation from one state to another in an extended
        logic program, each atom that is involved in a transformation must
        be flattened.

        Let:
  
        \begin{list}{}{}
          \item T = transformation
          \item S$^{'}$ = state before T is applied
          \item S$^{''}$ = state after T is applied
        \end{list}
  
        Then we can define the Res as the function that applies a 
        transformation onto a state:
  
        \begin{list}{}{}
          \item S$^{''}$ = Res(T, S$^{'}$)
        \end{list}
  
        Disregarding the placeholder variables and a conjunction of atoms, 
        a transformation T in its simplest form:
  
        \begin{list}{}{}
          \item T causes A if B
        \end{list}
  
        Therefore, a transformation may be represented in an extended logic
        program as an implication:
 
        \begin{list}{}{}
          \item holds(A, Res(T, S)) $\leftarrow$ holds(B, S)
        \end{list}

        where S is the state before transformation T is applied.

      \subsubsection{Handling States in Constraints}

        Converting constraints into an extended logic program is somewhat
        simpler than converting transformations. Since constraints must hold
        for all states in the domain, the following constraints:

      \begin{list}{}{}
        \item A implies B with absence C
        \item A provokes B
        \item always A
      \end{list}

        can be expressed by the following implications:

        \begin{list}{}{}
          \item 
            holds(B, S) $\leftarrow$ holds(A, S) $\land$ not holds(C, S)
          \item 
            holds(B, S) $\leftarrow$ holds(A, S)
          \item 
            holds(A, S) $\leftarrow$
        \end{list}

        for all states S in the domain.

      \subsubsection{Inertial Rules}

        Intuitively, one would expect that after a transformation, every fact
        in the previous state that was not affected by the transformation
        should in fact be carried over into the next state. The following
        implications are added to ensure that this property is held:

        \begin{list}{}{}
          \item
            holds(A, Res(T, S)) $\leftarrow$ holds(A, S) $\land$ not $\lnot$ 
            holds(A, Res(T, S))
          \item 
            $\lnot$ holds(A, Res(T, S)) $\leftarrow$ $\lnot$ holds(A, S) 
            $\land$ not holds(A, Res(T, S))
        \end{list}

        $\forall$(S, T, A $\mid$ S $\in$ $\sigma$, T $\in$ $\theta$, A $\in$ 
        $\alpha$)

        where: 

        \begin{itemize}
          \item $\sigma$ = set of all states in the domain
          \item $\theta$ = set of all transformations in the domain
          \item $\alpha$ = set of all atoms in the domain
        \end{itemize}

      \subsubsection{Example}

        The following lists a small program written in the proposed language:

        \begin{verbatim}
initially holds(s, a, o1), !holds(s, a, o2), !holds(s, a, o3);
holds(s, a, o2) implies holds(s, a, o3);
trans causes holds(s, a, o3) if !holds(s, a, o2);
        \end{verbatim}

        First we represent the atoms with simple propositions:

        \begin{list}{}{}
          \item A = holds(s, a, o$_{1}$)
          \item B = holds(s, a, o$_{2}$)
          \item C = holds(s, a, o$_{3}$)
        \end{list}

        The equivalent extended logic program is:

        \begin{list}{}{Initial State}
          \item holds(A, S$_{0}$)
          \item $\lnot$ holds(B, S$_{0}$)
          \item $\lnot$ holds(C, S$_{0}$)
        \end{list}

        \begin{list}{}{Constraint}
          \item holds(C, S$_{0}$) $\leftarrow$ holds(B, S$_{0}$)
          \item holds(C, S$_{1}$) $\leftarrow$ holds(B, S$_{1}$)
        \end{list}

        \begin{list}{}{Transformation}
          \item holds(B, S$_{1}$) $\leftarrow$ $\lnot$ holds(B, S$_{0}$)
        \end{list}

        \begin{list}{}{Inertial Rules}
          \item
            holds(A, S$_{1}$) $\leftarrow$ holds(A, S$_{0}$) $\land$ not
            $\lnot$ holds(A, S$_{1}$)
          \item
            $\lnot$ holds(A, S$_{1}$) $\leftarrow$ $\lnot$ holds(A, S$_{0}$) 
            $\land$ not holds(A, S$_{1}$)
          \item
            holds(B, S$_{1}$) $\leftarrow$ holds(B, S$_{0}$) $\land$ not
            $\lnot$ holds(B, S$_{1}$)
          \item
            $\lnot$ holds(B, S$_{1}$) $\leftarrow$ $\lnot$ holds(B, S$_{0}$) 
            $\land$ not holds(B, S$_{1}$)
          \item
            holds(C, S$_{1}$) $\leftarrow$ holds(C, S$_{0}$) $\land$ not
            $\lnot$ holds(C, S$_{1}$)
          \item
            $\lnot$ holds(C, S$_{1}$) $\leftarrow$ $\lnot$ holds(C, S$_{0}$)
            $\land$ not holds(C, S$_{1}$)
        \end{list}

    \subsection{Convertion to Normal Logic Program: Removal of Classical Negation}

      Translation to from extended logic program to normal logic program 
      requires one more additional step: to remove all instances of classical 
      negation. 

    \pagebreak

  \section{System Structure}

    \subsection{Parser1: Epilog}

    \subsection{Parser2: Vlad}

    \subsection{Application Program}

    \pagebreak

  \section{Conclusion}

    \pagebreak

  \begin{thebibliography}{}
    \bibitem{BV1}Y. Bai and V. Varadharajan, \emph{On Transformation of Authorization Policies}. School of Computing and Information Technology, University of Western Sydney, 1997.
    \bibitem{BV2}Y. Bai and V. Varadharajan, \emph{On Sequence of Authorization Policy Transformations}. School of Computing and Information Technology, University of Western Sydney, 1997.
    \bibitem{GV}M. Gelfond and V. Lifschitz, \emph{Classical Negation in Logic Programs and Disjunctive Databases}. New Generation Computing, 1991.
    \bibitem{RN}S. Russell and P. Norvig, \emph{Artificial Intelligence - A Modern Approach}. Prentice Hall, 1995.
    \bibitem{WL}.Y.C. Woo and S.S. Lam, \emph{Authorization in Distributed systems: A Formal Approach}. Proceedings of IEEE Symposium on Research in Security and Privacy, 1992.
  \end{thebibliography}

\end{document}
