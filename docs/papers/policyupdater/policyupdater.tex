\documentclass[10pt, twocolumn]{article}
\usepackage{latex8}
\usepackage{times}

\begin{document}
  \title{
    PolicyUpdater -- A System for Dynamic Access Control: \\
    Formalization, Implementation and a Case Study
  }

  \author{
    Vino Fernando Crescini and Yan Zhang \\
    School of Computing and Information Technology \\
    University of Western Sydney \\
    Penrith South DC, NSW 1797, Australia \\
    E-mail: \{jcrescin,yan\}@cit.uws.edu.au
  }

  \date{}

  \maketitle

  \begin{abstract}
    [text]
  \end{abstract}

  \section{Introduction}

    \subsection{Overview of Logic-Based Approaches}

    \subsection{Contribution}

    \subsection{Structure of this Paper}

  \section{Language $\cal{L}$}

    \subsection{Syntax}

      \subsubsection{Components of Language $\cal{L}$}

        \noindent\underline{\emph{Identifiers.}} Language $\cal{L}$ consists of
        8 types of disjoint identifier types: $subjects$, $access-rights$ and
        $objects$, with each type being either singular or group, plus the
        the $policy update$ identifiers and the variables:

        \begin{itemize}
          \item
            Subjects: priviledge holders (eg. alice, bob, charlie)
          \item
            Access-Rights: priviledges (eg. read, write, execute)
          \item
            Objects: resources (eg. document, file)
          \item
            Subject Groups: sets of subjects (eg. students, astrologers)
          \item
            Access-Right Groups: sets of access-rights (eg. own = \{read,
            write, execute\})
          \item
            Objects Groups: sets of objects (eg. database, directory)
          \item
            Policy Update: modifies the policy base (eg. grant\_read,
            revoke\_execute)
          \item
            Variables: placeholder for subjects, access-rights, objects
            or groups of these.
        \end{itemize}

        The first character of an identifier byst be an alpha character
        followed by 0 to 127 characters of alpha, digit or underscore
        characters.

        \begin{verbatim}[a-zA-Z]([a_zA-Z0-9_]){0,127}\end{verbatim}

        \noindent\underline{\emph{Atoms.}} Atoms are composed of a relation
        plus two or more identifier constants to represent a logical
        relationship between the identifiers. There are three types of atoms:

        \begin{itemize}
          \item
            $Holds$. An atom of this type states that the subject
            $sub$ holds the access-right $acc$ for the object $obj$.

            \begin{verbatim}holds(<sub>, <acc>, <obj>)\end{verbatim}
          \item
            $Membership$. This type of atom states that the singular identifier
            $elt$ is a member or element of the group identifier $grp$. It is
            important to note that $elt$ and $grp$ must be of the same base
            type (eg. subject and subject group).

            \begin{verbatim}memb(<elt>, <grp>)\end{verbatim}
          \item
            $Subset$. The subset atom claims that the groups $grp1$ and $grp2$
            are of the same types and that group $grp1$ is a subset of the
            group $grp2$. 
            \begin{verbatim}subst(<grp1>, <grp2>)\end{verbatim}
        \end{itemize}

        \noindent\underline{\emph{Facts.}} A fact makes a claim that the
        relationship represented by an atom or its negation holds. Atoms
        are negated by the use of the negation operator ($!$).

        \begin{verbatim}[!]<holds-atm>|<memb-atm>|<subst-atm>\end{verbatim}

        \noindent\underline{\emph{Expressions.}} An expression is either a
        fact, or a logical conjunction of facts, separated by the
        double-ampersand characters $\&\&$.

        \begin{verbatim}<fact1> [&& <fact2> [&& ...]]\end{verbatim}

      \subsubsection{Initial State Definition}

      \subsubsection{Constraint Definition}

      \subsubsection{Policy Update Definition}

      \subsubsection{Policy Update Directives}

        \noindent\underline{\emph{Adding an Update into the Sequence}}

        \noindent\underline{\emph{Listing the Updates in the Sequence}}

        \noindent\underline{\emph{Removing an Update from the Sequence}}

      \subsubsection{Query Directive}

    \subsection{Semantics}

      \subsubsection{Formal Domain Definition}

        \noindent\underline{\emph{Sorts}}

        \noindent\underline{\emph{Initial State}}

        \noindent\underline{\emph{Constraints}}

        \noindent\underline{\emph{Policy Updates}}

        \noindent\underline{\emph{Queries}}

      \subsubsection{Additional Constraints}

        \noindent\underline{\emph{Inheritance Rules}}

        \noindent\underline{\emph{Transitivity Rules}}

        \noindent\underline{\emph{Contradictory Rule}}

          $false$ $\leftarrow$ $A$ $\land$ $\lnot$ $A$

    \subsection{Example}

    \subsection{Properties}

  \section{Implementation}

    \subsection{System Structure}

    \subsection{Algorithms}

  \section{Case Study: Web Server Application}

  \section{Conclusion}

    [summary]

    [effectiveness of system/application]

    [future work]

\end{document}
