\documentclass[11pt, twocolumn]{article}
\usepackage{latex8}
\usepackage{times}
\usepackage{amsmath}
\usepackage{alltt}
\usepackage{epsfig}
\usepackage{rotating}
\usepackage{multirow}

\newtheorem{vtheorm}{Theorem}
\newtheorem{vdefinition}{Definition}
\newtheorem{vexmpl}{Example}

% create a theorem environment
\newenvironment{vtheorem}[1]
  {\begin{vtheorm}{\bf (#1)\hspace{0.5em}}}
  {\end{vtheorm}}
% create an example environment with a 2mm X 2mm box at the end
\newenvironment{vexample}
  {\begin{vexmpl}\rm}
  {\rule{2mm}{2mm}\end{vexmpl}}
% create a quote environment without a right-hand margin
\newenvironment{vquote}
  {\begin{list}{}{\leftmargin 1em}\item[]}
  {\end{list}}
% create a *smaller* alltt (verbatim) environment
\newenvironment{vverbatim}
  {\begin{alltt}}
  {\vspace{-\baselineskip}\end{alltt}}
% define a section command to be used in the appendices
\makeatletter
\newcommand{\vappsection}[1]{
  % change the numbering to Appendix *level*
  \renewcommand{\@seccntformat}[1]{
    \appendixname\hspace{0.5em}\csname the##1\endcsname \hspace{1em}
  }
  \section{#1}
  % change the numbering back
  \renewcommand{\@seccntformat}[1]{
    \csname the##1\endcsname\hspace{1em}
  }
}
\makeatother

\begin{document}
  \title{PolicyUpdater -- A System for Dynamic Access Control}
  \author{
    Vino Fernando Crescini and Yan Zhang                   \\
    School of Computing and Information Technology         \\
    University of Western Sydney                           \\
    Penrith South DC, NSW 1797, Australia                  \\
    E-mail: \{jcrescin,yan\}@cit.uws.edu.au
  }

  \date{}

  \maketitle

  \begin{abstract}
    {\em PolicyUpdater}\footnotemark is a fully-implemented access control
    system that provides policy evaluations as well as dynamic policy updates.
    These functions are achieved by the use of a logic-based language
    ${\cal L}$ to represent the underlying access control policies, constraints
    and update propositions. The system performs authorisation query
    evaluations and conditional policy updates by first translating the
    language $\cal{L}$ to a normal logic program in a form suitable for
    evaluation using the {\em Stable Model} semantics.
  \end{abstract}

  \footnotetext{
    web page at \tt \scriptsize http://www.cit.uws.edu.au/\~{}jcrescin/
    projects/PolicyUpdater/index.html
  }

  \section{Introduction}

    The traditional access control mechanism is the {\em Access Control Matrix}
    where columns represent subjects, rows represent objects and each cell
    contains the access-rights of a subject over a particular object. However,
    flexibility and scalability issues arise when such method is used on
    real-world applications. A more effective paradigm of access control
    systems is the logic-based approach. In this approach, instead of
    explicitly defining all access-rights of all subjects for all objects
    in a domain, a set of logical facts and rules are used to define the
    policy base.

    Recent advances in the field have produced a number of different approaches
    to logic-based access control systems \cite{HAL,LI}. Bertino, et. al.
    \cite{BE1} proposed such a system based on ordered logic with ordered
    domains. Jajodia, et. al. \cite{JAJ} on the other hand, proposed a general
    access control framework that features handling of multiple policies.
    Another important work is the system proposed by Bai and Varadharajan
    \cite{BA1,BA2}. Their system's key characteristic is the ability to
    dynamically update the otherwise static policy base. These systems,
    effective as they are, lack the details necessary to address the issues
    involved in the implementation of such a system.

    The {\em Policy Description Language}, or {\cal PDL}, developed by Lobo,
    et. al. \cite{LOB}, is a language for representing event and action
    oriented generic policies. {\cal PDL} is later extended by Chomicki, et.
    al. \cite{CHO} to include {\em policy monitors} which, in effect, are
    policy constraints. Bertino, et. al. \cite{BE2}, again took ${\cal PDL}$ a
    step further by extending {\em policy monitors} to allow users to express
    preferred constraints. While these generic languages are expressive enough
    to be used for access control systems, systems built for such languages
    will not have the ability to dynamically update the policies.

    To overcome these limitations, we propose the PolicyUpdater access control
    system, which, with its own access control language, provides a formal
    logic-based representation of policies, with variable resolution and
    default propositions, a mechanism to conditionally and dynamically
    perform a sequence of policy updates, and a means of evaluating queries
    against the policies.

    In Section \ref{sec-langl}, the paper introduces language ${\cal L}$, with
    its formal syntax, semantics and some examples. The implementation as
    discussed in Section \ref{sec-implement} gives a brief overview of
    PolicyUpdater system as a whole, its internal and external components, and
    most importantly, an outline of the underlying mechanisms responsible for
    processing of language ${\cal L}$ policies for query evaluation and dynamic
    updates. In Section \ref{sec-conclusion}, the concluding remarks addresses
    some possible extensions to this system. The case study presented in
    Appendix \ref{app-case} shows a typical application of the PolicyUpdater
    system: an access control system for web servers. Appendix \ref{app-store}
    shows detailed specifications of the data structures used by the system.

  \section{Language $\cal{L}$}
    \label{sec-langl}

    Language $\cal{L}$ is a first-order logic language that represents a policy
    base for an authorisation system. Two key features of the language are: (1)
    providing a means to conditionally and dynamically update the existing
    policy base and (2) having a mechanism by which queries may be evaluated
    from the updated policy base.

    \subsection{Syntax}
      \label{subsec-syntax}

      Logic programs of language ${\cal L}$ are composed of language
      statements, each terminated by a semicolon ";" character. C-style
      comments delimited by the "/*" and "*/" characters may appear anywhere in
      the logic program.

      \subsubsection{Components of Language $\cal{L}$}

        \paragraph{Identifiers.}
          The most basic unit of language $\cal{L}$ is the identifier.
          Identifiers are used to represent the different components of the
          language, and are divided into three main classes:

          \begin{itemize}
            \item
              {\em Entity Identifiers} represent constant entities that make up
              a logical atom. They are divided further into three types, with
              each type again divided into the {\em singular entity} and
              {\em group entity} categories:

              \begin{itemize}
                \item
                  {\em Subjects}: e.g. alice, lecturers, group.
                \item
                  {\em Access Rights}: e.g. read, write, own.
                \item
                  {\em Objects}: e.g. file, database, directory.
              \end{itemize}

              An entity identifier is defined as a lower-case alphabet
              character, followed by 0 to 127 characters of alphabet, digit or
              underscore characters. The following regular expression shows the
              syntax of entity identifiers:

              \begin{vverbatim}
[a-z]([a-zA-Z0-9\_]){0,127}
              \end{vverbatim}

            \item
              {\em Policy Update Identifiers} are used for the sole purpose of
              naming a policy update. These identifier names are then used as
              labels to refer to policy update definitions and directives. As
              labels, identifiers of this class occupy a different namespace
              from entity identifiers. For this reason, policy update
              identifiers share the same syntax with entity identifiers:


              \begin{vverbatim}
[a-z]([a-zA-Z0-9\_]){0,127}
              \end{vverbatim}

            \item
              {\em Variable Identifiers} are used as entity identifier
              place-holders. To distinguish them from entity and policy update
              identifiers, variable identifiers are prefixed with an upper-case
              character, followed by 0 to 127 alphanumeric and underscore
              characters. The following regular expression shows the syntax of
              variable identifiers:

              \begin{vverbatim}
[A-Z]([a-zA-Z0-9\_]){0,127}
              \end{vverbatim}
          \end{itemize}

        \paragraph{Atoms.}
          An atom is composed of a relation with 2 to 3 entity or variable
          identifiers that represent a logical relationship between the
          entities. There are three types of atoms:

          \begin{itemize}
            \item
              {\em Holds.} An atom of this type states that the subject
              identifier $sub$ holds the access right identifier $acc$
              for the object identifier $obj$.

              \begin{vverbatim}
holds(<sub>, <acc>, <obj>)
              \end{vverbatim}
            \item
              {\em Membership.} This type of atom states that the singular
              identifier $elt$ is a member or element of the group identifier
              $grp$. It is important to note that identifiers $elt$ and $grp$
              must be of the same base type (e.g. subject and subject group).

              \begin{vverbatim}
memb(<elt>, <grp>)
              \end{vverbatim}
            \item
              {\em Subset.} The subset atom states that the group identifiers
              $grp1$ and $grp2$ are of the same types and that group $grp1$ is
              a subset of the group $grp2$.

              \begin{vverbatim}
subst(<grp1>, <grp2>)
              \end{vverbatim}
          \end{itemize}

          Atoms that contain no variables, i.e. composed entirely of entity
          identifiers, are called {\em ground atoms}.

        \paragraph{Facts.}
          A fact makes a claim that the relationship represented by an atom or
          its negation holds in the current context. Facts are negated by the
          use of the negation operator ($!$). The following shows the formal
          syntax of a fact:

          \begin{vverbatim}
[!]<holds\_atom>|<memb\_atom>|
   <subst\_atom>
          \end{vverbatim}

          Note that facts may be made up of atoms that contain variable
          identifiers. Facts with no variable occurrences are called
          {\em ground facts}.

        \paragraph{Expressions.}
          An expression is either a fact, or a logical conjunction of facts,
          separated by the double-ampersand characters $\&\&$.

          \begin{vverbatim}
<fact1> [&& <fact2> [&& ...]]
          \end{vverbatim}

          Expressions that are made up of only ground facts are called
          {\em ground expressions}.

      \subsubsection{Definition Statements}

        \paragraph{Entity Identifier Definition.}

          All entity identifiers (subjects, access rights, objects and groups)
          must first be declared before any other statements to define the
          entity domain of the policy base. The following entity declaration
          syntax illustrates how to define one or more entity identifiers of a
          particular type.

          \begin{vverbatim}
ident sub|acc|obj[-grp]
  <entity\_id>[, ...];
          \end{vverbatim}

        \paragraph{Initial Fact Definition.}

          The initial facts of the policy base, those that hold before any
          policy updates are performed, are defined by using the following
          definition syntax:

          \begin{vverbatim}
initially <ground-exp>;
          \end{vverbatim}

        \paragraph{Constraint Definition.}

          Constraints are logical rules that hold regardless of any changes
          that may occur when the policy base is updated. The constraint rules
          are true in the initial state and remain true after any policy
          update.

          The constraint syntax below shows that for any state of the policy
          base, expression $exp1$ holds if expression $exp2$ is true and there
          is no evidence that $exp3$ is true. The $with$ $absence$ clause
          allows constraints to behave like default propositions, where the
          absence of proof that an expression holds  satisfies the clause
          condition of the proposition.

          It is important to note that the expressions $exp1$, $exp2$ and
          $exp3$ are non-ground expressions, which means identifiers within
          them may be variables.

          \begin{vverbatim}
always <exp1>
  [implied by <exp2>
  [with absence <exp3>]];
          \end{vverbatim}

        \paragraph{Policy Update Definition.}

          Before a policy update can be applied, it must first be defined by
          using the following syntax:

          \begin{vverbatim}
<up\_id>([<var\_id>[, ...]])
  causes <exp1>
  if <exp2>;
          \end{vverbatim}

          $up\_id$ is the policy update identifier to be used in referencing
          this policy update. The optional $var\_id$ list are the variable
          identifiers occurring in the expressions $exp1$ and $exp2$ and will
          eventually be replaced by entity identifiers when the update is
          referenced. The postcondition expression $exp1$ is an expression that
          will hold in the state after this update is applied. The expression
          $exp2$ is a precondition expression that must hold in the current
          state before this update is applied.

          It is important to note that a policy update definition will have no
          effect on the policy base until it is applied by one of the directives
          described in the following section.

        \subsubsection{Directive Statements}

        \paragraph{Policy Update Directives.}

        The policy update sequence list contains a list of references to
        defined policy updates in the domain. The policy updates in the
        sequence list are applied to the current state of the policy base one
        at a time to produce a policy base state upon which queries can be
        evaluated.

        The following four directives are the policy sequence manipulation
        features of language $\cal{L}$.

        \subparagraph{Adding an update into the sequence.}
          Defined policy updates are added into the sequence list through the
          use of the following directive:

          \begin{vverbatim}
seq add <up\_id>([<e\_id>[, ...]]);
          \end{vverbatim}

          \noindent where $up\_id$ is the identifier of a defined policy
          update and the $e\_id$ list is a comma-separated list of entity
          identifiers that will replace the variable identifiers that occur in
          the definition of the policy update.

        \subparagraph{Listing the updates in the sequence.}
          The following directive may be used to list the current contents of
          the policy update sequence list.

          \begin{vverbatim}
seq list;
          \end{vverbatim}

          This directive is answered with an ordinal list of policy updates in
          the form:

          \begin{vverbatim}
<n> <up\_id>([e\_id[, ...]])
          \end{vverbatim}

          \noindent where $n$ is the ordinal index of the policy update within
          the sequence list starting at 0. $up\_id$ is the policy update
          identifier and the $e\_id$ list is the list of entity identifiers
          used to replace the variable identifier place-holders.

        \subparagraph{Removing an update from the sequence.}
          The syntax below shows the directive to remove a policy update
          reference from the list. $n$ is the ordinal index of the policy
          update to be removed. Note that removing a policy update reference
          from the sequence list may change the ordinal index of other update
          references.

          \begin{vverbatim}
seq del <n>;
          \end{vverbatim}

        \subparagraph{Computing an update sequence.}

          The policy updates in the sequence list does not get applied until
          the $compute$ directive is issued. The directive causes the policy
          update references in the sequence list to be applied one at a time in
          the same order that they appear in the list. The directive also
          causes the system to generate the policy base models against which
          query requests can be evaluated.

          \begin{vverbatim}
compute;
          \end{vverbatim}

        \paragraph{Query Directive.}

          A ground query expression may be issued against the current state of
          the policy base. This current state is derived after all the updates
          in the update sequence have been applied, one at a time, upon the
          initial state. Query expressions are answered with a $true$, $false$
          or an $unknown$, depending on whether the queried expression holds,
          its negation holds, or neither, respectively. Syntax is as follows:

          \begin{vverbatim}
query <ground-exp>;
          \end{vverbatim}

        \begin{vexample}
          \label{ex-1}
          The following language ${\cal L}$ program code listing shows a simple
          rule-based document access control system scenario.

          In this example, the subject $alice$ is initially a member of the
          subject group $grp3$, which is a subset of group $grp2$, which in
          turn is a subset of group $grp1$. The group $grp1$ also initially
          holds a $read$ access right for the object $file$. The constraint
          states that if the group $grp1$ has $read$ access for $file$, and no
          other information is present to conclude that $grp1$ do not have
          $write$ access for $file$, then the group $grp1$ is granted $write$
          access for $file$.

          \begin{vverbatim}
ident sub alice;
ident sub-grp grp1, grp2, grp3;
ident acc read, write;
ident obj file;

initially
  memb(alice, grp3) &&
  holds(grp1, read, file) &&
  subst(grp3, grp2) &&
  subst(grp2, grp1);

always holds(grp1, write, file)
  implied by
    holds(grp1, read, file)
  with absence
    !holds(grp1, write, file);

delete\_read(SG0, OS0)
  causes !holds(SG0, read, OS0);

seq add delete\_read(grp1, file);

compute;

query holds(grp1, write, file);
query holds(grp1, read, file);
query holds(alice, write, file);
query holds(alice, read, file);
          \end{vverbatim}
        \end{vexample}

    \subsection{Semantics}
      \label{subsec-semantics}

      \subsubsection{Domain Description of Language ${\cal L}$}

        \begin{vdefinition}
          \label{def-domain}
          The domain description ${\cal D}_{\cal L}$ of language ${\cal L}$ is
          defined as a finite set of ground initial state facts, constraint
          rules and policy update definitions.
        \end{vdefinition}

        In addition to the domain description ${\cal D}_{\cal L}$, language
        ${\cal L}$ also includes an additional ordered set: the sequence list
        $\psi$. The sequence list $\psi$ is an ordered set that contains a
        sequence of references to policy update definitions. Each policy update
        reference consists of the policy update identifier and a series of zero
        or more identifier entities to replace the variable place-holders in
        the policy update definitions.

      \subsubsection{Language ${\cal L}^{*}$}

        In language ${\cal L}$, the policy base is subject to change, which is
        triggered by the application of policy updates. Such changes bring
        forth the concept of policy base states. Conceptually, a state may be
        thought of as a set of facts and constraints of the policy base at a
        particular instant. The state transition notation below shows that a
        new state $PB'$ is generated from the current state $PB$ after the
        policy update $u$ is applied.

        \begin{vquote}
          $PB$ $\overrightarrow{_{u}}$ $PB'$
        \end{vquote}

        This definition of a state means that for every policy update applied
        to the policy base, a new instance of the policy base or a new set of
        facts and constraints are generated. To precisely define the underlying
        semantics of domain description ${\cal D}_{\cal L}$ in language
        ${\cal L}$, we introduce language ${\cal L}^{*}$, which is an extended
        logic program representation of language ${\cal L}$, with state as an
        explicit sort.

        Language ${\cal L}^{*}$ contains only one special state constant
        $S_{0}$ to represent the initial state of a given domain description.
        All other states are represented as a resulting state obtained by
        applying the $Result$ function.

        The $Result$ function takes a policy update reference $u$, where $u$
        $\in$ $\psi$, and the current state $\sigma$ as input arguments and
        returns the resulting state $\sigma'$ after update $u$ has been applied
        to state $\sigma$:

        \begin{vquote}
          $\sigma'$ = $Result$($u$, $\sigma$)
        \end{vquote}

        Given an initial state $S_{0}$ and a sequence list $\psi$, each state
        $\sigma_{i}$ ($0$ $\leq$ $i$ $\leq$ $|\psi|$) may be represented as
        follows:

        \begin{vquote}
          $\sigma_{0} = S_{0}$

          $\sigma_{1} = Result(u_{0}, \sigma_{0})$

          $\vdots$

          $\sigma_{|\psi|} = Result(u_{|\psi| - 1}, \sigma_{|\psi| - 1})$
        \end{vquote}

        Substituting each state with a recursive call to the $Result$ function,
        the final state $S_{|\psi|}$ is defined as follows:

        \begin{vquote}
          \begin{math}
            \begin{aligned}[t]
              S_{|\psi|} = & Result(u_{|\psi| - 1}, Result(\ldots, \\
              & Result(u_{0}, S_{0})))
            \end{aligned}
          \end{math}
        \end{vquote}

        \paragraph{Entities.}

          The entity set ${\cal E}$ is the union of six disjoint entity sets:
          single subject ${\cal E}_{ss}$, group subject ${\cal E}_{sg}$,
          single access right ${\cal E}_{as}$, group access right
          ${\cal E}_{ag}$, single object ${\cal E}_{os}$ and group object
          ${\cal E}_{og}$. Each entity in set ${\cal E}$ corresponds directly
          to the {\em entity identifiers} of language ${\cal L}$.

          \begin{itemize}
            \item
              ${\cal E} = {\cal E}_{s} \cup {\cal E}_{a} \cup {\cal E}_{o}$
            \item
              ${\cal E}_{s} = {\cal E}_{ss} \cup {\cal E}_{sg}$
            \item
              ${\cal E}_{a} = {\cal E}_{as} \cup {\cal E}_{ag}$
            \item
              ${\cal E}_{o} = {\cal E}_{os} \cup {\cal E}_{og}$
          \end{itemize}

        \paragraph{Atoms.}

          The main difference between language ${\cal L}$ and language
          ${\cal L}^{*}$ lies in the definition of an atom. Atoms in language
          ${\cal L}^{*}$ represent a logical relationship of two to three
          entities, as with atoms of language ${\cal L}$. Furthermore, atoms of
          language ${\cal L}^{*}$ extends this definition by defining the
          state of the policy base in which the relationship holds. In this
          paper, atoms of language ${\cal L}^{*}$ are written with the
          hat character ($\hat{holds}$, $\hat{memb}$ and $\hat{subst}$) to
          differentiate from the atoms of language ${\cal L}$.

          The atom set ${\cal A}^{\sigma}$ is the set of all atoms in state
          $\sigma$.

          \begin{itemize}
            \item
              ${\cal A}^{\sigma} = {\cal A}^{\sigma}_{h} \cup {\cal A}^{\sigma}_{m} \cup {\cal A}^{\sigma}_{s}$
            \item
              \begin{math}
                \begin{aligned}[t]
                  {\cal A}^{\sigma}_{h} =&\{\forall (s \in {\cal E}_{s}, a \in {\cal E}_{a}, o \in {\cal E}_{o}),\\
                  &\hat{holds}(s, a, o, \sigma) \}
                \end{aligned}
              \end{math}
            \item
              ${\cal A}^{\sigma}_{m} = {\cal A}^{\sigma}_{ms} \cup {\cal A}^{\sigma}_{ma} \cup {\cal A}^{\sigma}_{mo}$
            \item
              ${\cal A}^{\sigma}_{s} = {\cal A}^{\sigma}_{ss} \cup {\cal A}^{\sigma}_{sa} \cup {\cal A}^{\sigma}_{so}$
            \item
              ${\cal A}^{\sigma}_{ms} = \{\forall (e \in {\cal E}_{ss}, g \in {\cal E}_{sg}), \hat{memb}(e, g, \sigma)\}$
            \item
              ${\cal A}^{\sigma}_{ma} = \{\forall (e \in {\cal E}_{as}, g \in {\cal E}_{ag}), \hat{memb}(e, g, \sigma)\}$
            \item
              ${\cal A}^{\sigma}_{mo} = \{\forall (e \in {\cal E}_{os}, g \in {\cal E}_{og}), \hat{memb}(e, g, \sigma)\}$
            \item
              ${\cal A}^{\sigma}_{ss} = \{\forall (g1, g2 \in {\cal E}_{sg}), \hat{subst}(g1, g2, \sigma)\}$
            \item
              ${\cal A}^{\sigma}_{sa} = \{\forall (g1, g2 \in {\cal E}_{ag}), \hat{subst}(g1, g2, \sigma)\}$
            \item
              ${\cal A}^{\sigma}_{so} = \{\forall (g1, g2 \in {\cal E}_{og}), \hat{subst}(g1, g2, \sigma)\}$
          \end{itemize}

        \paragraph{Facts.}

          A fact is a logical statement that makes a claim that an atom either
          holds or does not hold at a particular state. The following is the
          formal definition of fact $f$ in state $\sigma$:

          \begin{vquote}
            $f^{\sigma}$ = $[\lnot]$$\alpha$, $\alpha$ $\in$ ${\cal A}^{\sigma}$
          \end{vquote}

      \subsubsection{Translating Language ${\cal L}$ to Language ${\cal L^{*}}$}

        \begin{vdefinition}
          \label{def-cons}
          Given a domain description ${\cal D}_{\cal L}$ of language
          ${\cal L}$, $Trans$(${\cal D}_{\cal L}$) is defined as the
          extended logic program ${\cal L}^{*}$ translation of language
          ${\cal L}$. The domain description ${\cal D}_{\cal L}$ is said to be
          {\em consistent} if and only if $Trans$(${\cal D}_{\cal L}$) has
          a consistent answer set.
        \end{vdefinition}

        Before we can fully define $Trans$(${\cal D}_{\cal L}$), we should
        first define the following functions:

        \paragraph{}

          The $CopyAtom$ function takes two arguments: an atom $\hat{\alpha}$
          of language ${\cal L}^{*}$ at some state $\sigma$ and another state
          $\sigma'$. The function returns an equivalent atom of the same type
          and with the same entities, but in the new state specified.

          \begin{vquote}
            $CopyAtom$($\hat{\alpha}$, $\sigma'$)

            =
            \begin{math}
              \begin{cases}
                \mbox{$\hat{holds}$($s$, $a$, $o$, $\sigma'$), if $\hat{\alpha}$ = $\hat{holds}$($s$, $a$, $o$, $\sigma$)} \\
                \mbox{$\hat{memb}$($e$, $g$, $\sigma'$), if $\hat{\alpha}$ = $\hat{memb}$($e$, $g$, $\sigma$)} \\
                \mbox{$\hat{subst}$($g_{1}$, $g_{2}$, $\sigma'$), if $\hat{\alpha}$ = $\hat{subst}$($g_{1}$, $g_{2}$, $\sigma$)}
              \end{cases}
            \end{math}
          \end{vquote}

        \paragraph{}

          Another function, $TransAtom$, takes an atom $\alpha$ of language
          ${\cal L}$ and an arbitrary state $\sigma$ and returns the equivalent
          atom of language ${\cal L}^{*}$.

          \begin{vquote}
            $TransAtom$($\alpha$, $\sigma$)

            =
            \begin{math}
              \begin{cases}
                \mbox{$\hat{holds}$($s$, $a$, $o$, $\sigma$), if $\alpha$ = $holds$($s$, $a$, $o$)} \\
                \mbox{$\hat{memb}$($e$, $g$, $\sigma$), if $\alpha$ = $memb$($e$, $g$)} \\
                \mbox{$\hat{subst}$($g_{1}$, $g_{2}$, $\sigma$), if $\alpha$ = $subst$($g_{1}$, $g_{2}$)}
              \end{cases}
            \end{math}
          \end{vquote}

        \paragraph{}

          The other function, $TransFact$, is similar to the $TransAtom$
          function, but instead of translating an atom, it takes a fact
          from language ${\cal L}$ and a state then returns the equivalent
          fact in language ${\cal L}^{*}$.

        \paragraph{Initial Fact Expression.}

          Translating initial fact expressions of language ${\cal L}$ to
          language ${\cal L}^{*}$ is a trivial procedure: translate each fact
          that make up the initial fact expression of language ${\cal L}$
          with its corresponding equivalent initial state atom of language
          ${\cal L}^{*}$.

          Given an initial fact expression of language ${\cal L}$ that is
          composed of $n$ facts:

          \begin{vverbatim}
\(f\sb{0}\) && \ldots && \(f\sb{n}\)
          \end{vverbatim}

          In language ${\cal L}^{*}$, the statement above is represented as:

          \begin{vquote}
            $\hat{f_{0}}$ $\leftarrow$

            $\vdots$

            $\hat{f_{n}}$ $\leftarrow$

            where $\hat{f_{i}}$ $=$ $TransFact$($f_{i}$, $S_{0}$),
            $0$ $\leq$ $i$ $\leq$ $n$
          \end{vquote}

          The following code below shows an example of language ${\cal L}$
          $initially$ statements:

          \begin{vverbatim}
initially
  holds(admins, exec, sys_tools);
initially
  holds(admins, read, sys_data);
initially
  memb(alice, admins) &&
  memb(bob, admins);
          \end{vverbatim}

        In language ${\cal L}^{*}$, the above statements are translated to:

        \begin{vquote}
          $\hat{holds}$($admins$, $exec$, $system\_tools$, $S_{0}$) $\leftarrow$

          $\hat{holds}$($admins$, $read$, $system\_data$, $S_{0}$) $\leftarrow$

          $\hat{memb}$($alice$, $admins$, $S_{0}$) $\leftarrow$

          $\hat{memb}$($bob$, $admins$, $S_{0}$) $\leftarrow$
        \end{vquote}

        \paragraph{Constraint Rules.}

          Each constraint rule in language ${\cal L}$ is expressed as a series
          of logical rules in language ${\cal L}^{*}$. Given that all variable
          occurrences have been grounded to entity identifiers, a constraint in
          language ${\cal L}$, with $m$, $n$, $o$ $\geq$ $0$ may be represented
          as:

          \begin{vverbatim}
always \(a\sb{0}\) && \ldots && \(a\sb{m}\)
  implied by \(b\sb{0}\) && \ldots && \(b\sb{n}\)
  with absence \(c\sb{0}\) && \ldots && \(c\sb{o}\)
          \end{vverbatim}

          Each fact in the $always$ clause of language ${\cal L}$ corresponds
          to a new rule, where it is the consequent. Each of these new rules
          will have expression $b$ in the $implied$ $by$ clause as the positive
          premise and the expression $c$ in the $with$ $absence$ clause as the
          negative premise.

          \begin{vquote}
            $a_{0}$ $\leftarrow$
            $b_{0}$, \ldots, $b_{n}$,
            $not$ $c_{0}$, \ldots, $not$ $c_{o}$

            $\vdots$

            $a_{m}$ $\leftarrow$
            $b_{0}$, \ldots, $b_{n}$,
            $not$ $c_{0}$, \ldots, $not$ $c_{o}$
          \end{vquote}

          Under the definition of constraints, each of the rules above must be
          made to hold in all states as defined by the sequence list $\psi$.
          This can be accomplished by translating each of the rules above to
          a set of $|\psi|$ rules, one for each state.

           \begin{vquote}
            $\hat{a}^{S_{0}}_{0}$ $\leftarrow$
            $\hat{b}^{S_{0}}_{0}$, \ldots, $\hat{b}^{S_{0}}_{n}$,
            $not$ $\hat{c}^{S_{0}}_{0}$, \ldots, $not$ $\hat{c}^{S_{0}}_{o}$

            $\vdots$

            $\hat{a}^{S_{|\psi|}}_{0}$ $\leftarrow$
            $\hat{b}^{S_{|\psi|}}_{0}$, \ldots, $\hat{b}^{S_{|\psi|}}_{n}$,
            $not$ $\hat{c}^{S_{|\psi|}}_{0}$, \ldots, $not$ $\hat{c}^{S_{|\psi|}}_{o}$

            $\vdots$

            $\hat{a}^{S_{0}}_{m}$ $\leftarrow$
            $\hat{b}^{S_{0}}_{0}$, \ldots, $\hat{b}^{S_{0}}_{n}$,
            $not$ $\hat{c}^{S_{0}}_{0}$, \ldots, $not$ $\hat{c}^{S_{0}}_{o}$

            $\vdots$

            $\hat{a}^{S_{|\psi|}}_{m}$ $\leftarrow$
            $\hat{b}^{S_{|\psi|}}_{0}$, \ldots, $\hat{b}^{S_{|\psi|}}_{n}$,
            $not$ $\hat{c}^{S_{|\psi|}}_{0}$, \ldots, $not$ $\hat{c}^{S_{|\psi|}}_{o}$

            where

            $\hat{a}^{\sigma}_{i}$ $=$ $TransFact$($a_{i}$, $\sigma$),
            $0$ $\leq$ $i$ $\leq$ $m$,

            $\hat{b}^{\sigma}_{j}$ $=$ $TransFact$($b_{j}$, $\sigma$),
            $0$ $\leq$ $j$ $\leq$ $n$,

            $\hat{c}^{\sigma}_{k}$ $=$ $TransFact$($c_{k}$, $\sigma$),
            $0$ $\leq$ $k$ $\leq$ $o$,

            $S_{0}$ $\leq$ $\sigma$ $\leq$ $S_{|\psi|}$
          \end{vquote}

          The example below shows how the following language ${\cal L}$ code
          fragment is translated to language ${\cal L}^{*}$:

          \begin{vverbatim}
always
  holds(alice, read, secret_data) &&
  holds(alice, write, secret_data)
implied by
  memb(alice, admin)
with absence
  !holds(alice, read, secret_data);
          \end{vverbatim}

          Given a policy update reference in the sequence list $\psi$ (i.e.
          $|\psi|$ $=$ $1$), the language ${\cal L}^{*}$ equivalent is as
          follows:

          \begin{vquote}
            \begin{math}
              \begin{aligned}[t]
                \hat{holds}&(alice, read, secret\_data, S_{0}) \leftarrow \\
                 & \hat{memb}(alice, admin, S_{0}), \\
                 & not \lnot \hat{holds}(alice, read, secret\_data, S_{0})
              \end{aligned}
            \end{math}

            \begin{math}
              \begin{aligned}[t]
                \hat{holds}&(alice, write, secret\_data, S_{0}) \leftarrow \\
                 & \hat{memb}(alice, admin, S_{0}), \\
                 & not \lnot \hat{holds}(alice, read, secret\_data, S_{0})
              \end{aligned}
            \end{math}

            \begin{math}
              \begin{aligned}[t]
                \hat{holds}&(alice, read, secret\_data, S_{1}) \leftarrow \\
                 & \hat{memb}(alice, admin, S_{1}), \\
                 & not \lnot \hat{holds}(alice, read, secret\_data, S_{1})
              \end{aligned}
            \end{math}

            \begin{math}
              \begin{aligned}[t]
                \hat{holds}&(alice, write, secret\_data, S_{1}) \leftarrow \\
                 & \hat{memb}(alice, admin, S_{1}), \\
                 & not \lnot \hat{holds}(alice, read, secret\_data, S_{1})
              \end{aligned}
            \end{math}
          \end{vquote}

        \paragraph{Policy Updates.}

          With all occurrences of variable place-holders grounded to entity
          identifiers, a policy update $u$, as defined by language ${\cal L}$
          is in the form:

          \begin{vquote}
            $u$ causes $a_{0}$ \&\& \ldots \&\& $a_{m}$
            if $b_{0}$ \&\& \ldots \&\& $b_{n}$
          \end{vquote}

          In language ${\cal L}^{*}$, such policy updates may be represented as
          a set of implications, with each fact $a$ in the postcondition
          expression as the consequent and precondition expression $b$ as the
          premise. However, the translation process must also take into account
          that the premise of the implication holds in the state before the
          policy update is applied and that the consequent holds in the state
          after the application.

          \begin{vquote}
            $\hat{a}_{0}$ $\leftarrow$ $\hat{b}_{0}$, \ldots, $\hat{b}_{n}$

            $\vdots$

            $\hat{a}_{m}$ $\leftarrow$ $\hat{b}_{0}$, \ldots, $\hat{b}_{n}$

            where

            $\hat{a}_{i}$ $=$ $TransFact$($a_{i}$, $Result$($u$, $\sigma$)),
            $0$ $\leq$ $i$ $\leq$ $m$,

            $\hat{b}_{j}$ $=$ $TransFact$($b_{j}$, $\sigma$),
            $0$ $\leq$ $j$ $\leq$ $n$
          \end{vquote}

          For example, given the following 2 language ${\cal L}$ policy update
          definitions:

          \begin{vverbatim}
grant\_read()
  causes
    holds(alice, read, file)
  if
    memb(alice, readers);

grant\_write()
  causes
    holds(alice, write, file)
  if
    memb(alice, writers);
          \end{vverbatim}

          Given the update sequence list $\psi$ $=$
          \{$grant\_read$, $grant\_write$\}, the above statements are written
          in language ${\cal L}^{*}$ as:

          \begin{vquote}
            \begin{math}
              \begin{aligned}[t]
                \hat{holds}&(alice, read, file, S_{1}) \leftarrow \\
                & \hat{memb}(alice, readers, S_{0})
              \end{aligned}
            \end{math}

            \begin{math}
              \begin{aligned}[t]
                \hat{holds}&(alice, write, file, S_{2}) \leftarrow \\
                & \hat{memb}(alice, writers, S_{1})
              \end{aligned}
            \end{math}
          \end{vquote}

        \paragraph{Additional Constraints.}

          In addition to the above translations, there are a few other implicit
          constraint rules implied by language ${\cal L}$ that needs to be
          explicitly defined in language ${\cal L}^{*}$.

          \subparagraph{Inheritance rules.}

            All properties held by a group is inherited by all the members and
            subsets of that group. This rule is easy to apply for subject group
            entities. However, careful attention must be given to access right
            and object groups. A subject holding an access right for an object
            group implies that that subject also holds that access right for
            all objects in the object group. Similarly, a subject holding an
            access right group for a particular object implies that the subject
            holds all access rights contained in the access right group for
            that object.

            A conflict is encountered when a particular property is to be
            inherited by an entity from a group of which it is a member or
            subset, and the contained entity already holds the negation of
            that property. This conflict is resolved by giving negative facts
            higher precedence over its positive counterpart: by allowing member
            or subset entities to inherit its parent group's properties only if
            the entities do not already hold the negation of those properties.

            The following are the inheritance constraint rules to allow the
            properties held by a group to propagate to its members and
            subsets that do not already hold the negation of the properties.

            \begin{enumerate}
              \item
                Subject Group Membership Inheritance

                $\forall$ ($s_{s}$, $s_{g}$, $a$, $o$, $\sigma$), where

                $s_{s} \in {\cal E}_{ss}$,
                $s_{g} \in {\cal E}_{sg}$,
                $a \in {\cal E}_{a}$,
                $o \in {\cal E}_{o}$,
                $S_{0} \leq \sigma \leq S_{|\psi|}$

                \begin{math}
                  \begin{aligned}[t]
                    \hat{holds}(s_{s}, a, o, \sigma) \leftarrow &
                    \hat{holds}(s_{g}, a, o, \sigma), \\
                    & \hat{memb}(s_{s}, s_{g}, \sigma), \\
                    & not \lnot \hat{holds}(s_{s}, a, o, \sigma) \\
                    \lnot \hat{holds}(s_{s}, a, o, \sigma) \leftarrow &
                    \lnot \hat{holds}(s_{g}, a, o, \sigma), \\
                    & \hat{memb}(s_{s}, s_{g}, \sigma)
                  \end{aligned}
                \end{math}

              \item
                Access Right Group Membership Inheritance

                $\forall$ ($s$, $a_{s}$, $a_{g}$, $o$, $\sigma$), where

                $s \in {\cal E}_{s}$,
                $a_{s} \in {\cal E}_{as}$,
                $a_{g} \in {\cal E}_{ag}$,
                $o \in {\cal E}_{o}$,
                $S_{0} \leq \sigma \leq S_{|\psi|}$

                \begin{math}
                  \begin{aligned}[t]
                    \hat{holds}(s, a_{s}, o, \sigma) \leftarrow &
                    \hat{holds}(s, a_{g}, o, \sigma), \\
                    & \hat{memb}(a_{s}, a_{g}, \sigma), \\
                    & not \lnot \hat{holds}(s, a_{s}, o, \sigma) \\
                    \lnot \hat{holds}(s, a_{s}, o, \sigma) \leftarrow &
                    \lnot \hat{holds}(s, a_{g}, o, \sigma), \\
                    & \hat{memb}(a_{s}, a_{g}, \sigma)
                  \end{aligned}
                \end{math}

              \item
                Object Group Membership Inheritance

                $\forall$ ($s$, $a$, $o_{s}$, $o_{g}$, $\sigma$) where

                $s \in {\cal E}_{s}$,
                $a \in {\cal E}_{a}$,
                $o_{s} \in {\cal E}_{os}$,
                $o_{g} \in {\cal E}_{og}$,
                $S_{0} \leq \sigma \leq S_{|\psi|}$

                \begin{math}
                  \begin{aligned}[t]
                    \hat{holds}(s, a, o_{s}, \sigma) \leftarrow &
                    \hat{holds}(s, a, o_{g}, \sigma), \\
                    & \hat{memb}(o_{s}, o_{g}, \sigma), \\
                    & not \lnot \hat{holds}(s, a, o_{s}, \sigma) \\
                    \lnot \hat{holds}(s, a, o_{s}, \sigma) \leftarrow &
                    \lnot \hat{holds}(s, a, o_{g}, \sigma), \\
                    & \hat{memb}(o_{s}, o_{g}, \sigma)
                  \end{aligned}
                \end{math}

              \item
                Subject Group Subset Inheritance

                $\forall$ ($s_{g1}$, $s_{g2}$, $a$, $o$, $\sigma$), where

                $s_{g1}, s_{g2} \in {\cal E}_{sg}$,
                $a \in {\cal E}_{a}$,
                $o \in {\cal E}_{o}$,
                $S_{0} \leq \sigma \leq S_{|\psi|}$

                \begin{math}
                  \begin{aligned}[t]
                    \hat{holds}(s_{g1}, a, o, \sigma) \leftarrow &
                    \hat{holds}(s_{g2}, a, o, \sigma), \\
                    & \hat{subst}(s_{g1}, s_{g2}, \sigma), \\
                    & not \lnot \hat{holds}(s_{g1}, a, o, \sigma) \\
                    \lnot \hat{holds}(s_{g1}, a, o, \sigma) \leftarrow &
                    \lnot \hat{holds}(s_{g2}, a, o, \sigma), \\
                    & \hat{subst}(s_{g1}, s_{g2}, \sigma)
                  \end{aligned}
                \end{math}

              \item
                Access Right Group Subset Inheritance

                $\forall$ ($s$, $a_{g1}$, $a_{g2}$, $o$, $\sigma$), where

                $s \in {\cal E}_{s}$,
                $a_{g1}, a_{g2} \in {\cal E}_{ag}$,
                $o \in {\cal E}_{o}$,
                $S_{0} \leq \sigma \leq S_{|\psi|}$

                \begin{math}
                  \begin{aligned}[t]
                    \hat{holds}(s, a_{g1}, o, \sigma) \leftarrow &
                    \hat{holds}(s, a_{g2}, o, \sigma), \\
                    & \hat{subst}(a_{g1}, a_{g2}, \sigma), \\
                    & not \lnot \hat{holds}(s, a_{g1}, o, \sigma) \\
                    \lnot \hat{holds}(s, a_{g1}, o, \sigma) \leftarrow &
                    \lnot \hat{holds}(s, a_{g2}, o, \sigma), \\
                    & \hat{subst}(a_{g1}, a_{g2}, \sigma)
                  \end{aligned}
                \end{math}

              \item
                Object Group Subset Inheritance

                $\forall$ ($s$, $a$, $o_{g1}$, $o_{g2}$, $\sigma$), where

                $s \in {\cal E}_{s}$,
                $a \in {\cal E}_{a}$,
                $o_{g1}, o_{g2} \in {\cal E}_{og}$,
                $S_{0} \leq \sigma \leq S_{|\psi|}$

                \begin{math}
                  \begin{aligned}[t]
                    \hat{holds}(s, a, o_{g1}, \sigma) \leftarrow &
                    \hat{holds}(s, a, o_{g2}, \sigma), \\
                    & \hat{subst}(o_{g1}, o_{g2}, \sigma), \\
                    & not \lnot \hat{holds}(s, a, o_{g1}, \sigma) \\
                    \lnot \hat{holds}(s, a, o_{g1}, \sigma) \leftarrow &
                    \lnot \hat{holds}(s, a, o_{g2}, \sigma), \\
                    & \hat{subst}(o_{g1}, o_{g2}, \sigma)
                  \end{aligned}
                \end{math}
            \end{enumerate}

          \subparagraph{Transitivity rules.}

            Given three groups $G$, $G'$ and $G''$. If $G$ is a subset of $G'$
            and $G'$ is a subset of $G''$, then $G$ must also be a subset of
            $G''$. The following rules ensure that the transitive property of
            subject, access right and object groups hold:

            \begin{enumerate}
              \item
                Subject Group Transitivity

                $\forall$ ($sg_{1}$, $sg_{2}$, $sg_{3}$, $\sigma$), where

                $sg_{1}, sg_{2}, sg_{3} \in {\cal E}_{sg}$,
                $S_{0} \leq \sigma \leq S_{|\psi|}$

                \begin{math}
                  \begin{aligned}[t]
                    \hat{subst}(sg_{1}, sg_{3}, \sigma) \leftarrow &
                    \hat{subst}(sg_{1}, sg_{2}, \sigma), \\
                    & \hat{subst}(sg_{2}, sg_{3}, \sigma)
                  \end{aligned}
                \end{math}

              \item
                Access Right Group Transitivity

                $\forall$ ($ag_{1}$, $ag_{2}$, $ag_{3}$, $\sigma$), where

                $ag_{1}, ag_{2}, ag_{3} \in {\cal E}_{ag}$,
                $S_{0} \leq \sigma \leq S_{|\psi|}$

                \begin{math}
                  \begin{aligned}[t]
                    \hat{subst}(ag_{1}, ag_{3}, \sigma) \leftarrow &
                    \hat{subst}(ag_{1}, ag_{2}, \sigma), \\
                    & \hat{subst}(ag_{2}, ag_{3}, \sigma)
                  \end{aligned}
                \end{math}

              \item
                Object Group Transitivity

                $\forall$ ($og_{1}$, $og_{2}$, $og_{3}$, $\sigma$), where

                $og_{1}, og_{2}, og_{3} \in {\cal E}_{og}$,
                $S_{0} \leq \sigma \leq S_{|\psi|}$

                \begin{math}
                  \begin{aligned}[t]
                    \hat{subst}(og_{1}, og_{3}, \sigma) \leftarrow &
                    \hat{subst}(og_{1}, og_{2}, \sigma), \\
                    & \hat{subst}(og_{2}, og_{3}, \sigma)
                  \end{aligned}
                \end{math}
            \end{enumerate}

          \subparagraph{Inertial rules.}

            Intuitively, all facts in the current state that are not affected
            by a policy update should be carried over to the next state after
            the update. In language ${\cal L}^{*}$, this rule must be
            explicitly stated as a constraint. Formally, the inertial rules
            are expressed as follows:

            \begin{vquote}
              $\hat{\alpha}'$ $\leftarrow$ $\hat{\alpha}$, $not$ $\lnot$ $\hat{\alpha}'$

              $\lnot$ $\hat{\alpha}'$ $\leftarrow$ $\lnot$ $\hat{\alpha}$, $not$ $\hat{\alpha}'$

              $\forall$($\hat{\alpha}$,$u$), where
              $\hat{\alpha}$ $\in$ ${\cal A}^{\sigma}$,
              $u$ $\in$ $\psi$,

              $\hat{\alpha}'$ = $CopyAtom$($\hat{\alpha}$, $Result$($u$, $\sigma$))
            \end{vquote}

        \begin{vexample}
          \label{ex-2}
          The following shows the language ${\cal L}$ program listing in
          Example \ref{ex-1},  translated into language ${\cal L}^{*}$.

          \begin{enumerate}
            \item
              Initial State Facts

              $\hat{memb}(alice, grp3, S_{0}) \leftarrow$

              $\hat{holds}(grp1, read, file,S_{0}) \leftarrow$

              $\hat{subst}(grp3, grp2, S_{0}) \leftarrow$

              $\hat{subst}(grp2, grp1, S_{0}) \leftarrow$

            \item
              Constraints

              \begin{math}
                \begin{aligned}[t]
                  % constraints (S0)
                  \hat{holds}&(grp1, write, file, S_{0}) \leftarrow \\
                  & \hat{holds}(grp1, read, file, S_{0}), \\
                  & not \lnot \hat{holds}(grp1, write, file, S_{0}) \\
                  % constraints (S1)
                  \hat{holds}&(grp1, write, file, S_{1}) \leftarrow \\
                  & \hat{holds}(grp1, read, file, S_{1}), \\
                  & not \lnot \hat{holds}(grp1, write, file, S_{1})
                \end{aligned}
              \end{math}

            \item
              Policy Update

              \begin{math}
                \lnot \hat{holds}(grp1, read, file, S_{1}) \leftarrow
              \end{math}

            \item
              Inheritance Rules

              \begin{math}
                \begin{aligned}[t]
                  % inheritance rules (positive, read, S0)
                  \hat{holds}&(alice, read, file, S_{0}) \leftarrow \\
                  & \hat{holds}(grp1, read, file, S_{0}), \\
                  & \hat{memb}(alice, grp1, S_{0}), \\
                  & not \lnot \hat{holds}(alice, read, file, S_{0}) \\
                  % inheritance rules (negative, read, S0)
                  \lnot \hat{holds}&(alice, read, file, S_{0}) \leftarrow \\
                  & \lnot \hat{holds}(grp1, read, file, S_{0}), \\
                  & \hat{memb}(alice, grp1, S_{0})
                \end{aligned}
              \end{math}

              \begin{math}
                \begin{aligned}[t]
                  % inheritance rules (positive, write, S0)
                  \hat{holds}&(alice, write, file, S_{0}) \leftarrow \\
                  & \hat{holds}(grp1, write, file, S_{0}), \\
                  & \hat{memb}(alice, grp1, S_{0}), \\
                  & not \lnot \hat{holds}(alice, write, file, S_{0}) \\
                  % inheritance rules (negative, write, S0)
                  \lnot \hat{holds}&(alice, write, file, S_{0}) \leftarrow \\
                  & \lnot \hat{holds}(grp1, write, file, S_{0}), \\
                  & \hat{memb}(alice, grp1, S_{0})
                \end{aligned}
              \end{math}

              \hspace{1cm} $\vdots$

              \begin{math}
                \begin{aligned}[t]
                  % inheritance rules (positive, read, S1)
                  \hat{holds}&(alice, read, file, S_{1}) \leftarrow \\
                  & \hat{holds}(grp3, read, file, S_{1}), \\
                  & \hat{memb}(alice, grp3, S_{1}), \\
                  & not \lnot \hat{holds}(alice, read, file, S_{1}) \\
                  % inheritance rules (negative, read, S1)
                  \lnot \hat{holds}&(alice, read, file, S_{1}) \leftarrow \\
                  & \lnot \hat{holds}(grp3, read, file, S_{1}), \\
                  & \hat{memb}(alice, grp3, S_{1})
                \end{aligned}
              \end{math}

              \begin{math}
                \begin{aligned}[t]
                  % inheritance rules (positive, write, S1)
                  \hat{holds}&(alice, write, file, S_{1}) \leftarrow \\
                  & \hat{holds}(grp3, write, file, S_{1}), \\
                  & \hat{memb}(alice, grp3, S_{1}), \\
                  & not \lnot \hat{holds}(alice, write, file, S_{1}) \\
                  % inheritance rules (negative, write, S1)
                  \lnot \hat{holds}&(alice, write, file, S_{1}) \leftarrow \\
                  & \lnot \hat{holds}(grp3, write, file, S_{1}), \\
                  & \hat{memb}(alice, grp3, S_{1})
                \end{aligned}
              \end{math}

            \item
              Transitivity Rules

              \begin{math}
                \begin{aligned}[t]
                  \hat{subst}&(grp1, grp1, S_{0}) \leftarrow \\
                  & \hat{subst}(grp1, grp1, S_{0}), \\
                  & \hat{subst}(grp1, grp1, S_{0})
                \end{aligned}
              \end{math}

              \hspace{1cm} $\vdots$

              \begin{math}
                \begin{aligned}[t]
                  \hat{subst}&(grp3, grp3, S_{1}) \leftarrow \\
                  & \hat{subst}(grp3, grp3, S_{1}), \\
                  & \hat{subst}(grp3, grp3, S_{1})
                \end{aligned}
              \end{math}

            \item
              Inertial Rules

              \begin{math}
                \begin{aligned}[t]
                  % inertial rule holds(alice, read, file)
                  \hat{holds}&(alice, read, file, S_{1}) \leftarrow \\
                  & \hat{holds}(alice, read, file, S_{0}), \\
                  & not \lnot \hat{holds}(alice, read, file, S_{1}) \\
                  % inertial rule !holds(alice, read, file)
                  \lnot \hat{holds}&(alice, read, file, S_{1}) \leftarrow \\
                  & \lnot \hat{holds}(alice, read, file, S_{0}), \\
                  & not \lnot \hat{holds}(alice, read, file, S_{1})
                \end{aligned}
              \end{math}

              \begin{math}
                \begin{aligned}[t]
                  % inertial rule holds(alice, write, file)
                  \hat{holds}&(alice, write, file, S_{1}) \leftarrow \\
                  & \hat{holds}(alice, write, file, S_{0}), \\
                  & not \lnot \hat{holds}(alice, write, file, S_{1}) \\
                  % inertial rule !holds(alice, write, file)
                  \lnot \hat{holds}&(alice, write, file, S_{1}) \leftarrow \\
                  & \lnot \hat{holds}(alice, write, file, S_{0}), \\
                  & not \lnot \hat{holds}(alice, write, file, S_{1})
                \end{aligned}
              \end{math}

              \begin{math}
                \begin{aligned}[t]
                  % inertial rule holds(grp1, read, file)
                  \hat{holds}&(grp1, read, file, S_{1}) \leftarrow \\
                  & \hat{holds}(grp1, read, file, S_{0}), \\
                  & not \lnot \hat{holds}(grp1, read, file, S_{1}) \\
                  % inertial rule !holds(grp1, read, file)
                  \lnot \hat{holds}&(grp1, read, file, S_{1}) \leftarrow \\
                  & \lnot \hat{holds}(grp1, read, file, S_{0}), \\
                  & not \lnot \hat{holds}(grp1, read, file, S_{1})
                \end{aligned}
              \end{math}

              \hspace{1cm} $\vdots$

              \begin{math}
                \begin{aligned}[t]
                  % inertial rule holds(grp3, read, file)
                  \hat{holds}&(grp3, read, file, S_{1}) \leftarrow \\
                  & \hat{holds}(grp3, read, file, S_{0}), \\
                  & not \lnot \hat{holds}(grp3, read, file, S_{1}) \\
                  % inertial rule !holds(grp3, read, file)
                  \lnot \hat{holds}&(grp3, read, file, S_{1}) \leftarrow \\
                  & \lnot \hat{holds}(grp3, read, file, S_{0}), \\
                  & not \lnot \hat{holds}(grp3, read, file, S_{1})
                \end{aligned}
              \end{math}

              \begin{math}
                \begin{aligned}[t]
                  % inertial rule holds(grp1, write, file)
                  \hat{holds}&(grp1, write, file, S_{1}) \leftarrow \\
                  & \hat{holds}(grp1, write, file, S_{0}), \\
                  & not \lnot \hat{holds}(grp1, write, file, S_{1}) \\
                  % inertial rule !holds(grp1, write, file)
                  \lnot \hat{holds}&(grp1, write, file, S_{1}) \leftarrow \\
                  & \lnot \hat{holds}(grp1, write, file, S_{0}), \\
                  & not \lnot \hat{holds}(grp1, write, file, S_{1})
                \end{aligned}
              \end{math}

              \hspace{1cm} $\vdots$

              \begin{math}
                \begin{aligned}[t]
                  % inertial rule holds(grp3, write, file)
                  \hat{holds}&(grp3, write, file, S_{1}) \leftarrow \\
                  & \hat{holds}(grp3, write, file, S_{0}), \\
                  & not \lnot \hat{holds}(grp3, write, file, S_{1}) \\
                  % inertial rule !holds(grp3, write, file)
                  \lnot \hat{holds}&(grp3, write, file, S_{1}) \leftarrow \\
                  & \lnot \hat{holds}(grp3, write, file, S_{0}), \\
                  & not \lnot \hat{holds}(grp3, write, file, S_{1})
                \end{aligned}
              \end{math}

              \begin{math}
                \begin{aligned}[t]
                  % inertial rule memb(alice, grp1)
                  \hat{memb}&(alice, grp1, S_{1}) \leftarrow \\
                  & \hat{memb}(alice, grp1, S_{0}), \\
                  & not \lnot \hat{memb}(alice, grp1, S_{1}) \\
                  % inertial rule !memb(alice, grp1)
                  \lnot \hat{memb}&(alice, grp1, S_{1}) \leftarrow \\
                  & \lnot \hat{memb}(alice, grp1, S_{0}), \\
                  & not \hat{memb}(alice, grp1, S_{1})
                \end{aligned}
              \end{math}

              \hspace{1cm} $\vdots$

              \begin{math}
                \begin{aligned}[t]
                  % inertial rule memb(alice, grp3)
                  \hat{memb}&(alice, grp3, S_{1}) \leftarrow \\
                  & \hat{memb}(alice, grp3, S_{0}), \\
                  & not \lnot \hat{memb}(alice, grp3, S_{1}) \\
                  % inertial rule !memb(alice, grp3)
                  \lnot \hat{memb}&(alice, grp3, S_{1}) \leftarrow \\
                  & \lnot \hat{memb}(alice, grp3, S_{0}), \\
                  & not \hat{memb}(alice, grp3, S_{1})
                \end{aligned}
              \end{math}

              \begin{math}
                \begin{aligned}[t]
                  % inertial rule subst(grp1, grp1)
                  \hat{subst}&(grp1, grp1, S_{1}) \leftarrow \\
                  & \hat{subst}(grp1, grp1, S_{0}), \\
                  & not \lnot \hat{subst}(grp1, grp1, S_{1}) \\
                  % inertial rule !subst(grp1, grp1)
                  \lnot \hat{subst}&(grp1, grp1, S_{1}) \leftarrow \\
                  & \lnot \hat{memb}(grp1, grp1, S_{0}), \\
                  & not \hat{memb}(grp1, grp1, S_{1})
                \end{aligned}
              \end{math}

              \hspace{1cm} $\vdots$

              \begin{math}
                \begin{aligned}[t]
                  % inertial rule subst(grp3, grp3)
                  \hat{subst}&(grp3, grp3, S_{1}) \leftarrow \\
                  & \hat{subst}(grp3, grp3, S_{0}), \\
                  & not \lnot \hat{subst}(grp3, grp3, S_{1}) \\
                  % inertial rule !subst(grp3, grp3)
                  \lnot \hat{subst}&(grp3, grp3, S_{1}) \leftarrow \\
                  & \lnot \hat{memb}(grp3, grp3, S_{0}), \\
                  & not \hat{memb}(grp3, grp3, S_{1}) \\
                \end{aligned}
              \end{math}
          \end{enumerate}
        \end{vexample}
      \subsection{Domain Consistency Checking and Evaluation}

        From Definition \ref{def-cons}, it is clear that a domain description
        of language ${\cal L}$ must be consistent in order generate a
        consistent answer set for the evaluation of queries. This section
        considers two issues: the problem of identifying whether a given domain
        description is consistent, and how query evaluation is performed given
        a consistent language domain description.

        Before the above issues can be considered, a few notational constructs
        must first be introduced. Given a domain description ${\cal D_{L}}$
        composed of the following language ${\cal L}$ statements:

        \begin{vverbatim}
initially
  \(a\sb{0}\) &&\ldots&& \(a\sb{n}\) && !\(b\sb{0}\) &&\ldots&& !\(b\sb{m}\)

always
  \(c\sb{0}\) &&\ldots &&\(c\sb{o}\) && !\(d\sb{0}\) &&\ldots&& !\(d\sb{p}\)
  implied by
    \(e\sb{0}\) &&\ldots&& \(e\sb{q}\) && !\(f\sb{0}\) &&\ldots&& !\(f\sb{r}\)
  with absence
    \(g\sb{0}\) &&\ldots&& \(g\sb{s}\) && !\(h\sb{0}\) &&\ldots&& !\(h\sb{t}\)

update()
  causes
    \(i\sb{0}\) &&\ldots && \(i\sb{u}\) && !\(j\sb{0}\) &&\ldots&& !\(j\sb{v}\)
  if
    \(k\sb{0}\) &&\ldots && \(k\sb{w}\) && !\(l\sb{0}\) &&\ldots&& !\(l\sb{x}\)
        \end{vverbatim}

        We define the following sets of ground facts:

        \begin{itemize}
          \item
            ${\cal F}^{+}_{int}$ = \{$a_{z}$ $\mid$ $0$ $\leq$ $z$ $\leq$ $n$\}
          \item
            ${\cal F}^{-}_{int}$ = \{$b_{z}$ $\mid$ $0$ $\leq$ $z$ $\leq$ $m$\}
          \item
            ${\cal F}^{+}_{con}$ = \{$c_{z}$ $\mid$ $0$ $\leq$ $z$ $\leq$ $o$\}
          \item
            ${\cal F}^{-}_{con}$ = \{$d_{z}$ $\mid$ $0$ $\leq$ $z$ $\leq$ $p$\}
          \item
            ${\cal F}^{+}_{upd}$ = \{$i_{z}$ $\mid$ $0$ $\leq$ $z$ $\leq$ $u$\}
          \item
            ${\cal F}^{-}_{upd}$ = \{$j_{z}$ $\mid$ $0$ $\leq$ $z$ $\leq$ $v$\}
        \end{itemize}

        Additionally, we use the complementary set notation
        $\overline{{\cal F}}$ to denote a set containing the negation of
        facts in set ${\cal F}$, i.e. $\overline{{\cal F}}$ =
        \{$\lnot\rho$ $\mid$ $\rho$ $\in$ ${\cal F}$\}.

        Let $\gamma$ be an initial, constraint or update definition statement of
        language ${\cal L}$. We then define $Eff$($\gamma$) to be the set
        \{$a_{0}$,\ldots,$a_{n}$,$b_{0}$,\ldots,$b_{m}$\},
        \{$c_{0}$,\ldots,$c_{o}$,$d_{0}$,\ldots,$d_{p}$\} or
        \{$i_{0}$,\ldots,$i_{u}$,$j_{0}$,\ldots,$j_{v}$\}, respectively;
        $Def$($\gamma$) to be the set $\emptyset$,
        \{$g_{0}$,\ldots,$g_{s}$,$h_{0}$,\ldots,$h_{t}$\} or
        $\emptyset$, respectively; and $Pre$($\gamma$) to be the set
        $\emptyset$, \{$e_{0}$,\ldots,$e_{q}$,$f_{0}$,\ldots,$f_{r}$\} or
        \{$k_{0}$,\ldots,$k_{w}$,$l_{0}$,\ldots,$l_{x}$\}, respectively.

        \begin{vdefinition}
          \label{dev-mutex}
          Given a domain description ${\cal D_{L}}$ of language ${\cal L}$,
          two ground facts $\rho$ and $\rho'$ are {\bf mutually exclusive}
          in ${\cal D_{L}}$ if:
          \begin{vquote}
            $\rho$ $\in$ \{${\cal F}^{+}_{int}$ $\cup$
            $\overline{{\cal F}^{-}_{int}}$ $\cup$ ${\cal F}^{+}_{con}$ $\cup$
            $\overline{{\cal F}^{-}_{con}}$ $\cup$ ${\cal F}^{+}_{upd}$ $\cup$
            $\overline{{\cal F}^{-}_{upd}}$\}

            implies

            $\rho'$ $\not\in$ \{${\cal F}^{+}_{int}$ $\cup$
            $\overline{{\cal F}^{-}_{int}}$ $\cup$ ${\cal F}^{+}_{con}$ $\cup$
            $\overline{{\cal F}^{-}_{con}}$ $\cup$ ${\cal F}^{+}_{upd}$ $\cup$
            $\overline{{\cal F}^{-}_{upd}}$\}
          \end{vquote}
        \end{vdefinition}

        Simply stated, a pair of mutually exclusive facts cannot both be true
        in any given state. The following two definitions refer to language
        ${\cal L}$ statements.

        \begin{vdefinition}
          \label{def-comp}
          Given a domain description ${\cal D_{L}}$ of language ${\cal L}$,
          two statements $\gamma$ and $\gamma'$ are {\bf complementary} in
          ${\cal D_{L}}$ if one of the following conditions holds:
          \begin{enumerate}
            \item
              $\gamma$ and $\gamma'$ are both constraint statements and
              $Eff(\gamma)$ = $\overline{Eff(\gamma')}$.
            \item
              $\gamma$ is a constraint statement, $\gamma'$ is an update
              statement and $Eff(\gamma)$ = $\overline{Eff(\gamma')}$.
          \end{enumerate}
        \end{vdefinition}

        \begin{vdefinition}
          \label{def-norm}
          Given a domain description ${\cal D_{L}}$, ${\cal D_{L}}$ is said to
          be {\bf normal} if it satisfies all of the following conditions:
          \begin{enumerate}
            \item
              ${\cal F}^{+}_{int}$ $\cap$ ${\cal F}^{-}_{int}$ = $\emptyset$
            \item
              For all constraint statements $\gamma$ in ${\cal D_{L}}$,
              $\overline{Eff(\gamma)}$ $\cap$ $Pre(\gamma)$ = $\emptyset$.
            \item
              For any two constraint statements $\gamma$ and $\gamma'$ in
              ${\cal D_{L}}$, $Def(\gamma)$ $\cap$ $Eff(\gamma')$ =
              $\emptyset$.
            \item
              For any two complementary statements $\gamma$ and $\gamma'$ in
              ${\cal D_{L}}$, there exists a pair of ground expression $\epsilon$
              $\in$ $\gamma$ and $\epsilon'$ $\in$ $\gamma'$ such that $\epsilon$ and
              $\epsilon'$ are mutually exclusive.
          \end{enumerate}
        \end{vdefinition}

        \begin{vtheorem}{Domain Consistency}
          \label{the-cons}
          A {\bf normal} domain description of language ${\cal L}$ is also
          {\bf consistent} \footnotemark.
        \end{vtheorem}

        \footnotetext{
          The proof of this theorem will be published in the full version
          of this paper \cite{CRE}.
        }

        By using this theorem to identify consistent domain descriptions of
        language ${\cal L}$, it is now possible to define how queries are
        evaluated from such domain descriptions.

        \begin{vdefinition}
          \label{def-eval}
          Given a consistent domain description ${\cal D}_{\cal L}$, ground
          query expression $\phi$ and a finite sequence list $\psi$, we say the
          query $\phi$ holds in ${\cal D}_{\cal L}$ after the policy updates
          in the sequence list $\psi$ has been applied, denoted as

          \begin{vquote}
            ${\cal D}_{\cal L}$ $\models$ \{$\phi$, $\psi$\},
          \end{vquote}

          \noindent if and only if for every atom $\alpha$ in $\phi$, $\alpha$
          $\in$ ${\cal A}^{|\psi|}$, [$\lnot$] $\alpha$ is in every answer
          set of $Trans$(${\cal D}_{\cal L}$).
        \end{vdefinition}

    \begin{vexample}
      \label{ex-3}
      Given the language ${\cal L}$ code listing in Example \ref{ex-1} and its
      semantic translation in Example \ref{ex-2}, where $\psi$ =
      \{$delete\_read$($grp1$, $file$)\}. The following shows the
      results of each query $\phi$:

      \begin{vquote}
        \begin{math}
          \begin{aligned}[t]
            &\phi_{0} = holds(grp1, write, file) : TRUE \\
            &\phi_{1} = holds(grp1, read, file) : FALSE \\
            &\phi_{2} = holds(alice, write, file) : TRUE \\
            &\phi_{3} = holds(alice, read, file) : FALSE
          \end{aligned}
        \end{math}
      \end{vquote}
    \end{vexample}

  \section{Implementation}
    \label{sec-implement}

    \subsection{System Structure}

    \begin{figure}[ht]
      \begin{center}
        \epsfig{file=system.eps}
        \caption{Structure of PolicyUpdater}
        \label{fig-1}
      \end{center}
    \end{figure}

      As shown in Figure \ref{fig-1}, the PolicyUpdater system works in
      collaboration with an authorisation agent program that queries the
      policy base to determine whether to allow users access to resources.
      Through an authorisation agent program, the PolicyUpdater system also
      allows administrators to dynamically update the policy base by adding
      or removing update directives in the policy update table.

      \subsubsection{Parsers}

        \paragraph{Policy Parser.}

          The policy parser is responsible for correctly reading the policy
          file into the core PolicyUpdater system. The parser ensures that
          the policy file strictly adheres to the language $\cal{L}$ syntax
          then systematically stores entity identifiers into the symbol table
          and initial state facts, constraint expressions and policy update
          definitions are stored into their respective tables in the policy
          base.

        \paragraph{Agent Parser.}

          The agent parser is the direct link between the core PolicyUpdater
          system and the authorisation agent program. The parser's sole purpose
          is to receive language $\cal{L}$ directives from an agent, perform
          the directive upon the policy base and return a reply if the
          directive requires one. Such directives may be to query the policy
          base or to manipulate the policy update sequence table.

      \subsubsection{Data Structures}

        As a program of language ${\cal L}$ is parsed, each statement
        containing entity declarations, facts, constraint rules and policy
        updates must be stored into a structure before the translation process
        is started. As shown in Appendix \ref{app-store}, the structure is composed
        of the symbol table, the policy base and the policy update sequence
        table.

        The symbol table is used to store all entity identifiers defined in the
        policy, while the rest of the policy definitions are stored into the
        policy base. On the other hand, the sequence of policy update
        directives are stored separately into the update table.

    \subsection{System Processes}

      The processes presented in this section shows how the language
      ${\cal L}$ policy stored in the data structures is translated into a
      normal logic program and how it can be dynamically updated and
      manipulated to evaluate queries. The flowchart in Figure \ref{fig-2}
      gives an overview of the system processes.

      \begin{figure}[ht]
        \begin{center}
          \epsfig{file=flowchart.eps}
          \caption{System Flowchart}
          \label{fig-2}
        \end{center}
      \end{figure}

      \subsubsection{Grounding Constraint Variables}

        While constraints are being added into the constraints table, each
        variable identifier that occurs within each constraint is grounded by
        replacing that constraint with a set of constraints wherein each
        instance of the variable is replaced by all entity identifiers defined
        in the symbol table. Note that only those entity identifiers that are
        valid for each fact in the current constraint are used to replace the
        variable (e.g. only singular subject entity identifiers are used to
        replace an element variable occurring in a subject member fact).

        For example, given that the symbol table contains three singular
        subject entity identifiers: $alice$, $bob$ and $charlie$, and the
        following constraint:

        \begin{vverbatim}
always holds(SSUB, write, file)
  implied by
    holds(SSUB, read, file) &&
    memb(SSUB, students)
  with absence
    !holds(SSUB, write, file)
        \end{vverbatim}

        Grounding the constraint above yields three new constraints, each
        replacing occurrences of the variable $SSUB$ with $alice$, $bob$ and
        $charlie$, respectively.

      \subsubsection{Policy Updates}

        In Section \ref{subsec-semantics}, it is shown that policy updates are
        performed by treating each update as a constraint. This constraint is
        composed of a premise, which are the preconditions in the current state
        and a consequent, which is the postcondition of the resulting state
        after the application of the policy update. The resulting state in this
        procedure represents the updated policy.

        The most crucial step in performing a policy update is the translation
        of the policy updates into normal logic program constraints. This step
        involves identifying which policy updates are to be applied from the
        update sequence table and then composing the required constraint from
        the update definition the policy base. Once the policy update
        constraints are composed, they are then treated as any other
        constraint rules and are translated with the rest of the policy into
        a normal logic program.

      \subsubsection{Translation to Normal Logic Program}

        The semantics of language ${\cal L}$ shows that any language
        ${\cal L}$ program can be translated into an equivalent extended logic
        program then translated again into an equivalent normal logic program.
        However, the implementation of such translations can be greatly
        simplified by translating language ${\cal L}$ programs directly into
        normal logic programs.

        \paragraph{Removal of Classical Negation.}

          In order to remove classical negation from facts of language
          ${\cal L}$, each classically negated fact $\lnot$$f$ is replaced by
          a new and unique positive fact $f'$ that represents the negation of
          fact $f$. To preserve the consistency of the policy base for all
          facts $f$ in the domain, the following constraint rule must be added:

          \begin{vquote}
            $FALSE$ $\leftarrow$ $f$, $f'$
          \end{vquote}

          The removal process involves adding a boolean parameter to each fact
          to indicate whether the fact is classically negated or not. For
          example, given the fact:

          \begin{vquote}
            $\lnot$ $holds$($alice$, $exec$, $file$)
          \end{vquote}

          To remove classical negation, this fact is replaced by:

          \begin{vquote}
            $holds$($alice$, $exec$, $file$, $false$)
          \end{vquote}

          For consistency, the following constraint is added:

          \begin{vquote}
            \begin{math}
              \begin{aligned}[t]
                FALSE \leftarrow & holds(alice, exec, file, true), \\
                & holds(alice, exec, file, false)
              \end{aligned}
            \end{math}
          \end{vquote}

        \paragraph{Representing Facts in Propositional Form.}

          A fact expressed in normal logic program form is composed of the
          atom relation, the state in which it holds and a boolean flag to
          indicate classical negation. For notational simplicity, this tuple
          may be represented by a unique positive integer $i$, where $0$ $\leq$
          $i$ $<$ $n$ ($n$ is the total number of possible facts in the
          domain). The process of translating facts of language ${\cal L}$
          into normal logic program form is achieved by performing the
          following steps:

          \subparagraph{Enumerate all possible atoms.}
            By using all the entities in the symbol table, all possible
            language ${\cal L}$ atoms may be enumerated by grouping together
            2 to 3 entities together. All possible atoms of type $holds$ are
            generated by enumerating all possible combinations of subject,
            access right and object entities. The set of $member$ atoms is
            generated from all the different combinations of singular and
            group entities of types subject, access right and object.
            Similarly, the set of $subset$ atoms is derived from different
            subject, access right and object group pair combinations.

          \subparagraph{Arrange the atoms in a predefined order.}
            This procedure relies on the assumption that the list of all
            possible atoms derived from the step above is arranged in a
            predefined order. In this step we ensure that the atoms are
            enumerated in the following order (first to last): $holds$,
            $subject$ $member$, $access$ $right$ $member$, $object$ $member$,
            $subject$ $subset$, $access$ $right$ $subset$ and $object$
            $subset$. In addition to the ordering of atom types, atoms of each
            type are themselves sorted according to the order in which their
            entities appear in the symbol table.

          \subparagraph{Assign an ordinal index for each enumerated atom.}
            Since the enumerated list of atoms are ordered, consecutive
            positive integers may be assigned to each atom as an ordinal
            index $i$, where $0$ $\leq$ $i$ $<$ $n$ ($n$ is the total
            number of atoms enumerated).

          \subparagraph{Extend indexing procedure to represent facts.} Since
            negative facts are just mirror images of their positive
            counterparts, their indices are calculated by adding $n$ to the
            indices of the corresponding positive facts. Thus, indices $i$
            ($n$ $\leq$ $i$ $<$ $2n$) are negative facts. Furthermore, this
            procedure is again extended to represent the states of the facts.
            The process is similar: indices $i$ ($0$ $\leq$ $i$ $<$ $2n$)
            represent facts of state $S_{0}$, indices $i$ ($2n$ $\leq$ $i$ $<$
            $4n$) represent facts of state $S_{1}$, and so on.

          The steps outlined above may be summarised by the function below:

          \begin{vquote}
            $index$ = $encode$($ent1$, $ent2$, $ent3$, $state$, $truth$)
          \end{vquote}

        \paragraph{Generating the Normal Logic Program from the Policy Base.}

          With the language ${\cal L}$ policy elements stored into the storage
          structure (see Appendix \ref{app-store}), a normal logic program can
          then be generated for evaluation. Using the translation methods
          described in Section \ref{subsec-semantics} and the method for
          eliminating classical negation, each initial state expression,
          constraint rule and policy update becomes a set of simple logical
          rules. By applying the $encode$ function above for each fact in each
          of these rules, a normal logic program is generated.

      \subsubsection{Query Evaluation}

        Once a normal logic program has been generated from the policy stored
        in the storage structure, a set of answer sets may then be generated
        by using the stable model semantics with the {\em smodels}\footnotemark
        program. Query evaluation then becomes possible by checking whether
        each fact of a given query holds in each generated answer set of the
        normal logic program.

        \footnotetext{Smodels from http://www.tcs.hut.fi/Software/smodels}

        If a given fact indeed holds in all the answer sets, it is then
        evaluated to be true. On the other hand, if the negation of a fact
        holds in every answer set, then it is evaluated to be false. A fact
        or its negation that does not hold in every answer set is neither true
        nor false.

  \section{Conclusion}
    \label{sec-conclusion}

    In this paper, we have presented the PolicyUpdater system, a logic-based
    authorisation system that features query evaluation and dynamic policy
    updates. This is made possible by the use of a first-order logic
    authorisation language, language ${\cal L}$, for the definition, updating
    and querying of access control policies. As we have shown, language
    ${\cal L}$ is expressive enough to represent constraints and default rules.

    The case study in Appendix \ref{app-case} is a demonstration of how the
    PolicyUpdater system can be adapted to be used in a real-world web server
    authorisation application.

    One possible future extension to this work is the integration of temporal
    logic to language ${\cal L}$ to allow time properties to be expressed in
    access control policies. Such extension will be useful in access control
    systems such as e-commerce applications where authorisations are granted
    or denied based on policies that are time dependent. Another possible
    extension to the system and language is to consider multi-agent
    environments.

  \newpage
  \appendix

  \vappsection{Case Study: Web Server Application}
    \label{app-case}

    \begin{figure}[ht]
      \begin{center}
        \epsfig{file=app.eps}
        \caption{PolicyUpdater module for Apache}
        \label{fig-3}
      \end{center}
    \end{figure}

    The expressiveness of language $\cal{L}$ and the effectiveness of the
    PolicyUpdater system can be best demonstrated by a web server
    authorisation application. In this application, PolicyUpdater serves as
    an authorisation module for the {\em Apache}\footnotemark web server.

    \footnotetext{Apache Web Server (\tt http://www.apache.org)}

    The Apache web server provides a generic access control system as provided
    by its {\em mod\_auth} and {\em mod\_access} modules \cite{AP,LAU}. With
    this built-in access control system, Apache provides the standard HTTP
    {\em Basic} and {\em Digest} authentication schemes \cite{HTTP2}, as well
    as an authorisation system to enforce access control policies. Although the
    PolicyUpdater module do not provide the full functionality of Apache's
    built-in authorisation module {\em mod\_auth}, it does provide a flexible
    logic-based authorisation mechanism.

    As shown in Figure \ref{fig-3}, Apache's Access Control module, together
    with its policy base, is replaced by the PolicyUpdater module and its own
    policy base. The sole purpose of the PolicyUpdater module is to act as an
    interface between the web server and the core PolicyUpdater system. The
    system works as follows: as the server is started, the PolicyUpdater
    module initialises the core PolicyUpdater system by sending the policy
    base. When a client makes an arbitrary HTTP request for a resource from
    the server (1), the client (user) is authenticated against the password
    table by the built-in authentication module; once the client is properly
    authenticated (2) the request is transferred to the PolicyUpdater module,
    which in turn generates a language ${\cal L}$ query (3) from the request
    details, then sends the query to the core PolicyUpdater system for
    evaluation; if the query is successful and access control is granted,
    the original request is sent to the other request handlers of the web
    server (4) where the request is eventually honoured; then finally (5),
    the resource (or acknowledgement for HTTP requests other than GET) is sent
    back to the client. Optionally, client can be an administrator who,
    after being authenticated, is presented with a special administrator
    interface by the module to allow the policy base to be updated.

    \subsection{Policy Description in Language ${\cal L'}$}

      The policy description in the policy base is written in language
      ${\cal L'}$, which is syntactically and semantically similar to
      language ${\cal L}$ except for the lack of entity identifier
      definitions. Entity identifiers need not be explicitly defined in
      the policy definition:

      \begin{itemize}
        \item
          {\bf Subjects} of the access control policies are the users. Since
          all users must first be authenticated, the password table used
          in authentication may also be used to extract the list of subjects.
        \item
          {\bf Access Rights} of the policies are built in: they are the
          HTTP request methods as defined by the HTTP 1.1 standard
          \cite{HTTP1} (i.e. OPTIONS, GET, HEAD, POST, PUT, DELETE, TRACE
          and CONNECT).
        \item
          {\bf Objects} are the resources available in the server themselves.
          Assuming that the document root is a hierarchy of directories and
          files, each of these are mapped as a unique object of language
          ${\cal L'}$.
      \end{itemize}

      Like language ${\cal L}$, language ${\cal L'}$ allows the definition of
      initial state facts, constraint rules and policy update definitions.

    \subsection{Mapping the Policies to Language ${\cal L}$}

      As mentioned above, one task of the PolicyUpdater module is to generate
      a language ${\cal L}$ policy from the given language ${\cal L'}$ to be
      evaluated by the core PolicyUpdater system. This process is outlined
      below:

      \begin{itemize}
        \item
          {\bf Generating entity identifier definitions}. Subjects are
          extracted from the authentication (password) table; access rights
          are hard-coded built-ins; and the list of objects are generated by
          traversing the document root for files and directories.
        \item
          {\bf Generating additional constraints}. The module also generates
          additional constraints to preserve the relationship between groups
          and elements. This is useful to model the assertion that unless
          explicitly stated, users holding particular access rights to a
          directory automatically hold those access rights to every file in
          that directory (recursively, if with subdirectories). The
          PolicyUpdater module achieves this assertion by generating
          non-conditional constraint rules that state that each file
          (object) is a member of the directory (object group) in which it
          is contained.
      \end{itemize}

      All other language ${\cal L'}$ statements (initial state definitions,
      constraint definitions and policy update definitions) are already in
      language ${\cal L}$ form.

    \subsection{Evaluation of HTTP Requests}

      A HTTP request may be represented as a simplified tuple:

      \begin{quote}
        $<$$usr$, $req\_meth$, $req\_res$$>$
      \end{quote}

      $usr$ is the authenticated username that made the request (subject);
      $req\_meth$ is any of the standard HTTP request methods (access
      right); and $req\_res$ is the resource associated with the request
      (object). Intuitively, such a tuple may be expressed as a language
      ${\cal L}$ atom:

      \begin{vverbatim}
holds(usr, req\_meth, req\_res)
      \end{vverbatim}

      With each request expressed as language ${\cal L}$ atoms, a language
      ${\cal L}$ query statement can be composed to check if the request is
      to be honoured:

      \begin{vverbatim}
query holds(usr, req\_meth, req\_res);
      \end{vverbatim}

      Once the query statement is composed, it is then sent by the
      PolicyUpdater module to the core PolicyUpdater system for evaluation
      against the policy base.

    \subsection{Policy Updates by Administrators}

      After being properly authenticated, an administrator can perform policy
      updates through the use of a special interface generated by the
      PolicyUpdater module. This interface lists all the predefined policy
      updates that are allowed, as defined in the policy description in
      language ${\cal L'}$, as well as all the policy updates that have been
      previously applied and are in effect. As with the core PolicyUpdater
      system, administrators are allowed only the following operations:

      \begin{itemize}
        \item
          Apply a policy update or a sequence of policy updates to the policy
          base. Note that like language ${\cal L}$, in language ${\cal L'}$
          policy updates are predefined within the policy base themselves.
        \item
          Revert to a previous state of the policy base by removing a
          previously applied policy update from the policy base.
      \end{itemize}

  \vappsection{Storage Structures}

    \label{app-store}

    The data structures outlined in this section are used as a storage
    structure to hold the elements of language ${\cal L}$ before any
    operations are performed.

    Each of the tables and lists used in the system inherits from a generic
    ordered and indexed list implementation. Each node in this list holds a
    generic data type that can be used to store strings, an arbitrary data
    type or another list type.

    \subsection{Symbol Table}

      The symbol table is used to store the identifier entities defined in
      the entity identifier declaration section of language ${\cal L}$
      programs. The symbol table is composed of 6 separate string lists:

      \begin{vquote}
        \begin{tabular}[t]{|l|l|l|}
          \hline
          \textbf{Field} & \textbf{Type} & \textbf{Description} \\
          \hline
          $ss$ & string list & single subject \\
          \hline
          $sg$ & string list & group subject \\
          \hline
          $as$ & string list & single access right \\
          \hline
          $ag$ & string list & group access right \\
          \hline
          $os$ & string list & single object \\
          \hline
          $og$ & string list & group object \\
          \hline
        \end{tabular}
      \end{vquote}

      Each entity identifier are sorted in the above lists according to
      their type, and ordered according to the order in which they are
      declared in the program. Each list is indexed by consecutive
      positive integers starting from zero.

    \subsection{Policy Base}

      When a language ${\cal L}$ program is parsed, each of the facts,
      rules and policy updates must first be stored into the policy base.
      The policy base is composed of 4 tables to store the following:
      initial state facts, constraint rules, policy update definitions and
      the policy update sequence.

      \subsubsection{Atoms}

        The three types of atoms (subject, access right and object) are
        represented as structures of 2 to 3 strings, with each string
        matching an entity identifier in the symbol table.

        \begin{vquote}
          \begin{tabular}[t]{|l|l|l|l|}
            \hline
            \textbf{Field} & \textbf{Type} & \textbf{Description} & \\
            \hline
            $sub$ & string & subject entity & {\multirow{3}{*}{hol}} \\
            \cline{1-3}
            $acc$ & string & access right entity & \\
            \cline{1-3}
            $obj$ & string & object entity & \\
            \hline
            \hline
            $elt$ & string & single entity & {\multirow{2}{*}{mem}} \\
            \cline{1-3}
            $grp$ & string & group entity & \\
            \hline
            \hline
            $grp1$ & string & subgroup entity & {\multirow{2}{*}{sub}} \\
            \cline{1-3}
            $grp2$ & string & supergroup entity & \\
            \hline
          \end{tabular}
        \end{vquote}

      \subsubsection{Facts}

        Facts are represented as a three-element structure composed of the
        following: polymorphic type which can be any of the three atom
        structures above; a type indicator to specify whether the fact is
        holds, member or subset; and a truth flag, to indicate whether the
        atom is classically negated or not ({\em true} if the fact holds
        and {\em false} if the classical negation of the fact holds.

        \begin{vquote}
          \begin{tabular}[t]{|l|l|l|}
            \hline
            \textbf{Field} & \textbf{Type} & \textbf{Description} \\
            \hline
            $atom$ & atom type & polymorphic structure \\
            \hline
            $type$ & \{h$|$m$|$s\} & holds, member or subset \\
            \hline
            $truth$ & boolean & negation indicator \\
            \hline
          \end{tabular}
        \end{vquote}

      \subsubsection{Expressions}

        Since expressions are simply conjunctions of facts, they are
        represented as a list of fact structures.

      \subsubsection{Initial State Facts Table}

        The initial state facts table is represented as a single list of
        fact structures, or an expression. Each fact in all {\em initially}
        statements are added into the initial state facts table.

      \subsubsection{Constraint Table}

        The constraint table is represented as a list of constraint
        structures, with each structure composed of the following:

        \begin{vquote}
          \begin{tabular}[t]{|l|l|l|}
            \hline
            \textbf{Field} & \textbf{Type} & \textbf{Description} \\
            \hline
            $exp$ & expression type & consequent \\
            \hline
            $pcond$ & expression type & positive premise \\
            \hline
            $ncond$ & expression type & negative premise \\
            \hline
          \end{tabular}
        \end{vquote}

      \subsubsection{Policy Update Definition Table}

        Another list of structures is the policy update table. Each element
        structure of this table is composed of the following 4 fields:

        \begin{vquote}
          \begin{tabular}[t]{|l|l|l|}
            \hline
            \textbf{Field} & \textbf{Type} & \textbf{Description} \\
            \hline
            $name$ & string & update identifier \\
            \hline
            $vlist$ & ordered string list & variables \\
            \hline
            $pre$ & expression type & precondition \\
            \hline
            $post$ & expression type & postcondition \\
            \hline
          \end{tabular}
        \end{vquote}

    \subsection{Policy Update Sequence Table}

      The policy update sequence table is an ordered list of sequence
      structures, each with the following elements:

      \begin{vquote}
        \begin{tabular}[t]{|l|l|l|}
          \hline
          \textbf{Field} & \textbf{Type} & \textbf{Description} \\
          \hline
          $name$ & string & update identifier \\
          \hline
          $ilist$ & ordered string list & identifiers \\
          \hline
        \end{tabular}
      \end{vquote}

  \begin{thebibliography}{5}
    \bibitem{AP}
      Apache Software Foundation,
      Authentication, Authorization and Access Control.
      In {\em Apache HTTP Server 2.1 Documentation},
      {\scriptsize \tt http://httpd.apache.org/docs-2.1/}, 2004.

    \bibitem{BA1}
      Bai, Y., Varadharajan, V.,
      On Formal Languages for Sequences of Authorization Transformations.
      In {\em Proceedings of Safety, Reliability and Security of Computer
      Systems}. Also in {\em Lecture Notes in Computer Science},
      vol. 1698, pp. 375-384. Springer-Verlag, 1999.

    \bibitem{BA2}
      Bai, Y., Varadharajan, V.,
      On Transformation of Authorization Policies.
      In {\em Data and Knowledge Engineering},
      vol. 45, no. 3, pp. 333-357, 2003.

    \bibitem{BE1}
      Bertino, E., Buccafurri, F., Ferrari, E., Rullo, P.,
      A Logic-based Approach for Enforcing Access Control.
      In {\em Journal of Computer Security},
      vol. 8, no. 2-3, pp. 109-140, IOS Press, 2000.

    \bibitem{BE2}
      Bertino, E., Mileo A., Provetti, A.
      Policy Monitoring with User-Preferences in PDL.
      In {\em Proceedings of IJCAI-03 Workshop for Nonmonotonic Reasoning,
      Action and Change},
      pp. 37-44, 2003.

    \bibitem{CHO}
      Chomicki, J., Lobo, J., Naqvi S.,
      A Logic Programming Approach to Conflict Resolution in Policy Management.
      In {\em Proceedings of KR2000, 7th International Conference on Principles
      of Knowledge Representation and Reasoning},
      pp. 121-132, Kaufmann, 2000.

    \bibitem{CRE}
      Crescini, V.F., Zhang Y.,
      {\em PolicyUpdater - A System for Dynamic Access Control}.
      (manuscript) 2004.

    \bibitem{HAL}
      Halpern, J.Y., Weissman V.,
      Using First-Order Logic to Reason About Policies.
      In {\em Proceedings of the 16th IEEE Computer Security Foundations
      Workshop}, pp.187-201, 2003.

    \bibitem{JAJ}
      Jajodia, S., Samarati, P., Sapino, M. L., Subrahmanian, V. S.,
      Flexible Support for Multiple Access Control Policies.
      In {\em ACM Transactions on Database Systems},
      vol. 29, no. 2, pp. 214-260, ACM, 2001.

    \bibitem{LAU}
      Laurie, B., Laurie, P.,
      {\em Apache: The Definitive Guide} (3rd Edition).
      O'Reilly \& Associates Inc., 2003.

    \bibitem{LI}
      Li N., Grosof B.N., Feigenbaum, J.,
      Delegation Logic: A Logic-based Approach to Distributed Authorization.
      In {\em ACM Transactions on Information and System Security (TISSEC)},
      vol. 6, no. 1, pp. 128-171, ACM, 2003.

    \bibitem{LOB}
      Lobo J., Bhatia R., Naqvi S.,
      A Policy Description Language.
      In {\em Proceedings of AAAI 16th National Conference on Artificial
      Intelligence and 11th Conference on Innovative Applications of Artificial
      Intelligence }, pp. 291-298, AAAI Press, 1999.

    \bibitem{HTTP1}
      Network Working Group,
      {\em HTTP 1.1 (RFC 2616)}.
      The Internet Society, \\
      {\scriptsize \tt ftp://ftp.isi.edu/in-notes/rfc2616.txt},
      1999.

    \bibitem{HTTP2}
      Network Working Group,
      {\em HTTP Authentication (RFC 2617)}.
      The Internet Society, \\
      {\scriptsize \tt ftp://ftp.isi.edu/in-notes/rfc2617.txt},
      1999.
  \end{thebibliography}
\end{document}
