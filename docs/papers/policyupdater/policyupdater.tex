\documentclass[10pt, twocolumn]{article}
\usepackage{latex8}
\usepackage{times}

\newtheorem{definition}{Definition}

\begin{document}
  \title{
    PolicyUpdater -- A System for Dynamic Access Control:  \\
    Formalization, Implementation and a Case Study
  }

  \author{
    Vino Fernando Crescini and Yan Zhang                   \\
    School of Computing and Information Technology         \\
    University of Western Sydney                           \\
    Penrith South DC, NSW 1797, Australia                  \\
    E-mail: \{jcrescin,yan\}@cit.uws.edu.au
  }

  \date{}

  \maketitle

  \begin{abstract}
    [text]
  \end{abstract}

  \section{Introduction}

    \subsection{Overview of Logic-Based Approaches}

    \subsection{Contribution}

    \subsection{Structure of this Paper}

  \section{Language $\cal{L}$}

    \subsection{Syntax}

      [language statements, statement terminators, and comments]

      \subsubsection{Components of Language $\cal{L}$}

        \noindent \textbf{\emph{Identifiers}}

          The most fundamental unit of Language $\cal{L}$ is an identifier. An
          identifier is used to represent the different components of the
          language. 
    
          The first character of an identifier must be an alpha character
          followed by 0 to 127 characters of alpha, digit or underscore
          characters.

          \begin{verbatim}[a-zA-Z]([a_zA-Z0-9_]){0,127}\end{verbatim}

          An identifier is divided into three main classes:

          \begin{itemize}
            \item
              $Entity$ $Identifiers$ represent entities that make up a logical
              atom. They are further divided into three types, with each type
              again divided into the $singular$ $entity$ and $group$ $entity$
              categories:
              \begin{itemize}
                \item
                  $Subjects$: priviledge holders (eg. alice, astrologers,
                  students).
                \item
                  $Access-Rights$: priviledges (eg. read, write, own).
                \item
                  $Objects$: priviledged entities (eg. file, database,
                  directory).
              \end{itemize}
            \item
              $Policy$ $Update$ $Identifiers$ are used to name a policy
              update definition or directive.
            \item
              $Variable$ $Identifiers$ are used as entity identifier
              placeholders in policy updates.
          \end{itemize}

        \noindent \textbf{\emph{Atoms}}
 
          Atoms are composed of a relation plus 2 to 3 entity identifiers
          that represent a logical relationship between the entities. There are
          three types of atoms:

          \begin{itemize}
            \item
              $Holds$. An atom of this type states that the subject identifier
              $sub$ holds the access-right identifier $acc$ for the object
              identifier $obj$.
         
              \begin{verbatim}holds(<sub>, <acc>, <obj>)\end{verbatim}
            \item
              $Membership$. This type of atom states that the singular
              identifier $elt$ is a member or element of the group identifier
              $grp$. It is important to note that identifiers $elt$ and $grp$
              must be of the same base type (eg. subject and subject group).
         
              \begin{verbatim}memb(<elt>, <grp>)\end{verbatim}
            \item
              $Subset$. The subset atom states that the groups identifiers
              $grp1$ and $grp2$ are of the same types and that group $grp1$
              is a subset of the group $grp2$.

              \begin{verbatim}subst(<grp1>, <grp2>)\end{verbatim}
          \end{itemize}

        \noindent \textbf{\emph{Facts}}

          A fact makes a claim that the relationship represented by an atom or
          its negation holds in the current context. Facts are negated by the
          use of the negation operator ($!$). The following shows the formal
          syntax of a fact:
 
          \begin{verbatim}[!]<holds_atm>|<memb_atm>|<subst_atm>\end{verbatim}
 
        \noindent \textbf{\emph{Expressions}}

          An expression is either a fact, or a logical conjunction of facts,
          separated by the double-ampersand characters $\&\&$.

          \begin{verbatim}<fact1> [&& <fact2> [&& ...]]\end{verbatim}

      \subsubsection{Identifier Entity Declaration}

        All entity identifiers (subjects, access-rights, objects and groups)
        must first be declared before any other statements to define the
        entity domain of the policy base. The following entity declaration
        syntax illustrates how to define one or more entity identifiers of a
        particular type.

        \begin{verbatim}ident sub|acc|obj[-grp] <name>[, ...]];\end{verbatim}

        where $name$ is restricted to the formal identifier syntax.

      \subsubsection{Initial State Definition}

        The initial facts in the policy base can be defined by the following
        syntax:

        \begin{verbatim}initially <expression>;\end{verbatim}

        where $expression$ is restricted to the formal expression syntax.

      \subsubsection{Constraint Definition}

        Constraints are logical rules that holds regardless of any changes
        that may occur when the policy base is updated. The constraint rules
        are true in the initial state and remains true after any policy update.

        The constraint syntax below shows that for any state of the policy
        base, expression $exp1$ holds if expression $exp2$ is true and there
        is no evidence that $exp3$ is true.

        Note that the conditional clauses $implied$ $by$ and $with$ $absence$
        are optional.

\begin{verbatim}
always <exp1>
  [implied by <exp2>
  [with absence <exp3>]];
\end{verbatim}

      \subsubsection{Policy Update Definition}

        Before a policy update can be applied, it must first be defined by
        by using the following syntax:

\begin{verbatim}
<up_ident>([<var_ident>[, ...]])
  causes <exp1>
  if <exp2>;
\end{verbatim}

        $up\_ident$ is the policy update identifier to be used in referencing
        this identifier. The optional $var\_ident$ list are the variable
        identifiers occuring the expressions and will eventually be replaced
        by entity identifiers when the update is referenced. The postcondition
        expression $exp1$ is an expression that will hold in the state after
        this update is applied. The expression $exp2$ is a precondition
        expression that must hold in the current state before this update is
        applied.

        It is important to note that a defined policy state has no effect
        whatsoever on the policy base until it is applied by one of the
        directives described in the next section.

      \subsubsection{Policy Update Directives}

        The policy update sequence list contains a list of references to
        defined policy updates in the domain. The policy updates in the
        sequence list are applied to the current state of the policy base one
        at a time to produce a policy base state upon which queries can be
        evaluated. The order in which the policy update references appear in
        sequence list is the same order in which they were added.

        The following four directives are the policy sequence manipulation
        features of Language $\cal{L}$.

        \emph{Adding an Update into the Sequence}. Defined policy updates are
        added into the sequence list through the syntax below.

        \begin{verbatim}seq add <up_ident>([<e_ident>[, ...]]);\end{verbatim}

        where $up\_ident$ is the identifier of a defined policy update and 
        the $e\_ident$ list is a comma-separated list of entity identifiers
        that will replace the variable identifiers that occur in the
        definition of the policy update. It is important to note that the
        replacement of variable identifiers with entity identifiers is
        dependent on the order in which they appear in the list.

        \emph{Listing the Updates in the Sequence}. The following directive may
        be used to list the current contents of the policy update sequence
        list.

        \begin{verbatim}seq list;\end{verbatim}

        This directive is answered with an ordinal list of policy updates in
        the form:

        \begin{verbatim}<n> <up_ident>([e_ident[, ...]])\end{verbatim}

        where $n$ is the ordinal index of the policy update within the sequence
        list starting at 0. $up_ident$ is the policy update identifier and the
        $e_ident$ list is the list of entity identifiers used to replace the
        variable identifier placeholders.

        \emph{Removing an Update from the Sequence}. The syntax below shows the
        directive to remove a policy update reference from the list. $n$ is the
        ordinal index of the policy update to be removed. Note that removing
        a policy update reference from the sequence list may change the ordinal
        index of the other update references.

        \begin{verbatim}seq del <n>;\end{verbatim}

      \emph{Computing after a Polcy Update}. The policy updates in the sequence
      list does not get applied until the compute directive is issued. The
      directive causes the policy update references in the sequence list to be
      applied one at a time in the same order that they appear in the list. The
      directive also causes the system to generate the policy base models
      against which query requests can be evaluated.

      \begin{verbatim}compute;\end{verbatim}

      \subsubsection{Query Directive}

        A query expression may be issued against the current state of the
        policy base. This current state is derived after all the updates in
        the update sequence has been applied, one at a time, onto the initial
        state. Query expressions are answered with a \emph{true}, \emph{false}
        or an \emph{unknown}, depending on whether the queried expression
        holds, its negation holds, or neither, respectively. Syntax is as
        follows:

        \begin{verbatim}query <exp>;\end{verbatim} 

    \subsection{Semantics}

      \subsubsection{Entity Semantics}

        \begin{definition}

          The entity set ${\cal E}$ is the union of six disjoint entity sets:
          single-subject ${\cal E}_{SS}$, group-subject ${\cal E}_{SG}$,
          single-access-right ${\cal E}_{AS}$, group-access-right
          ${\cal E}_{AG}$, single-object ${\cal E}_{OS}$ and group-object
          ${\cal E}_{OG}$.

          \begin{enumerate}
            \item
              ${\cal E} = {\cal E}_{S} \cup {\cal E}_{A} \cup {\cal E}_{O}$
            \item
              ${\cal E}_{S} = {\cal E}_{SS} \cup {\cal E}_{SG}$
            \item
              ${\cal E}_{A} = {\cal E}_{AS} \cup {\cal E}_{AG}$
            \item
              ${\cal E}_{O} = {\cal E}_{OS} \cup {\cal E}_{OG}$
          \end{enumerate}

        \end{definition}

      \subsubsection{State Semantics}

        Conceptually, a state may be thought of as a set of facts and
        constraints of the policy base at a particular instant. However, the
        semantics of Language ${\cal L}$ gives a more formal and less tangible
        definition of state.

        \begin{definition}
          A state is a property of facts and rules used to distinguish between 
          the facts and rules of a specific policy base instance from another
          policy base instance caused by the application of a policy update.
        \end{definition}

        The notation $PB$ $\overrightarrow{_{U}}$ $PB'$ shows that the set of
        rules and constraints of the policy base $PB$ yields the policy base
        set $PB'$ after the policy update $U$ is applied.

        \begin{definition}
          The ordered sequence set ${\cal S}$ contains all the policy updates
          to be applied. The number of states in the domain is $|{\cal S}|$ + 
          1.
        \end{definition}

        \begin{definition}
          
          The atom set $\cal{A}$ is the set of all atoms in the domain. Each
          atom represents a logical relationship of a particular state of the
          policy base.

          \begin{enumerate}
            \item
              ${\cal A} = {\cal A}_{H} \cup {\cal A}_{M} \cup {\cal A}_{S}$
            \item
              ${\cal A}_{H} = \{\forall (s \in {\cal E}_{S}, a \in {\cal E}_{A}, o \in {\cal E}_{O}, holds(s, a, o) \}$
            \item
              ${\cal A}_{M} = {\cal A}_{MS} \cup {\cal A}_{MA} \cup {\cal A}_{MO}$
            \item
              ${\cal A}_{S} = {\cal S}_{SS} \cup {\cal S}_{SA} \cup {\cal A}_{SO}$
            \item
              ${\cal A}_{MS} = \{\forall (e \in {\cal E}_{SS}, g \in {\cal E}_{SG}), memb(e, g)\}$
            \item
              ${\cal A}_{MA} = \{\forall (e \in {\cal E}_{AS}, g \in {\cal E}_{AG}), memb(e, g)\}$
            \item
              ${\cal A}_{MO} = \{\forall (e \in {\cal E}_{OS}, g \in {\cal E}_{OG}), memb(e, g)\}$
            \item
              ${\cal A}_{SS} = \{\forall (g1, g2 \in {\cal E}_{SG}), subst(g1, g2)\}$
            \item
              ${\cal A}_{SA} = \{\forall (g1, g2 \in {\cal E}_{AG}), subst(g1, g2)\}$
            \item
              ${\cal A}_{SO} = \{\forall (g1, g2 \in {\cal E}_{OG}), subst(g1, g2)\}$
          \end{enumerate}

        \end{definition} 

        \noindent \emph{Initial State}

        \noindent \emph{Constraints}

        \noindent \emph{Policy Updates}

        \noindent \emph{Queries}

      \subsubsection{Additional Constraints}

        \noindent \emph{Inheritance Rules}

        \noindent \emph{Transitivity Rules}

        \noindent \emph{Contradictory Rule}

          $false$ $\leftarrow$ $A$ $\land$ $\lnot$ $A$

    \subsection{Example}

    \subsection{Properties}

  \section{Implementation}

    \subsection{System Structure}

    \subsection{Algorithms}

  \section{Case Study: Web Server Application}

  \section{Conclusion}

    [summary]

    [effectiveness of system/application]

    [future work]

\end{document}
