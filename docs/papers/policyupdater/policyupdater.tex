\documentclass[global,twocolumn,final]{svjour}
\usepackage{times}
\usepackage{amsmath}
\usepackage{alltt}
\usepackage{graphicx}
\usepackage{rotating}
\usepackage{multirow}

% definition environment (from svglobal)
\newenvironment{vdefinition}
  {\begin{definition}\hspace{0.25em}}
  {\end{definition}}
% theorem environment (from svglobal)
\newenvironment{vtheorem}[1]
  {\begin{theorem}[#1]\hspace{0.25em}}
  {\end{theorem}}
% proof environment (from svglobal)
\newenvironment{vproof}
  {\begin{proof}\hspace{0.25em}}
  {\qed\end{proof}}
% example environment (from svglobal)
\newenvironment{vexample}
  {\begin{example}\hspace{0.25em}}
  {\end{example}}
% create a quote environment without a right-hand margin
\newenvironment{vquote}
  {\begin{list}{}{\leftmargin 1em}\item[]}
  {\end{list}}
% create a *smaller* alltt (verbatim) environment
\newenvironment{vverbatim}
  {\begin{alltt}}
  {\vspace{-\baselineskip}\end{alltt}}
% define a section command to be used in the appendices
\makeatletter
\newcommand{\vappsection}[1]{
  % change the numbering to Appendix *level*
  \renewcommand{\@seccntformat}[1]{
    \appendixname\hspace{0.5em}\csname the##1\endcsname \hspace{1em}
  }
  \section{#1}
  % change the numbering back
  \renewcommand{\@seccntformat}[1]{
    \csname the##1\endcsname\hspace{1em}
  }
}
\makeatother

\begin{document}
  \title{PolicyUpdater -- A System for Dynamic Access Control}
  \author{Vino Fernando Crescini and Yan Zhang}
  \institute{
    School of Computing and Information Technology \\
    University of Western Sydney                   \\
    Penrith South DC, NSW 1797, Australia          \\
    \email{\{jcrescin,yan\}@cit.uws.edu.au}
  }

  \maketitle

  \begin{abstract}
    {\em PolicyUpdater} is a fully-implemented authorisation system that
    provides policy evaluations as well as dynamic policy updates. These
    functions are achieved by the use of a logic-based language ${\cal L}$ to
    represent the underlying access control policies, constraints and update
    propositions. The system performs access control query evaluations and
    conditional policy updates by translating the language $\cal{L}$ to a
    normal logic program in a form suitable for evaluation using the
    {\em Stable Model} semantics. In this paper, we show the underlying
    mechanisms that make up the PolicyUpdater system, including its system
    structure, some theoretical background as well as implementation issues and
    some performance analysis.
  \end{abstract}

  \keywords{
    access control, authorisation, artificial intelligence, logic programming,
    policy update
  }

  \section{Introduction}

    The traditional access control mechanism is the {\em Access Control Matrix}
    where columns represent subjects, rows represent objects and each cell
    contains the access-rights of a subject over a particular object. However,
    flexibility and scalability issues arise when such method is used on
    real-world applications. A more effective paradigm of access control
    systems is the logic-based approach. In this approach, instead of
    explicitly defining all access-rights of all subjects for all objects
    in a domain, a set of logical facts and rules are used to define the
    policy base.

    Recent advances in the field have produced a number of different approaches
    to logic-based access control systems, e.g. \cite{HAL,LI}. One such
    access control system was proposed by Bai and Varadharajan \cite{BA1,BA2}.
    Their system's key characteristic is its ability to dynamically update an
    otherwise static policy base.

    Another system, proposed by Bertino, et. al. \cite{BE1}, uses an
    authorisation mechanism based on ordered logic. This powerful mechanism
    supports both positive and negative authorisations as well as rule
    derivations and default propositions. Other notable features of this system
    include the distinction between weak and strong authorisations, support
    for administrative authorisation delegation and more importantly, conflict
    resolution.

    Jajodia, et. al. \cite{JAJ} argued that most authorisation system models
    work on a single specific access control policy. Although it is
    theoretically possible for such systems to handle multiple policies, in
    practice, only one specific policy can be applied in a given system. As
    a solution to this problem, they proposed a general access control
    framework whose main feature is its flexibility to handle multiple
    policies in one system. Other features of this framework includes support
    for groups and roles, conflict resolution mechanisms and support for
    different decision strategies.

    These systems, effective as they are, lack the details necessary to address
    the issues involved in the implementation of such systems.

    The {\em Policy Description Language}, or {\em PDL}, developed by Lobo,
    et. al. \cite{LOB}, is a language for representing event and action
    oriented generic policies. {\em PDL} is later extended by Chomicki, et. al.
    \cite{CHO} to include {\em policy monitors} which, in effect, are policy
    constraints. Bertino, et. al. \cite{BE2}, again took {\em PDL} a step
    further by extending {\em policy monitors} to allow users to express
    preferred constraints. While these generic languages are expressive enough
    to be used for access control systems, systems built for such languages
    will not have the ability to dynamically update the policies.

    To overcome these limitations, we propose the PolicyUpdater access control
    system, which, with its own access control language, provides a formal
    logic-based representation of policies, with variable resolution and
    default propositions, a mechanism to conditionally and dynamically
    perform a sequence of policy updates, and a means of evaluating queries
    against the policies.

    The rest of this paper is organised as follows. In Section \ref{sec-langl},
    the paper introduces language ${\cal L}$, with its formal syntax, semantics
    and some examples. In Section \ref{sec-cons}, the issues of consistency and
    query evaluation are addressed. The implementation, as discussed in Section
    \ref{sec-implement}, gives an overview of the PolicyUpdater system as
    a whole, with its internal and external components. The section also
    includes a few algorithms that outline the underlying mechanisms and also
    some experimental results that show the relationship between input size
    and execution time. The case study presented in Section \ref{sec-case}
    shows a typical application of the PolicyUpdater system: an access control
    system for web servers. Section \ref{sec-future} outlines the issues
    involved in extending the system to include temporal authorisations.
    Finally, Section \ref{sec-conclusion} contains a summary of the paper.

    The PolicyUpdater system was originally introduced in the conference
    proceedings paper \cite{CR1}. Another conference proceedings paper
    \cite{CR2} focuses on a web server authorisation system based on the
    core PolicyUpdater system.

  \section{Language $\cal{L}$}
    \label{sec-langl}

    Language $\cal{L}$ is a first-order logic language that represents a policy
    base for an authorisation system. Two key features of the language are: (1)
    providing a means to conditionally and dynamically update the existing
    policy base and (2) having a mechanism by which queries may be evaluated
    from the updated policy base.

    \subsection{Syntax}
      \label{subsec-syntax}

      Logic programs of language ${\cal L}$ are composed of language
      statements, each terminated by a semicolon ";" character. C-style
      comments delimited by the "/*" and "*/" characters may appear anywhere in
      the logic program.

      \subsubsection{Components of Language $\cal{L}$.}

        Each language ${\cal L}$ statement is composed of the following
        components: identifiers, atoms, facts and expressions.

        \paragraph{Identifiers.}
          The most basic unit of language $\cal{L}$ is the identifier.
          Identifiers are used to represent the different components of the
          language, and are divided into three main classes:

          \begin{itemize}
            \item
              {\em Entity Identifiers} represent constant entities that make up
              a logical atom. They are divided further into three types, with
              each type again divided into the {\em singular entity} and
              {\em group entity} categories:

              \begin{itemize}
                \item
                  {\em Subjects}: e.g. alice, lecturers, group.
                \item
                  {\em Access Rights}: e.g. read, write, own.
                \item
                  {\em Objects}: e.g. file, database, directory.
              \end{itemize}

              An entity identifier is defined as a lower-case alphabet
              character, followed by 0 to 127 characters of alphabet, digit or
              underscore characters. The following regular expression shows the
              syntax of entity identifiers:

              \begin{vverbatim}
  [a-z]([a-zA-Z0-9\_])\{0,127\}
              \end{vverbatim}

            \item
              {\em Policy Update Identifiers} are used for the sole purpose of
              naming a policy update. These identifier names are then used as
              labels to refer to policy update definitions and directives. As
              labels, identifiers of this class occupy a different namespace
              from entity identifiers. For this reason, policy update
              identifiers share the same syntax with entity identifiers:


              \begin{vverbatim}
  [a-z]([a-zA-Z0-9\_])\{0,127\}
              \end{vverbatim}

            \item
              {\em Variable Identifiers} are used as place-holders for entity
              identifiers. To distinguish them from entity and policy update
              identifiers, variable identifiers are prefixed with an upper-case
              character, followed by 0 to 127 alphanumeric and underscore
              characters. The following regular expression shows the syntax of
              variable identifiers:

              \begin{vverbatim}
  [A-Z]([a-zA-Z0-9\_])\{0,127\}
              \end{vverbatim}
          \end{itemize}

        \paragraph{Atoms.}
          An atom is composed of a relation with 2 to 3 entity or variable
          identifiers that represent a logical relationship between the
          entities. There are three types of atoms:

          \begin{itemize}
            \item
              {\em Holds.} An atom of this type states that the subject
              identifier $sub$ holds the access right identifier $acc$
              for the object identifier $obj$.

              \begin{vverbatim}
  holds(<sub>, <acc>, <obj>)
              \end{vverbatim}
            \item
              {\em Membership.} This type of atom states that the singular
              identifier $elt$ is a member or element of the group identifier
              $grp$. It is important to note that identifiers $elt$ and $grp$
              must be of the same base type (e.g. subject and subject group).

              \begin{vverbatim}
  memb(<elt>, <grp>)
              \end{vverbatim}
            \item
              {\em Subset.} The subset atom states that the group identifiers
              $grp1$ and $grp2$ are of the same types and that group $grp1$ is
              a subset of the group $grp2$.

              \begin{vverbatim}
  subst(<grp1>, <grp2>)
              \end{vverbatim}
          \end{itemize}

          Atoms that contain no variables, i.e. composed entirely of entity
          identifiers, are called {\em ground atoms}.

        \paragraph{Facts.}
          A fact states that the relationship represented by an atom or
          its negation holds in the current context. Facts are negated by the
          use of the negation operator ($!$). The following shows the formal
          syntax of a fact:

          \begin{vverbatim}
  [!]<holds\_atom>|<memb\_atom>|
     <subst\_atom>
          \end{vverbatim}

          Note that facts may be made up of atoms that contain variable
          identifiers. Facts with no variable occurrences are called
          {\em ground facts}.

        \paragraph{Expressions.}
          An expression is either a fact, or a logical conjunction of facts,
          separated by the double-ampersand characters $\&\&$.

          \begin{vverbatim}
  <fact1> [&& <fact2> [&& ...]]
          \end{vverbatim}

          Expressions that are made up of only ground facts are called
          {\em ground expressions}.

      \subsubsection{Definition Statements.}

        These statements are used to define the different rules that make up
        the policy base.

        \paragraph{Entity Identifier Definition.}

          All entity identifiers (subjects, access rights, objects and groups)
          must first be declared before any other statements to define the
          entity domain of the policy base. The following entity declaration
          syntax illustrates how to define one or more entity identifiers of a
          particular type.

          \begin{vverbatim}
  ident sub|acc|obj[-grp]
    <entity\_id>[, ...];
          \end{vverbatim}

        \paragraph{Initial Fact Definition.}

          The initial facts of the policy base, those that hold before any
          policy updates are performed, are defined by using the following
          definition syntax:

          \begin{vverbatim}
  initially <ground\_exp>;
          \end{vverbatim}

        \paragraph{Constraint Definition.}

          A constraint statement is a logical rule that holds regardless of any
          changes that may occur when the policy base is updated. Constraint
          rules are true in the initial state and remain true after any policy
          update.

          The constraint syntax below shows that for any state of the policy
          base, expression $exp1$ holds if expression $exp2$ is true and there
          is no evidence that $exp3$ is true. The $with$ $absence$ clause
          allows constraints to have a default proposition behaviour, where
          the absence of proof that an expression holds  satisfies the clause
          condition of the proposition.

          It is important to note that the expressions $exp1$, $exp2$ and
          $exp3$ may be non-ground expressions, which allows identifiers
          occurring in these expressions to be variables.

          \begin{vverbatim}
  always <exp1>
    [implied by <exp2>
    [with absence <exp3>]];
          \end{vverbatim}

        \paragraph{Policy Update Definition.}

          Before a policy update can be applied, it must first be defined by
          using the following syntax:

          \begin{vverbatim}
  <upd\_id>([<var\_id>[, ...]])
    causes <exp1>
    [if <exp2>];
          \end{vverbatim}

          $upd\_id$ is the policy update identifier to be used in referencing
          this policy update. The optional parameter $var\_id$ is a list which
          contains the variable identifiers occurring in the expressions $exp1$
          and $exp2$ and will eventually be replaced by entity identifiers when
          the update is referenced. The postcondition expression $exp1$ is an
          expression that will hold in the state after this update is applied.
          The expression $exp2$ is a precondition expression that must hold in
          the current state before this update is applied.

          Note that a policy update definition will have no effect on the
          policy base until it is applied by one of the directives described in
          the following section.

        \subsubsection{Directive Statements.}

          These statements are used to issue policy update and query directives
          to the PolicyUpdater system.

        \paragraph{Policy Update Directives.}

        The policy update sequence list contains a list of references to
        define policy updates in the domain. The policy updates in the
        sequence list are applied to the current state of the policy base one
        at a time to produce a policy base state upon which queries can be
        evaluated.

        The following four directives are the policy sequence manipulation
        features of language $\cal{L}$.

        \begin{itemize}
          \item
            {\em Adding an update into the sequence.}
            Defined policy updates are added into the sequence list through the
            use of the following directive:

            \begin{vverbatim}
  seq add <upd\_id>([<elt\_id>[, ...]]);
            \end{vverbatim}

            \noindent where $upd\_id$ is the identifier of a defined policy
            update and the $elt\_id$ list is a comma-separated list of entity
            identifiers that will replace the variable identifiers that occur
            in the definition of the policy update.

          \item
            {\em Listing the updates in the sequence.}
            The following directive may be used to list the current contents of
            the policy update sequence list.

            \begin{vverbatim}
  seq list;
            \end{vverbatim}

            This directive is answered with an ordinal list of policy updates
            in the form:

            \begin{vverbatim}
  <n> <upd\_id>([<elt\_id>[, ...]])
            \end{vverbatim}

            \noindent where $n$ is the ordinal index of the policy update
            within the sequence list starting at 0. $upd\_id$ is the policy
            update identifier and the $elt\_id$ list is the list of entity
            identifiers used to replace the variable identifier place-holders.

          \item
            {\em Removing an update from the sequence.}
            The syntax below shows the directive to remove a policy update
            reference from the list. $n$ is the ordinal index of the policy
            update to be removed. Note that removing a policy update reference
            from the sequence list may change the ordinal index of other update
            references.

            \begin{vverbatim}
  seq del <n>;
            \end{vverbatim}

          \item
            {\em Computing an update sequence.}
            The policy updates in the sequence list does not get applied until
            the $compute$ directive is issued. The directive causes the policy
            update references in the sequence list to be applied one at a time
            in the same order that they appear in the list. The directive also
            causes the system to generate the policy base models against which
            query requests can be evaluated.

            \begin{vverbatim}
  compute;
            \end{vverbatim}
        \end{itemize}

        \paragraph{Query Directive.}

          A ground query expression may be issued against the current state of
          the policy base. This current state is derived after all the updates
          in the update sequence have been applied, one at a time, upon the
          initial state. Query expressions are answered with a $true$, $false$
          or an $unknown$, depending on whether the queried expression holds,
          its negation holds, or neither, respectively. Syntax is as follows:

          \begin{vverbatim}
  query <ground\_exp>;
          \end{vverbatim}

        \begin{vexample}
          \label{ex-1}
          The following language ${\cal L}$ program code listing shows a simple
          rule-based document access control system scenario.

          In this example, the subject $alice$ is initially a member of the
          subject group $grp2$, which is a subset of group $grp1$. The group
          $grp1$ also initially holds a $read$ access right for the object
          $file$. The constraint states that if the group $grp1$ has $read$
          access for $file$, and no other information is present to indicate
          that $grp3$ does not have $write$ access for $file$, then the group
          $grp1$ is granted $write$ access for $file$. For simplicity, we only
          consider one policy update $delete\_read$ and a few queries that are
          evaluated after the policy update is performed.

          \begin{vverbatim}
  ident sub alice;
  ident sub-grp grp1, grp2, grp3;
  ident acc read, write;
  ident obj file;

  initially
    memb(alice, grp2) &&
    holds(grp1, read, file) &&
    subst(grp2, grp1);

  always holds(grp1, write, file)
    implied by
      holds(grp1, read, file)
    with absence
      !holds(grp3, write, file);

  delete\_read(SG0, OS0)
    causes !holds(SG0, read, OS0);

  seq add delete\_read(grp1, file);

  compute;

  query holds(grp1, write, file);
  query holds(grp1, read, file);
  query holds(alice, write, file);
  query holds(alice, read, file);
          \end{vverbatim}
        \end{vexample}

    \subsection{Semantics}
      \label{subsec-semantics}

      After giving a detailed syntactic definition of language ${\cal L}$,
      we now define its formal semantics.

      The semantics for language ${\cal L}$ is based on the well-known answer
      set (stable model) semantics of extended logic programs proposed by
      Gelfond and Lifschitz \cite{GEL}. The definition below formally defines
      the answer set of a logic program.

      \begin{vdefinition}
        \label{def-ans}
        Given an extended logic program $\pi$ composed of ground facts and
        rules that do not have the negation-as-failure operator $not$ and a set
        ${\cal F}$ of all ground facts in $\pi$. A set $\lambda$ is then said to
        be an answer set of $\pi$ if it is the smallest set that satisfies the
        following conditions:

        \begin{enumerate}
          \item
            For any rule of the form $\rho_{0}$ $\leftarrow$ $\rho_{1}$,
            $\hdots$, $\rho_{n}$ where $n$ $\geq$ 1, if $\rho_{1}$, $\hdots$,
            $\rho_{n}$ $\in$ $\lambda$, then
            $\rho_{0}$ $\in$ $\lambda$.
          \item
            If $\lambda$ contains a pair of complementary facts (i.e. a fact and
            its negation), then $\lambda$ = ${\cal F}$.
        \end{enumerate}

        For a ground extended logic program $\pi$ that is composed of rules
        that may have the negation-as-failure operator $not$, a set $\lambda$ is
        the answer set of $\pi$ if and only if $\lambda$ is the answer set of
        $\pi'$, where $\pi'$ is obtained from $\pi$ by deleting the following:

        \begin{enumerate}
          \item
            Each rule that contains a fact of the form $not$ $\rho$ in its body
            where $\rho$ $\in$ $\lambda$.
          \item
            All facts of the form $not$ $\rho$ in the bodies of the remaining
            rules.
        \end{enumerate}
      \end{vdefinition}

      \subsubsection{Domain Description of Language ${\cal L}$.}

        The definition below gives a formal definition of the domain
        description of language ${\cal L}$.

        \begin{vdefinition}
          \label{def-domain}
          The domain description ${\cal D}_{\cal L}$ of language ${\cal L}$ is
          defined as a finite set of ground initial state facts, constraint
          rules and policy update definitions.
        \end{vdefinition}

        In addition to the domain description ${\cal D}_{\cal L}$, language
        ${\cal L}$ also includes an additional ordered set: the sequence list
        $\psi$. The sequence list $\psi$ is an ordered set that contains a
        sequence of references to policy update definitions. Each policy update
        reference consists of the policy update identifier and a series of zero
        or more identifier entities to replace the variable place-holders in
        the policy update definitions.

      \subsubsection{Language ${\cal L}^{*}$.}

        In language ${\cal L}$, the policy base is subject to change, which is
        triggered by the application of policy updates. Such changes bring
        forth the concept of policy base states. Conceptually, a state may be
        thought of as a set of facts and constraints of the policy base at a
        particular instant. The state transition notation below shows that a
        new state $PB'$ is generated from the current state $PB$ after the
        policy update $u$ is applied.

        \begin{vquote}
          $PB$ $\overrightarrow{_{u}}$ $PB'$
        \end{vquote}

        This concept of a state means that for every policy update applied
        to the policy base, a new instance of the policy base or a new set of
        facts and constraints are generated. To precisely define the underlying
        semantics of domain description ${\cal D}_{\cal L}$ in language
        ${\cal L}$, we introduce language ${\cal L}^{*}$, which is an extended
        logic program representation of language ${\cal L}$, with state as an
        explicit sort.

        Language ${\cal L}^{*}$ contains only one special state constant
        $S_{0}$ to represent the initial state of a given domain description.
        All other states are represented as a resulting state obtained by
        applying the $Res$ function.

        The $Res$ function takes a policy update reference $u$ ($u$ $\in$
        $\psi$) and the current state $\sigma$ as input arguments and
        returns the resulting state $\sigma'$ after update $u$ has been applied
        to state $\sigma$:

        \begin{vquote}
          $\sigma'$ = $Res$($u$, $\sigma$)
        \end{vquote}

        Given an initial state $S_{0}$ and a policy update sequence list
        $\psi$, each state $\sigma_{i}$ ($0$ $\leq$ $i$ $\leq$ $|\psi|$)
        may be represented as follows:

        \begin{vquote}
          $\sigma_{0}$ = $S_{0}$

          $\sigma_{1}$ = $Res$($u_{0}$, $\sigma_{0}$)

          $\vdots$

          $\sigma_{|\psi|}$ = $Res$($u_{|\psi| - 1}$, $\sigma_{|\psi| - 1}$)
        \end{vquote}

        Substituting each state with a recursive call to the $Res$ function,
        the final state $S_{|\psi|}$ is defined as follows:

        \begin{vquote}
          $S_{|\psi|}$ = $Res$($u_{|\psi| - 1}$, $Res$($\ldots$, $Res$($u_{0}$, $S_{0}$)))
        \end{vquote}

        \paragraph{Entities.}

          The entity set ${\cal E}$ is the union of six disjoint entity sets:
          single subject ${\cal E}_{ss}$, group subject ${\cal E}_{sg}$,
          single access right ${\cal E}_{as}$, group access right
          ${\cal E}_{ag}$, single object ${\cal E}_{os}$ and group object
          ${\cal E}_{og}$. Each entity in set ${\cal E}$ corresponds directly
          to the {\em entity identifiers} of language ${\cal L}$.

          \begin{vquote}
            ${\cal E}$ =
            ${\cal E}_{s}$ $\cup$ ${\cal E}_{a}$ $\cup$ ${\cal E}_{o}$

            ${\cal E}_{s}$ = ${\cal E}_{ss}$ $\cup$ ${\cal E}_{sg}$

            ${\cal E}_{a}$ = ${\cal E}_{as}$ $\cup$ ${\cal E}_{ag}$

            ${\cal E}_{o}$ = ${\cal E}_{os}$ $\cup$ ${\cal E}_{og}$
          \end{vquote}

        \paragraph{Atoms.}

          The main difference between language ${\cal L}$ and language
          ${\cal L}^{*}$ lies in the definition of an atom. Atoms in language
          ${\cal L}^{*}$ represent a logical relationship of two to three
          entities, as with atoms of language ${\cal L}$. Furthermore, atoms of
          language ${\cal L}^{*}$ extends this definition by defining the
          state of the policy base in which the relationship holds. In this
          paper, atoms of language ${\cal L}^{*}$ are written with the
          hat character ($\hat{holds}$, $\hat{memb}$ and $\hat{subst}$) to
          differentiate from the atoms of language ${\cal L}$.

          The atom set ${\cal A}^{\sigma}$ is the set of all atoms in state
          $\sigma$.

          \begin{vquote}
            ${\cal A}^{\sigma}$ =
            ${\cal A}^{\sigma}_{h}$ $\cup$ ${\cal A}^{\sigma}_{m}$ $\cup$
            ${\cal A}^{\sigma}_{s}$

            ${\cal A}^{\sigma}_{h}$ =
            $\{\hat{holds}$($s$, $a$, $o$, $\sigma$) $\mid$ $s$ $\in$
            ${\cal E}_{s}$, $a$ $\in$ ${\cal E}_{a}$, $o$ $\in$
            ${\cal E}_{o}\}$

            ${\cal A}^{\sigma}_{m}$ =
            ${\cal A}^{\sigma}_{ms}$ $\cup$ ${\cal A}^{\sigma}_{ma}$ $\cup$
            ${\cal A}^{\sigma}_{mo}$

            ${\cal A}^{\sigma}_{s}$ =
            ${\cal A}^{\sigma}_{ss}$ $\cup$ ${\cal A}^{\sigma}_{sa}$ $\cup$
            ${\cal A}^{\sigma}_{so}$

            ${\cal A}^{\sigma}_{ms}$ =
            $\{\hat{memb}$($e$, $g$, $\sigma$) $\mid$ $e$ $\in$
            ${\cal E}_{ss}$, $g$ $\in$ ${\cal E}_{sg}\}$

            ${\cal A}^{\sigma}_{ma}$ =
            $\{\hat{memb}$($e$, $g$, $\sigma$) $\mid$ $e$ $\in$
            ${\cal E}_{as}$, $g$ $\in$ ${\cal E}_{ag}\}$

            ${\cal A}^{\sigma}_{mo}$ =
            $\{\hat{memb}$($e$, $g$, $\sigma$) $\mid$ $e$ $\in$
            ${\cal E}_{os}$, $g$ $\in$ ${\cal E}_{og}\}$

            ${\cal A}^{\sigma}_{ss}$ = $\{\hat{subst}$($g_{1}$, $g_{2}$,
            $\sigma$) $\mid$ $g_{1}$, $g_{2}$ $\in$ ${\cal E}_{sg}\}$

            ${\cal A}^{\sigma}_{sa}$ =
            $\{\hat{subst}(g_{1}, g_{2}, \sigma)$ $\mid$ $g_{1}$, $g_{2}$ $\in$
            ${\cal E}_{ag}\}$

            ${\cal A}^{\sigma}_{so}$ =
            $\{\hat{subst}(g_{1}, g_{2}, \sigma)$ $\mid$ $g_{1}$, $g_{2}$ $\in$
            ${\cal E}_{og}\}$
          \end{vquote}

        \paragraph{Facts.}

          A fact is a logical statement that makes a claim that an atom either
          holds or does not hold at a particular state. The following is the
          formal definition of fact $\hat{\rho}$ in state $\sigma$:

          \begin{vquote}
            $\hat{\rho}^{\sigma}$ =
            $[\lnot]$$\hat{\alpha}$, $\hat{\alpha}$ $\in$ ${\cal A}^{\sigma}$
          \end{vquote}

      \subsubsection{Translating Language ${\cal L}$ to Language ${\cal L^{*}}$.}

        Given a domain description ${\cal D_{L}}$ of language ${\cal L}$, we
        translate ${\cal D_{L}}$ into an extended logic program of language
        ${\cal L^{*}}$, as denoted by $Trans$(${\cal D_{L}}$). The semantics of
        ${\cal D_{L}}$ are provided by the answer sets of the extended logic
        program $Trans$(${\cal D_{L}}$). Before we can fully define
        $Trans$(${\cal D}_{\cal L}$), we must first define the following
        functions:

        The $CopyAtom$ function takes two arguments: an atom $\hat{\alpha}$
        of language ${\cal L}^{*}$ at some state $\sigma$ and another state
        $\sigma'$. The function returns an equivalent atom of the same type
        and with the same entities, but in the new state specified.

        \begin{vquote}
          $CopyAtom$($\hat{\alpha}$, $\sigma'$)

          \hspace{1em}
          =
          \begin{math}
            \begin{cases}
              \mbox{$\hat{holds}$($s$, $a$, $o$, $\sigma'$), if $\hat{\alpha}$ = $\hat{holds}$($s$, $a$, $o$, $\sigma$)} \\
              \mbox{$\hat{memb}$($e$, $g$, $\sigma'$), if $\hat{\alpha}$ = $\hat{memb}$($e$, $g$, $\sigma$)} \\
              \mbox{$\hat{subst}$($g_{1}$, $g_{2}$, $\sigma'$), if $\hat{\alpha}$ = $\hat{subst}$($g_{1}$, $g_{2}$, $\sigma$)}
            \end{cases}
          \end{math}
        \end{vquote}

        Another function, $TransAtom$, takes an atom $\alpha$ of language
        ${\cal L}$ and an arbitrary state $\sigma$ and returns the equivalent
        atom of language ${\cal L}^{*}$.

        \begin{vquote}
          $TransAtom$($\alpha$, $\sigma$)

          \hspace{1em}
          =
          \begin{math}
            \begin{cases}
              \mbox{$\hat{holds}$($s$, $a$, $o$, $\sigma$), if $\alpha$ = $holds$($s$, $a$, $o$)} \\
              \mbox{$\hat{memb}$($e$, $g$, $\sigma$), if $\alpha$ = $memb$($e$, $g$)} \\
              \mbox{$\hat{subst}$($g_{1}$, $g_{2}$, $\sigma$), if $\alpha$ = $subst$($g_{1}$, $g_{2}$)}
            \end{cases}
          \end{math}
        \end{vquote}

        The $TransFact$ function is similar to the $TransAtom$
        function, but instead of translating an atom, it takes a fact
        from language ${\cal L}$ and a state then returns the equivalent
        fact in language ${\cal L}^{*}$.

        \paragraph{Initial Fact Rules.}

          The process of translating initial fact expressions of language
          ${\cal L}$ to language ${\cal L}^{*}$ rules is a trivial procedure:
          translate each fact that make up the initial fact expression of
          language ${\cal L}$ with its corresponding equivalent initial state
          atom of language ${\cal L}^{*}$. Given the following {\em initially}
          statement in language ${\cal L}$:

          \begin{vverbatim}
  initially \(\rho\sb{0}\) && \ldots && \(\rho\sb{n}\);
          \end{vverbatim}

          \noindent
          The language ${\cal L}^{*}$ translation of this statement is shown
          below:

          \begin{vquote}
            $\hat{\rho}_{0}$ $\leftarrow$

            $\vdots$

            $\hat{\rho}_{n}$ $\leftarrow$
          \end{vquote}

          \begin{vquote}
            where

            $\hat{\rho}_{i}$ = $TransFact$($\rho_{i}$, $S_{0}$),

            $0$ $\leq$ $i$ $\leq$ $n$
          \end{vquote}

          As shown above, the number of initial fact rules generated from the
          translation is the number of facts $n$ in the given language
          ${\cal L}$ initial fact expression.

          The following code shows a more realistic example of language
          ${\cal L}$ $initially$ statements:

          \begin{vverbatim}
  initially
    holds(admins, read, sys\_data) &&
    memb(alice, admins);

  initially
    memb(bob, admins);
          \end{vverbatim}

        \noindent
        In language ${\cal L}^{*}$, the above statements are translated to:

        \begin{vquote}
          $\hat{holds}$($admins$, $read$, $sys\_data$, $S_{0}$) $\leftarrow$

          $\hat{memb}$($alice$, $admins$, $S_{0}$) $\leftarrow$

          $\hat{memb}$($bob$, $admins$, $S_{0}$) $\leftarrow$
        \end{vquote}

        \paragraph{Constraint Rules.}

          Each constraint rule in language ${\cal L}$ is expressed as a series
          of logical rules in language ${\cal L}^{*}$. Given that all variable
          occurrences have been grounded to entity identifiers, a constraint in
          language ${\cal L}$, with $m$, $n$, $o$ $\geq$ $0$ may be represented
          as:

          \begin{vverbatim}
  always \(a\sb{0}\) && \ldots && \(a\sb{m}\)
    implied by \(b\sb{0}\) && \ldots && \(b\sb{n}\)
    with absence \(c\sb{0}\) && \ldots && \(c\sb{o}\);
          \end{vverbatim}

          Each fact in the $always$ clause of language ${\cal L}$ corresponds
          to a new rule, where it is the consequent. Each of these new rules
          will have expression $b$ in the $implied$ $by$ clause as the positive
          premise and the expression $c$ in the $with$ $absence$ clause as the
          negative premise.

          \begin{vquote}
            $a_{0}$ $\leftarrow$
            $b_{0}$, \ldots, $b_{n}$,
            $not$ $c_{0}$, \ldots, $not$ $c_{o}$

            $\vdots$

            $a_{m}$ $\leftarrow$
            $b_{0}$, \ldots, $b_{n}$,
            $not$ $c_{0}$, \ldots, $not$ $c_{o}$
          \end{vquote}

          Under the definition of constraints, each of the rules listed above
          must be made to hold in all states as defined by the sequence list
          $\psi$. This can be accomplished by translating each of the above
          rules to a set of $|\psi|$ rules, one for each state.

           \begin{vquote}
            $\hat{a}^{S_{0}}_{0}$ $\leftarrow$
            $\hat{b}^{S_{0}}_{0}$, \ldots, $\hat{b}^{S_{0}}_{n}$,
            $not$ $\hat{c}^{S_{0}}_{0}$, \ldots, $not$ $\hat{c}^{S_{0}}_{o}$

            $\vdots$

            $\hat{a}^{S_{|\psi|}}_{0}$ $\leftarrow$
            $\hat{b}^{S_{|\psi|}}_{0}$, \ldots, $\hat{b}^{S_{|\psi|}}_{n}$,
            $not$ $\hat{c}^{S_{|\psi|}}_{0}$, \ldots, $not$ $\hat{c}^{S_{|\psi|}}_{o}$

            $\vdots$

            $\hat{a}^{S_{0}}_{m}$ $\leftarrow$
            $\hat{b}^{S_{0}}_{0}$, \ldots, $\hat{b}^{S_{0}}_{n}$,
            $not$ $\hat{c}^{S_{0}}_{0}$, \ldots, $not$ $\hat{c}^{S_{0}}_{o}$

            $\vdots$

            $\hat{a}^{S_{|\psi|}}_{m}$ $\leftarrow$
            $\hat{b}^{S_{|\psi|}}_{0}$, \ldots, $\hat{b}^{S_{|\psi|}}_{n}$,
            $not$ $\hat{c}^{S_{|\psi|}}_{0}$, \ldots, $not$ $\hat{c}^{S_{|\psi|}}_{o}$
          \end{vquote}

          \begin{vquote}
            where

            $\hat{a}^{\sigma}_{i}$ = $TransFact$($a_{i}$, $\sigma$),
            $0$ $\leq$ $i$ $\leq$ $m$,

            $\hat{b}^{\sigma}_{j}$ = $TransFact$($b_{j}$, $\sigma$),
            $0$ $\leq$ $j$ $\leq$ $n$,

            $\hat{c}^{\sigma}_{k}$ = $TransFact$($c_{k}$, $\sigma$),
            $0$ $\leq$ $k$ $\leq$ $o$,

            $S_{0}$ $\leq$ $\sigma$ $\leq$ $S_{|\psi|}$
          \end{vquote}

          For a given language ${\cal L}$ constraint rule, the number of
          constraint rules generated in the translation is:

          \begin{quote}
            $m$ $|\psi|$
          \end{quote}

          \begin{quote}
            where

            $m$ is the number of facts in the $always$ clause

            $|\psi|$ is the number of states
          \end{quote}

          The example below shows how the following language ${\cal L}$ code
          fragment is translated to language ${\cal L}^{*}$:

          \begin{vverbatim}
  always
    holds(alice, read, data) &&
    holds(alice, write, data)
  implied by
    memb(alice, admin)
  with absence
    !holds(alice, own, data);
          \end{vverbatim}

          Given a policy update reference in the sequence list $\psi$ (i.e.
          $|\psi|$ = $1$), the language ${\cal L}^{*}$ equivalent is as
          follows:

          \begin{vquote}
            $\hat{holds}$($alice$, $read$, $data$, $S_{0}$) $\leftarrow$

            \hspace{1em}
            $\hat{memb}$($alice$, $admin$, $S_{0}$),

            \hspace{1em}
            $not$ $\lnot\hat{holds}$($alice$, $own$, $data$, $S_{0}$)

            $\hat{holds}$($alice$, $write$, $data$, $S_{0}$) $\leftarrow$

            \hspace{1em}
            $\hat{memb}$($alice$, $admin$, $S_{0}$),

            \hspace{1em}
            $not$ $\lnot\hat{holds}$($alice$, $own$, $data$, $S_{0}$)
          \end{vquote}

          \begin{vquote}
            $\hat{holds}$($alice$, $read$, $data$, $S_{1}$) $\leftarrow$

            \hspace{1em}
            $\hat{memb}$($alice$, $admin$, $S_{1}$),

            \hspace{1em}
            $not$ $\lnot\hat{holds}$($alice$, $own$, $data$, $S_{1}$)

            $\hat{holds}$($alice$, $write$, $data$, $S_{1}$) $\leftarrow$

            \hspace{1em}
            $\hat{memb}$($alice$, $admin$, $S_{1}$),

            \hspace{1em}
            $not$ $\lnot\hat{holds}$($alice$, $own$, $data$, $S_{1}$)
          \end{vquote}

        \paragraph{Policy Update Rules.}

          Given that $m$, $n$ $\geq$ $0$, all occurrences of variable
          place-holders grounded to entity identifiers, a policy update $u$ in
          language ${\cal L}$ is in the form:

          \begin{vverbatim}
  \(u\) causes \(a\sb{0}\) && \ldots && \(a\sb{m}\)
  if \(b\sb{0}\) && \ldots && \(b\sb{n}\);
          \end{vverbatim}

          In language ${\cal L}^{*}$, such policy updates may be represented as
          a set of implications, with each fact $a$ in the postcondition
          expression as the consequent and precondition expression $b$ as the
          premise. However, the translation process must also take into account
          that the premise of the implication holds in the state before the
          policy update is applied and that the consequent holds in the state
          after the application.

          \begin{vquote}
            $\hat{a}_{0}$ $\leftarrow$ $\hat{b}_{0}$, \ldots, $\hat{b}_{n}$

            $\vdots$

            $\hat{a}_{m}$ $\leftarrow$ $\hat{b}_{0}$, \ldots, $\hat{b}_{n}$
          \end{vquote}

          \begin{vquote}
            where

            $\hat{a}_{i}$ = $TransFact$($a_{i}$, $Res$($u$, $\sigma$)),
            $0$ $\leq$ $i$ $\leq$ $m$,

            $\hat{b}_{j}$ = $TransFact$($b_{j}$, $\sigma$),
            $0$ $\leq$ $j$ $\leq$ $n$
          \end{vquote}

          Intuitively, given a language ${\cal L}$ policy update definition,
          the number of language ${\cal L}^{*}$ rules generated in the
          translation is $m$, which is the number of facts in the
          postcondition expression.

          For example, given the following 2 language ${\cal L}$ policy update
          definitions:

          \begin{vverbatim}
  grant\_read()
    causes holds(alice, read, file)
    if memb(alice, readers);

  grant\_write()
    causes holds(alice, write, file)
    if memb(alice, writers);
          \end{vverbatim}

          Given the update sequence list $\psi$ contains
          \{$grant\_read$, $grant\_write$\}, the above statements are written
          in language ${\cal L}^{*}$ as:

          \begin{vquote}
            $\hat{holds}$($alice$, $read$, $file$, $S_{1}$) $\leftarrow$

            \hspace{1em}
            $\hat{memb}$($alice$, $readers$, $S_{0}$)
          \end{vquote}

          \begin{vquote}
            $\hat{holds}$($alice$, $write$, $file$, $S_{2}$) $\leftarrow$

            \hspace{1em}
            $\hat{memb}$($alice$, $writers$, $S_{1}$)
          \end{vquote}

        \paragraph{Additional Constraints.}

          In addition to the translations discussed above, there are a few
          other implicit constraint rules implied by language ${\cal L}$
          that need to be explicitly defined in language ${\cal L}^{*}$.

          \begin{itemize}
            \item
              {\em Inheritance rules.}
              All properties held by a group is inherited by all the members
              and subsets of that group. This rule is easy to apply for subject
              group entities. However, careful attention must be given to access
              right and object groups. A subject holding an access right for an
              object group implies that the subject also holds that access
              right for all objects in the object group. Similarly, a subject
              holding an access right group for a particular object implies
              that the subject holds all access rights contained in the access
              right group for that object.

              A conflict is encountered when a particular property is to be
              inherited by an entity from a group of which it is a member or
              subset, and the contained entity already holds the negation of
              that property. This conflict is resolved by giving negative facts
              higher precedence over its positive counterpart: by allowing
              member or subset entities to inherit its parent group's
              properties only if the entities do not already hold the negation
              of those properties.

              The following are the inheritance constraint rules to allow the
              properties held by a group to propagate to its members and
              subsets that do not already hold the negation of the properties.

              \begin{enumerate}
                \item
                  Subject Group Membership Inheritance

                  \begin{vquote}
                    $\forall$ ($s_{s}$, $s_{g}$, $a$, $o$, $\sigma$),
                  \end{vquote}

                  \begin{vquote}
                    $\hat{holds}$($s_{s}$, $a$, $o$, $\sigma$) $\leftarrow$

                    \hspace{1em}
                    $\hat{holds}$($s_{g}$, $a$, $o$, $\sigma$),
                    $\hat{memb}$($s_{s}$, $s_{g}$, $\sigma$),

                    \hspace{1em}
                    $not$ $\lnot\hat{holds}$($s_{s}$, $a$, $o$, $\sigma$)
                  \end{vquote}

                  \begin{vquote}
                    $\lnot\hat{holds}$($s_{s}$, $a$, $o$, $\sigma$) $\leftarrow$

                    \hspace{1em}
                    $\lnot\hat{holds}$($s_{g}$, $a$, $o$, $\sigma$),
                    $\hat{memb}$($s_{s}$, $s_{g}$, $\sigma$)
                  \end{vquote}

                  \begin{vquote}
                    where

                    $s_{s}$ $\in$ ${\cal E}_{ss}$,
                    $s_{g}$ $\in$ ${\cal E}_{sg}$,
                    $a$ $\in$ ${\cal E}_{a}$,
                    $o$ $\in$ ${\cal E}_{o}$,
                    $S_{0}$ $\leq$ $\sigma$ $\leq$ $S_{|\psi|}$
                  \end{vquote}

                \item
                  Access Right Group Membership Inheritance

                  \begin{vquote}
                    $\forall$ ($s$, $a_{s}$, $a_{g}$, $o$, $\sigma$),
                  \end{vquote}

                  \begin{vquote}
                    $\hat{holds}$($s$, $a_{s}$, $o$, $\sigma$) $\leftarrow$

                    \hspace{1em}
                    $\hat{holds}$($s$, $a_{g}$, $o$, $\sigma$),
                    $\hat{memb}$($a_{s}$, $a_{g}$, $\sigma$),

                    \hspace{1em}
                    $not$ $\lnot\hat{holds}$($s$, $a_{s}$, $o$, $\sigma$)
                  \end{vquote}

                  \begin{vquote}
                    $\lnot\hat{holds}$($s$, $a_{s}$, $o$, $\sigma$)
                    $\leftarrow$

                    \hspace{1em}
                    $\lnot\hat{holds}$($s$, $a_{g}$, $o$, $\sigma$),
                    $\hat{memb}$($a_{s}$, $a_{g}$, $\sigma$)
                  \end{vquote}

                  \begin{vquote}
                    where

                    $s$ $\in$ ${\cal E}_{s}$,
                    $a_{s}$ $\in$ ${\cal E}_{as}$,
                    $a_{g}$ $\in$ ${\cal E}_{ag}$,
                    $o$ $\in$ ${\cal E}_{o}$,
                    $S_{0}$ ${\leq}$ ${\sigma}$ ${\leq}$ $S_{|\psi|}$
                  \end{vquote}

                 \item
                  Object Group Membership Inheritance

                  \begin{vquote}
                    $\forall$ ($s$, $a$, $o_{s}$, $o_{g}$, $\sigma$),
                  \end{vquote}

                  \begin{vquote}
                    $\hat{holds}$($s$, $a$, $o_{s}$, $\sigma$) $\leftarrow$

                    \hspace{1em}
                    $\hat{holds}$($s$, $a$, $o_{g}$, $\sigma$),
                    $\hat{memb}$($o_{s}$, $o_{g}$, $\sigma$),

                    \hspace{1em}
                    $not$ $\lnot\hat{holds}$($s$, $a$, $o_{s}$, $\sigma$)
                  \end{vquote}

                  \begin{vquote}
                    $\lnot\hat{holds}$($s$, $a$, $o_{s}$, $\sigma$)
                    $\leftarrow$

                    \hspace{1em}
                    $\lnot\hat{holds}$($s$, $a$, $o_{g}$, $\sigma$),
                    $\hat{memb}$($o_{s}$, $o_{g}$, $\sigma$)
                  \end{vquote}

                  \begin{vquote}
                    where

                    $s$ $\in$ ${\cal E}_{s}$,
                    $a$ $\in$ ${\cal E}_{a}$,
                    $o_{s}$ $\in$ ${\cal E}_{os}$,
                    $o_{g}$ $\in$ ${\cal E}_{og}$,
                    $S_{0}$ ${\leq}$ ${\sigma}$ ${\leq}$ $S_{|\psi|}$
                  \end{vquote}

                \item
                  Subject Group Subset Inheritance

                  \begin{vquote}
                    $\forall$ ($s_{g1}$, $s_{g2}$, $a$, $o$, $\sigma$),
                  \end{vquote}

                  \begin{vquote}
                    $\hat{holds}$($s_{g1}$, $a$, $o$, $\sigma$) $\leftarrow$

                    \hspace{1em}
                    $\hat{holds}$($s_{g2}$, $a$, $o$, $\sigma$),
                    $\hat{subst}$($s_{g1}$, $s_{g2}$, $\sigma$),

                    \hspace{1em}
                    $not$ $\lnot\hat{holds}$($s_{g1}$, $a$, $o$, $\sigma$)
                  \end{vquote}

                  \begin{vquote}
                    $\lnot\hat{holds}$($s_{g1}$, $a$, $o$, $\sigma$)
                    $\leftarrow$

                    \hspace{1em}
                    $\lnot\hat{holds}$($s_{g2}$, $a$, $o$, $\sigma$),
                    $\hat{subst}$($s_{g1}$, $s_{g2}$, $\sigma$)
                  \end{vquote}

                  \begin{vquote}
                    where

                    $s_{g1}$, $s_{g2}$ $\in$ ${\cal E}_{sg}$,
                    $a$ $\in$ ${\cal E}_{a}$,
                    $o$ $\in$ ${\cal E}_{o}$,
                    $s_{g1}$ $\neq$ $s_{g2}$,

                    $S_{0}$ $\leq$ $\sigma$ $\leq S_{|\psi|}$
                  \end{vquote}

                \item
                  Access Right Group Subset Inheritance

                  \begin{vquote}
                    $\forall$ ($s$, $a_{g1}$, $a_{g2}$, $o$, $\sigma$),
                  \end{vquote}

                  \begin{vquote}
                    $\hat{holds}$($s$, $a_{g1}$, $o$, $\sigma$) $\leftarrow$

                    \hspace{1em}
                    $\hat{holds}$($s$, $a_{g2}$, $o$, $\sigma$),
                    $\hat{subst}$($a_{g1}$, $a_{g2}$, $\sigma$),

                    \hspace{1em}
                    $not$ $\lnot$ $\hat{holds}$($s$, $a_{g1}$, $o$, $\sigma$)
                  \end{vquote}

                  \begin{vquote}
                    $\lnot\hat{holds}$($s$, $a_{g1}$, $o$, $\sigma$)
                    $\leftarrow$

                    \hspace{1em}
                    $\lnot\hat{holds}$($s$, $a_{g2}$, $o$, $\sigma$),
                    $\hat{subst}$($a_{g1}$, $a_{g2}$, $\sigma$)
                  \end{vquote}

                  \begin{vquote}
                    where

                    $s$ $\in$ ${\cal E}_{s}$,
                    $a_{g1}$, $a_{g2}$ $\in$ ${\cal E}_{ag}$,
                    $o$ $\in$ ${\cal E}_{o}$,
                    $a_{g1}$ $\neq$ $a_{g2}$,

                    $S_{0}$ $\leq$ $\sigma$ $\leq$ $S_{|\psi|}$
                  \end{vquote}

                \item
                  Object Group Subset Inheritance

                  \begin{vquote}
                    $\forall$ ($s$, $a$, $o_{g1}$, $o_{g2}$, $\sigma$),
                  \end{vquote}

                  \begin{vquote}
                    $\hat{holds}$($s$, $a$, $o_{g1}$, $\sigma$) $\leftarrow$

                    \hspace{1em}
                    $\hat{holds}$($s$, $a$, $o_{g2}$, $\sigma$),
                    $\hat{subst}$($o_{g1}$, $o_{g2}$, $\sigma$),

                    \hspace{1em}
                    $not$ $\lnot\hat{holds}$($s$, $a$, $o_{g1}$, $\sigma$)
                  \end{vquote}

                  \begin{vquote}
                    $\lnot\hat{holds}$($s$, $a$, $o_{g1}$, $\sigma$)
                    $\leftarrow$

                    \hspace{1em}
                    $\lnot\hat{holds}$($s$, $a$, $o_{g2}$, $\sigma$),
                    $\hat{subst}$($o_{g1}$, $o_{g2}$, $\sigma$)
                  \end{vquote}

                  \begin{vquote}
                    where

                    $s$ $\in$ ${\cal E}_{s}$,
                    $a$ $\in$ ${\cal E}_{a}$,
                    $o_{g1}$, $o_{g2}$ $\in$ ${\cal E}_{og}$,
                    $o_{g1}$ $\neq$ $o_{g2}$,

                    $S_{0}$ $\leq$ $\sigma$ $\leq$ $S_{|\psi|}$
                  \end{vquote}
              \end{enumerate}

            \item
              {\em Transitivity rules.}
              Given three distinct groups $G$, $G'$ and $G''$. If $G$ is a
              subset of $G'$ and $G'$ is a subset of $G''$, then $G$ must also
              be a subset of $G''$. The following rules ensure that the
              transitive property of subject, access right and object groups
              hold:

              \begin{enumerate}
                \item
                  Subject Group Transitivity

                  \begin{vquote}
                    $\forall$ ($sg_{1}$, $sg_{2}$, $sg_{3}$, $\sigma$),
                  \end{vquote}

                  \begin{vquote}
                    $\hat{subst}$($sg_{1}$, $sg_{3}$, $\sigma$) $\leftarrow$
                  \end{vquote}

                  \begin{vquote}
                    \hspace{1em}
                    $\hat{subst}$($sg_{1}$, $sg_{2}$, $\sigma$),
                    $\hat{subst}$($sg_{2}$, $sg_{3}$, $\sigma$)
                  \end{vquote}

                  \begin{vquote}
                    where

                    $sg_{1}$, $sg_{2}$, $sg_{3}$ $\in$ ${\cal E}_{sg}$,
                    $sg_{1}$ $\neq$ $sg_{2}$ $\neq$ $sg_{3}$,

                    $S_{0}$ $\leq$ $\sigma$ $\leq$ $S_{|\psi|}$
                  \end{vquote}

                \item
                  Access Right Group Transitivity

                  \begin{vquote}
                    $\forall$ ($ag_{1}$, $ag_{2}$, $ag_{3}$, $\sigma$),
                  \end{vquote}

                  \begin{vquote}
                    $\hat{subst}$($ag_{1}$, $ag_{3}$, $\sigma$) $\leftarrow$
                  \end{vquote}

                  \begin{vquote}
                    \hspace{1em}
                    $\hat{subst}$($ag_{1}$, $ag_{2}$, $\sigma$),
                    $\hat{subst}$($ag_{2}$, $ag_{3}$, $\sigma$)
                  \end{vquote}

                  \begin{vquote}
                    where

                    $ag_{1}$, $ag_{2}$, $ag_{3}$ $\in$ ${\cal E}_{ag}$,
                    $ag_{1}$ $\neq$ $ag_{2}$ $\neq$ $ag_{3}$,

                    $S_{0}$ $\leq$ $\sigma$ $\leq$ $S_{|\psi|}$
                  \end{vquote}

                \item
                  Object Group Transitivity

                  \begin{vquote}
                    $\forall$ ($og_{1}$, $og_{2}$, $og_{3}$, $\sigma$),
                  \end{vquote}

                  \begin{vquote}
                    $\hat{subst}$($og_{1}$, $og_{3}$, $\sigma$) $\leftarrow$
                  \end{vquote}

                  \begin{vquote}
                    \hspace{1em}
                    $\hat{subst}$($og_{1}$, $og_{2}$, $\sigma$),
                    $\hat{subst}$($og_{2}$, $og_{3}$, $\sigma$)
                  \end{vquote}

                  \begin{vquote}
                    where

                    $og_{1}$, $og_{2}$, $og_{3}$ $\in$ ${\cal E}_{og}$,
                    $og_{1}$ $\neq$ $og_{2}$ $\neq$ $og_{3}$,

                    $S_{0}$ $\leq$ $\sigma$ $\leq$ $S_{|\psi|}$
                  \end{vquote}
               \end{enumerate}

            \item
              {\em Inertial rules.}
              Intuitively, all facts in the current state that are not affected
              by a policy update should be carried over to the next state after
              the update. In language ${\cal L}^{*}$, this rule must be
              explicitly stated as a constraint. Formally, the inertial rules
              are expressed as follows:

              \begin{vquote}
                $\forall$ ($\hat{\alpha}$,$u$) $\exists$$\hat{\alpha}'$,
              \end{vquote}

              \begin{vquote}
                $\hat{\alpha}'$ $\leftarrow$ $\hat{\alpha}$, $not$ $\lnot$ $\hat{\alpha}'$

                $\lnot$ $\hat{\alpha}'$ $\leftarrow$ $\lnot$ $\hat{\alpha}$, $not$ $\hat{\alpha}'$
              \end{vquote}

              \begin{vquote}
                where

                $\hat{\alpha}$ $\in$ ${\cal A}^{\sigma}$,
                $u$ $\in$ $\psi$,
                $\hat{\alpha}'$ = $CopyAtom$($\hat{\alpha}$, $Res$($u$, $\sigma$))
              \end{vquote}
            \item
              {\em Identity rules.}
              Lastly, explicit rules must be given to show that every set is
              a subset of itself.

              \begin{vquote}
                $\forall$ ($g$, $\sigma$),
              \end{vquote}

              \begin{vquote}
                $\hat{subst}$($g$, $g$, $\sigma$)
              \end{vquote}

              \begin{vquote}
                where

                $g$ $\in$ (${\cal E}_{sg}$ $\cup$ ${\cal E}_{ag}$ $\cup$ ${\cal E}_{og}$),
                $S_{0}$ $\leq$ $\sigma$ $\leq$ $S_{|\psi|}$
              \end{vquote}
          \end{itemize}

        \begin{vdefinition}
          \label{def-trans}
          Given a domain description ${\cal D_{L}}$ of language ${\cal L}$, its
          ${\cal L^{*}}$ translation $Trans$(${\cal D_{L}}$) is an extended
          logic program of language ${\cal L}$ consisting of: (1) initial fact
          rules, (2) constraint rules, (3) policy update rules, (4) inheritance
          rules, (5) transitivity rules, (6) inertial rules, and (7) identity
          rules as described above.

          The domain description ${\cal D_{L}}$ of language ${\cal L}$ is said
          to be {\em consistent} if and only if the translation
          $Trans$(${\cal D_{L}}$) has a consistent answer set.
        \end{vdefinition}

        Appendix \ref{app-trans} shows the language ${\cal L^{*}}$ translation
        of the language ${\cal L}$ code listing shown in Example \ref{ex-1}.
        Note that given a domain description ${\cal D_{L}}$, the translation
        $Trans$(${\cal D_{L}}$) may contain more rules than the original
        statements in ${\cal D_{L}}$. However, as the theorem below defines the
        maximum number of rules generated in a translation
        $Trans$(${\cal D_{L}}$), it shows that the size of a translated domain
        $|$$Trans$(${\cal D_{L}}$)$|$ can only be polynomially larger than the
        size of the given domain $|$${\cal D_{L}}$$|$. Therefore, from a
        computational viewpoint, computing the answer sets of
        $Trans$(${\cal D_{L}}$) is always feasible.

        \begin{vtheorem}{Translation Size}
          \label{the-size}
          Given a domain description ${\cal D_{L}}$; the sets ${\cal S}_{i}$,
          ${\cal S}_{c}$ and ${\cal S}_{u}$ containing the initially,
          constraint and policy update statements in ${\cal D_{L}}$,
          respectively; the set ${\cal E}$ containing all the entities in
          ${\cal D_{L}}$, including its subsets ${\cal E}_{s}$, ${\cal E}_{a}$,
          ${\cal E}_{s}$, ${\cal E}_{ss}$, ${\cal E}_{as}$, ${\cal E}_{os}$,
          ${\cal E}_{sg}$, ${\cal E}_{ag}$, ${\cal E}_{og}$; the set ${\cal A}$
          containing all the atoms in ${\cal D_{L}}$; the maximum number
          of facts $M_{i}$ in the expression of any $initially$ statement in
          ${\cal S}_{i}$; the maximum number of facts $M_{c}$ in the $always$
          clause expression of any constraint statement in ${\cal S}_{c}$;
          the maximum number of facts $M_{u}$ in the postcondition expression
          of any policy update statement in ${\cal S}_{u}$; and finally the
          policy update sequence list $\psi$, then the maximum size of the
          translation $Trans$(${\cal D_{L}}$) is:

         \begin{vquote}
            $|$$Trans$(${\cal D_{L}}$)$|$ $\leq$

            % initially
            \hspace{1em}
            $M_{i}$ $|{\cal S}_{i}|$ $+$

            % constraints
            \hspace{1em}
            $|\psi|$ $M_{c}$ $|{\cal S}_{c}|$ $+$

            % policy updates
            \hspace{1em}
            $|\psi|$ $M_{u}$ $+$

            % membership inheritance
            \hspace{1em}
            $2$ $|\psi|$ $|{\cal E}_{ss}|$ $|{\cal E}_{sg}|$ $|{\cal E}_{a}|$ $|{\cal E}_{o}|$ $+$

            \hspace{1em}
            $2$ $|\psi|$ $|{\cal E}_{s}|$ $|{\cal E}_{as}|$ $|{\cal E}_{ag}|$ $|{\cal E}_{o}|$ $+$

            \hspace{1em}
            $2$ $|\psi|$ $|{\cal E}_{s}|$ $|{\cal E}_{a}|$ $|{\cal E}_{os}|$ $|{\cal E}_{og}|$ $+$

            % subset inheritance
            \hspace{1em}
            $2$ $|\psi|$ $|{\cal E}_{sg}|^{2}$ $|{\cal E}_{a}|$ $|{\cal E}_{o}|$ $+$

            \hspace{1em}
            $2$ $|\psi|$ $|{\cal E}_{s}|$ $|{\cal E}_{ag}|^{2}$ $|{\cal E}_{o}|$ $+$

            \hspace{1em}
            $2$ $|\psi|$ $|{\cal E}_{s}|$ $|{\cal E}_{a}|$ $|{\cal E}_{og}|^{2}$ $+$

            % transitivity rules
            \hspace{1em}
            $|\psi|$
            $($$|{\cal E}_{sg}|^{3}$ $+$
            $|{\cal E}_{ag}|^{3}$ $+$
            $|{\cal E}_{og}|^{3}$$)$ $+$

            % inertial rules
            \hspace{1em}
            $2$ $|\psi|$ $|{\cal A}|$ $+$

            % identity rules
            \hspace{1em}
            $|\psi|$
            $($$|{\cal E}_{sg}|$ $+$ $|{\cal E}_{ag}|$ $+$ $|{\cal E}_{og}|$$)$
          \end{vquote}
        \end{vtheorem}

        \begin{vproof}
          From Definition \ref{def-trans}, it follows that the size of a
          language ${\cal L}^{*}$ translation is:

          \begin{vquote}
            $|$$Trans$(${\cal D_{L}}$)$|$ =

            \hspace{1em}
            $|{\cal F}_{in}|$ $+$
            $|{\cal F}_{co}|$ $+$
            $|{\cal F}_{up}|$ $+$
            $|{\cal F}_{ih}|$ $+$
            $|{\cal F}_{tr}|$ $+$
            $|{\cal F}_{ie}|$ $+$
            $|{\cal F}_{id}|$
          \end{vquote}

          where ${\cal F}_{in}$, ${\cal F}_{co}$, ${\cal F}_{up}$,
          ${\cal F}_{ih}$, ${\cal F}_{tr}$, ${\cal F}_{ie}$, and
          ${\cal F}_{id}$ are the sets of initial fact rules, constraint rules,
          policy update rules, inheritance rules, transitivity rules,
          inertial rules, and identity rules, respectively.

          As no $initially$ statement in ${\cal S}_{i}$ contain an expression
          with more than $M_{i}$ facts, the maximum number of initial fact
          rules generated in the translation is:

          \begin{vquote}
            $|{\cal F}_{in}|$ $\leq$ $M_{i}$ $|$${\cal S}_{i}$$|$
          \end{vquote}

          Each language ${\cal L}$ constraint statement in ${\cal S}_{c}$
          corresponds to $n$ rules in language ${\cal L}^{*}$, where $n$ is
          the number of policy update states times the number of facts in the
          {\em always} clause of the statement. With $M_{c}$ as the maximal
          number of facts in the {\em always} clause of any constraint
          statement, we have:

          \begin{vquote}
            $|{\cal F}_{co}|$ $\leq$ $|\psi|$ $M_{c}$ $|$${\cal S}_{c}$$|$
          \end{vquote}

          For policy update statements, only those that are applied are
          actually translated to language ${\cal L}^{*}$. With $M_{u}$ as the
          maximal number of facts in the postcondition expression of any
          applied policy update statement, we have:

          \begin{vquote}
            $|{\cal F}_{up}|$ $\leq$ $|\psi|$ $M_{u}$
          \end{vquote}

          The total number of inheritance rules generated in the translation is
          the sum of the number of member inheritance rules and the number of
          subset inheritance rules:

          \begin{vquote}
            $|{\cal F}_{ih}|$ =
            $|{\cal F}_{ih_{m}}|$ $+$
            $|{\cal F}_{ih_{s}}|$
          \end{vquote}

          Since the membership inheritance rules show the relationships between
          every possible combination of single and group entities times the
          number of states times 2 (for negative facts), we have:

          \begin{vquote}
            $|{\cal F}_{ih_{m}}|$ =

            \hspace{1em}
            $2$ $|\psi|$ $|{\cal E}_{ss}|$ $|{\cal E}_{sg}|$ $|{\cal E}_{a}|$ $|{\cal E}_{o}|$ $+$

            \hspace{1em}
            $2$ $|\psi|$ $|{\cal E}_{s}|$ $|{\cal E}_{as}|$ $|{\cal E}_{ag}|$ $|{\cal E}_{o}|$ $+$

            \hspace{1em}
            $2$ $|\psi|$ $|{\cal E}_{s}|$ $|{\cal E}_{a}|$ $|{\cal E}_{os}|$ $|{\cal E}_{og}|$
          \end{vquote}

          For subset inheritance rules, only the relationships between group
          entities are considered:

          \begin{vquote}
            $|{\cal F}_{ih_{s}}|$ =

            \hspace{1em}
            $2$ $|\psi|$ $|{\cal E}_{sg}|^{2}$ $|{\cal E}_{a}|$ $|{\cal E}_{o}|$ $+$

            \hspace{1em}
            $2$ $|\psi|$ $|{\cal E}_{s}|$ $|{\cal E}_{ag}|^{2}$ $|{\cal E}_{o}|$ $+$

            \hspace{1em}
            $2$ $|\psi|$ $|{\cal E}_{s}|$ $|{\cal E}_{a}|$ $|{\cal E}_{og}|^{2}$
          \end{vquote}

          As transitivity rules enumerate every possible combinations of any
          three group entities, for each entity type, the total number of
          transitivity rules is shown below:

          \begin{vquote}
            $|$${\cal F}_{tr}$$|$ =
            $|\psi|$
            ($|{\cal E}_{sg}|^{3}$ $+$
            $|{\cal E}_{ag}|^{3}$ $+$
            $|{\cal E}_{og}|^{3}$)
          \end{vquote}

          A single atom in language ${\cal L}$ corresponds to $n$ inertial
          rules in language ${\cal L}^{*}$, where $n$ is the number of states
          times 2 (for negative facts). This means the total number of
          inertial rules generated is:

          \begin{vquote}
            $|{\cal F}_{ie}|$ = $2$ $|\psi|$ $|{\cal A}|$
          \end{vquote}

          Lastly, the total number of identity rules is equal to the total
          number of group entities times the number of states:

          \begin{vquote}
            $|{\cal F}_{id}|$ =
            $|\psi|$
            $($$|{\cal E}_{sg}|$ $+$ $|{\cal E}_{ag}|$ $+$ $|{\cal E}_{og}|$$)$
          \end{vquote}
        \end{vproof}

  \section{Domain Consistency Checking and Evaluation}
    \label{sec-cons}

    A domain description of language ${\cal L}$ must be consistent in order to
    generate a consistent answer set for the evaluation of queries. This
    section considers two issues: the problem of identifying whether a given
    domain description is consistent, and how query evaluation is performed
    given a consistent language domain description.

    Before the above issues can be considered, a few notational constructs
    should first be introduced. Given a domain description ${\cal D_{L}}$
    composed of the following language ${\cal L}$ statements:

    \begin{vverbatim}
  initially
    \(a\sb{0}\) && \ldots && \(a\sb{m}\) && !\(b\sb{0}\) && \ldots && !\(b\sb{n}\)

  always
    \(c\sb{0}\) && \ldots && \(c\sb{o}\) && !\(d\sb{0}\) && \ldots && !\(d\sb{p}\)
    implied by
    \(e\sb{0}\) && \ldots && \(e\sb{q}\) && !\(f\sb{0}\) && \ldots && !\(f\sb{r}\)
    with absence
    \(g\sb{0}\) && \ldots && \(g\sb{s}\) && !\(h\sb{0}\) && \ldots && !\(h\sb{t}\)

  update()
    causes
    \(i\sb{0}\) && \ldots && \(i\sb{u}\) && !\(j\sb{0}\) && \ldots && !\(j\sb{v}\)
    if
    \(k\sb{0}\) && \ldots && \(k\sb{w}\) && !\(l\sb{0}\) && \ldots && !\(l\sb{x}\)
    \end{vverbatim}

    Let $\gamma_{int}$ be an initial fact definition statement, $\gamma_{con}$
    a constraint definition statement, and $\gamma_{upd}$ a policy update
    definition statement, where $\gamma_{int}$, $\gamma_{con}$, $\gamma_{upd}$
    $\in$ ${\cal D}_{\cal L}$. We then define the following set constructor
    functions:

    \begin{vquote}
      ${\cal F}^{+}_{int}$($\gamma_{int}$) = \{$a_{z}$ $\mid$ $0$ $\leq$ $z$ $\leq$ $m$\}

      ${\cal F}^{-}_{int}$($\gamma_{int}$) = \{$b_{z}$ $\mid$ $0$ $\leq$ $z$ $\leq$ $n$\}

      ${\cal F}^{+}_{con}$($\gamma_{upd}$) = \{$c_{z}$ $\mid$ $0$ $\leq$ $z$ $\leq$ $o$\}

      ${\cal F}^{-}_{con}$($\gamma_{upd}$) = \{$d_{z}$ $\mid$ $0$ $\leq$ $z$ $\leq$ $p$\}

      ${\cal F}^{+}_{upd}$($\gamma_{con}$) = \{$i_{z}$ $\mid$ $0$ $\leq$ $z$ $\leq$ $u$\}

      ${\cal F}^{-}_{upd}$($\gamma_{con}$) = \{$j_{z}$ $\mid$ $0$ $\leq$ $z$ $\leq$ $v$\}
    \end{vquote}

    Using these functions, we define the following sets of ground facts:

    \begin{vquote}
      ${\cal F}^{+}_{int}$ =
      \{$\rho$ $\mid$ $\rho$ $\in$ ${\cal F}^{+}_{int}$($\gamma_{int}$), $\gamma_{int}$ $\in$ ${\cal D}_{\cal L}$\}

      ${\cal F}^{-}_{int}$ =
      \{$\rho$ $\mid$ $\rho$ $\in$ ${\cal F}^{-}_{int}$($\gamma_{int}$), $\gamma_{int}$ $\in$ ${\cal D}_{\cal L}$\}

      ${\cal F}^{+}_{con}$ =
      \{$\rho$ $\mid$ $\rho$ $\in$ ${\cal F}^{+}_{con}$($\gamma_{con}$), $\gamma_{con}$ $\in$ ${\cal D}_{\cal L}$\}

      ${\cal F}^{-}_{con}$ =
      \{$\rho$ $\mid$ $\rho$ $\in$ ${\cal F}^{-}_{con}$($\gamma_{con}$), $\gamma_{con}$ $\in$ ${\cal D}_{\cal L}$\}

      ${\cal F}^{+}_{upd}$ =
      \{$\rho$ $\mid$ $\rho$ $\in$ ${\cal F}^{+}_{upd}$($\gamma_{upd}$), $\gamma_{upd}$ $\in$ ${\cal D}_{\cal L}$\}

      ${\cal F}^{-}_{upd}$ =
      \{$\rho$ $\mid$ $\rho$ $\in$ ${\cal F}^{-}_{upd}$($\gamma_{upd}$), $\gamma_{upd}$ $\in$ ${\cal D}_{\cal L}$\}
    \end{vquote}

    Additionally, we use the complementary set notation
    $\overline{{\cal F}}$ to denote a set containing the negation of
    facts in set ${\cal F}$.

    \begin{vquote}
      $\overline{{\cal F}}$ =
      \{$\lnot\rho$ $\mid$ $\rho$ $\in$ ${\cal F}$\}.
    \end{vquote}

    Let $\gamma$ be an initial, constraint or policy update definition
    statement of language ${\cal L}$. We then define the following functions:

    \begin{vquote}
      $Eff$($\gamma$)

      \hspace{1em}
      =
      \begin{math}
        \begin{cases}
          \mbox{\{$a_{0}$, \ldots, $a_{m}$, $\lnot$$b_{0}$, \ldots, $\lnot$$b_{n}$\}, if $\gamma$ is initially} \\
          \mbox{\{$c_{0}$, \ldots, $c_{o}$, $\lnot$$d_{0}$, \ldots, $\lnot$$d_{p}$\}, if $\gamma$ is constraint} \\
          \mbox{\{$i_{0}$, \ldots, $i_{u}$, $\lnot$$j_{0}$, \ldots, $\lnot$$j_{v}$\}, if $\gamma$ is update}
        \end{cases}
      \end{math}
    \end{vquote}

    \begin{vquote}
      $Def$($\gamma$)

      \hspace{1em}
      =
      \begin{math}
        \begin{cases}
          \mbox{$\emptyset$, if $\gamma$ is initially} \\
          \mbox{\{$g_{0}$, \ldots, $g_{s}$, $\lnot$$h_{0}$, \ldots, $\lnot$$h_{t}$\}, if $\gamma$ is constraint} \\
          \mbox{$\emptyset$, if $\gamma$ is update}
        \end{cases}
      \end{math}
    \end{vquote}

    \begin{vquote}
      $Pre$($\gamma$)

      \hspace{1em}
      =
      \begin{math}
        \begin{cases}
          \mbox{$\emptyset$, if $\gamma$ is initially} \\
          \mbox{\{$e_{0}$, \ldots, $e_{q}$, $\lnot$$f_{0}$, \ldots, $\lnot$$f_{r}$\}, if $\gamma$ is constraint} \\
          \mbox{\{$k_{0}$, \ldots, $k_{w}$, $\lnot$$l_{0}$, \ldots, $\lnot$$l_{x}$\}, if $\gamma$ is update}
        \end{cases}
      \end{math}
    \end{vquote}

    \begin{vdefinition}
      \label{def-mutex}
      Given a domain description ${\cal D_{L}}$ of language ${\cal L}$,
      two ground facts $\rho$ and $\rho'$ are {\em mutually exclusive}
      in ${\cal D_{L}}$ if:
      \begin{vquote}
        $\rho$ $\in$ \{${\cal F}^{+}_{int}$ $\cup$
        $\overline{{\cal F}^{-}_{int}}$ $\cup$ ${\cal F}^{+}_{con}$ $\cup$
        $\overline{{\cal F}^{-}_{con}}$ $\cup$ ${\cal F}^{+}_{upd}$ $\cup$
        $\overline{{\cal F}^{-}_{upd}}$\}

        implies

        $\rho'$ $\not\in$ \{${\cal F}^{+}_{int}$ $\cup$
        $\overline{{\cal F}^{-}_{int}}$ $\cup$ ${\cal F}^{+}_{con}$ $\cup$
        $\overline{{\cal F}^{-}_{con}}$ $\cup$ ${\cal F}^{+}_{upd}$ $\cup$
        $\overline{{\cal F}^{-}_{upd}}$\}
      \end{vquote}
    \end{vdefinition}

    Simply stated, a pair of mutually exclusive facts cannot both be true
    in any given state. The following two definitions refer to language
    ${\cal L}$ statements.

    \begin{vdefinition}
      \label{def-comp}
      Given a domain description ${\cal D_{L}}$ of language ${\cal L}$,
      two statements $\gamma$ and $\gamma'$ are {\em complementary} in
      ${\cal D_{L}}$ if one of the following conditions holds:
      \begin{enumerate}
        \item
          $\gamma$ and $\gamma'$ are both constraint statements and
          $Eff(\gamma)$ = $\overline{Eff(\gamma')}$.
        \item
          $\gamma$ is a constraint statement, $\gamma'$ is an update
          statement and $Eff(\gamma)$ = $\overline{Eff(\gamma')}$.
      \end{enumerate}
    \end{vdefinition}

    \begin{vdefinition}
      \label{def-norm}
      Given a domain description ${\cal D_{L}}$, ${\cal D_{L}}$ is said to
      be {\em normal} if it satisfies all of the following conditions:
      \begin{enumerate}
        \item
          \label{def-norm-1}
          ${\cal F}^{+}_{int}$ $\cap$ ${\cal F}^{-}_{int}$ = $\emptyset$.
        \item
          \label{def-norm-2}
          For any two constraint statements $\gamma$ and $\gamma'$ in
          ${\cal D_{L}}$, including $\gamma$ = $\gamma'$, $Def(\gamma)$ $\cap$
          $Eff(\gamma')$ = $\emptyset$.
        \item
          \label{def-norm-3}
          For all constraint statements $\gamma$ in ${\cal D_{L}}$,
          $\overline{Eff(\gamma)}$ $\cap$ $Pre(\gamma)$ = $\emptyset$.
        \item
          \label{def-norm-4}
          For any two {\em complementary} statements $\gamma$ and $\gamma'$ in
          ${\cal D_{L}}$, there exists a pair of ground expression $\epsilon$
          $\in$ $Pre(\gamma)$ and $\epsilon'$ $\in$ $Pre(\gamma')$ such that
          $\epsilon$ and $\epsilon'$ are {\em mutually exclusive}.
      \end{enumerate}
    \end{vdefinition}

    With the above definitions, we can now provide a sufficient condition to
    ensure the consistency of a domain description.

    \begin{vtheorem}{Domain Consistency}
      \label{the-cons}
      A {\em normal} domain description of language ${\cal L}$ is also
      {\em consistent}.
    \end{vtheorem}

    \begin{vproof}
      From Definition \ref{def-trans}, given a normal domain description
      ${\cal D}_{\cal L}$, we only need to show that
      $Trans$(${\cal D}_{\cal L}$) has at least one consistent answer set to
      prove that ${\cal D}_{\cal L}$ is also consistent.

      Given a normal domain description ${\cal D}_{\cal L}$, Condition
      \ref{def-norm-2} in Definition \ref{def-norm} ensures that the
      translation $Trans$(${\cal D}_{\cal L}$) do not contain rules of the
      following form:

      \begin{vquote}
        $\hat{\rho}_{0}$ $\leftarrow$ $\hdots$, $not$ $\hat{\rho}_{k}$, $\hdots$

        $\hat{\rho}_{1}$ $\leftarrow$ $\hdots$, $\hat{\rho}_{0}$, $\hdots$

        $\vdots$

        $\hat{\rho}_{k - 1}$ $\leftarrow$ $\hdots$, $\hat{\rho}_{k - 2}$, $\hdots$

        $\hat{\rho}_{k}$ $\leftarrow$ $\hdots$, $\hat{\rho}_{k - 1}$, $\hdots$
      \end{vquote}

      The absence of these rules means $Trans$(${\cal D}_{\cal L}$) is a
      program without negative cycles \cite{LIN}. As no other rules in
      ${\cal D}_{\cal L}$ can cause $Trans$(${\cal D}_{\cal L}$) to have these
      rules, we conclude that under Definition \ref{def-norm}, a normal domain
      description ${\cal D}_{\cal L}$ will generate an extended logic program
      $Trans$(${\cal D}_{\cal L}$) without negative cycles. Also, from
      \cite{BAR,LIN}, we further conclude that the translated program
      $Trans$(${\cal D}_{\cal L}$) must have an answer set.

      Condition \ref{def-norm-1} of Definition \ref{def-norm} prevents rules
      of the following form from occurring in $Trans$(${\cal D}_{\cal L}$):

      \begin{vquote}
        $\hat{\rho}^{S_{0}}$ $\leftarrow$

        $\lnot\hat{\rho}^{S_{0}}$ $\leftarrow$
      \end{vquote}

      \noindent
      This shows that a subset of the answer set which contains facts from the
      initial state $S_{0}$ is consistent.

      Condition \ref{def-norm-3} of Definition \ref{def-norm} guarantees that
      rules of the following form do not occur in $Trans$(${\cal D}_{\cal L}$):

      \begin{vquote}
        $\hat{\rho}$ $\leftarrow$ $\hdots$, $\lnot\hat{\rho}$, $\hdots$
      \end{vquote}

      \noindent
      This ensures that all constraint rules translated from
      ${\cal D}_{\cal L}$ are consistent.

      Finally, Condition \ref{def-norm-4} of Definition \ref{def-norm}
      ensures that rules in $Trans$(${\cal D}_{\cal L}$) of the following form:

      \begin{vquote}
        $\hat{\rho}$ $\leftarrow$ $\hdots$, $\hat{\rho}'$, $\hdots$

        $\lnot \hat{\rho}$ $\leftarrow$ $\hdots$, $\hat{\rho}''$, $\hdots$
      \end{vquote}

      \noindent
      cannot both affect the answer set as the premises $\rho'$ and $\rho''$
      are mutually exclusive and therefore only one is true in any given state.

      These guarantee that the answer set do not contain complementary facts,
      and therefore guarantee that the answer set is consistent.
    \end{vproof}

    As only consistent domain descriptions can be evaluated in terms of user
    queries, Theorem \ref{the-cons} may be used to check whether a domain
    description is consistent.

    \begin{vdefinition}
      \label{def-eval}
      Given a {\em consistent} domain description ${\cal D}_{\cal L}$, a ground
      query expression $\phi$ and a finite sequence list $\psi$, we say
      {\em query $\phi$ holds in ${\cal D}_{\cal L}$ after the policy updates
      in sequence list $\psi$ have been applied}, denoted as

      \begin{vquote}
        ${\cal D}_{\cal L}$ $\models$ \{$\phi$, $\psi$\}
      \end{vquote}

      \noindent
      if and only if

      \begin{vquote}
        $\forall$ ($\rho$, $\lambda$),
        $\hat{\rho}$ $\in$ $\lambda$

      \end{vquote}

      \begin{vquote}
        where

        $\rho$ $\in$ $\phi$,
        $\lambda$ $\in$ $\Lambda$,

        $\hat{\rho}$ = $TransFact$($\rho$, $S_{|\psi|}$),

        $\Lambda$ = answer sets of $Trans$(${\cal D}_{\cal L}$)
      \end{vquote}
    \end{vdefinition}

    Definition \ref{def-eval} shows that given a finite list of policy updates
    $\psi$, a query expression $\phi$ may be evaluated from a consistent
    language ${\cal L}$ domain ${\cal D_{L}}$. This is achieved by generating a
    set of answer sets from the normal logic program translation
    $Trans$(${\cal D_{L}}$). $\phi$ is then said to hold in ${\cal D_{L}}$
    after the policy updates in $\psi$ have been applied if and only if every
    answer set generated contains every fact in the query expression $\phi$.

    \begin{vexample}
      \label{ex-2}
      Given the language ${\cal L}$ code listing in Example \ref{ex-1} and its
      semantic translation in Appendix \ref{app-trans}, where the update
      sequence list $\psi$ = \{$delete\_read$($grp1$, $file$)\}. The following
      shows the evaluated results of each query $\phi$:

      \begin{vquote}
        $\phi_{0}$ = $holds$($grp1$, $write$, $file$) : $TRUE$ \\
        $\phi_{1}$ = $holds$($grp1$, $read$, $file$) : $FALSE$ \\
        $\phi_{2}$ = $holds$($alice$, $write$, $file$) : $TRUE$ \\
        $\phi_{3}$ = $holds$($alice$, $read$, $file$) : $FALSE$
      \end{vquote}
    \end{vexample}

  \section{Implementation}
    \label{sec-implement}

    As mentioned earlier, {\em PolicyUpdater} is a fully implemented system.
    In this section, we describe the implementation details of this system.
    Details of the application and the core system program may be found in
    the project homepage at:

    \begin{vquote}
      {\tt\scriptsize http://www.cit.uws.edu.au/\~{}jcrescin/projects/} \\
      {\tt\scriptsize PolicyUpdater/index.html}
    \end{vquote}

    \subsection{System Structure}

    \begin{figure}[ht]
      \begin{center}
        \includegraphics{figure-01}
        \caption{Structure of PolicyUpdater}
        \label{fig-1}
      \end{center}
    \end{figure}

      As shown in Figure \ref{fig-1}, the PolicyUpdater system works in
      conjunction with an authorisation agent program that queries the
      policy base to determine whether to allow users access to resources.
      Through an authorisation agent program, the PolicyUpdater system also
      allows administrators to dynamically update the policy base by adding
      or removing update directives in the policy update table.

      \subsubsection{Parsers.}

        As the policy itself is written in language ${\cal L}$, the system uses
        two parsers to act as interfaces to the authorisation agent and
        the language ${\cal L}$ policy.

        \paragraph{Policy Parser.}

          The policy parser is responsible for correctly reading the policy
          file into the core PolicyUpdater system. The parser ensures that
          the policy file strictly adheres to the language $\cal{L}$ syntax
          then systematically stores entity identifiers into the symbol table
          and initial state facts, constraint expressions and policy update
          definitions are stored into their respective tables in the policy
          base.

        \paragraph{Agent Parser.}

          The agent parser is the direct link between the core PolicyUpdater
          system and the authorisation agent program. The parser's sole purpose
          is to receive language $\cal{L}$ directives from an agent, perform
          the directive upon the policy base and return a reply if the
          directive requires one. Such directives may be to query the policy
          base or to manipulate the policy update sequence table.

      \subsubsection{Data Structures.}

        As language ${\cal L}$ program is parsed, each statement containing
        entity declarations, initial facts, constraint rules and policy
        updates must first be stored into a structure before the translation
        process is started. As shown in Appendix \ref{app-store}, the structure
        is composed of the symbol table, the policy base and the policy update
        sequence table.

        The symbol table is used to store all entity identifiers defined in the
        policy, while the rest of the policy definitions are stored into the
        policy base. On the other hand, the sequence of policy update
        directives are stored separately into the update table.

    \subsection{System Processes}

      The processes presented in this section shows how the language
      ${\cal L}$ policy stored in the data structures is translated into a
      normal logic program and how it can be dynamically updated and
      manipulated to evaluate queries. The flowchart in Figure \ref{fig-2}
      gives an overview of the system processes.

      \begin{figure}[ht]
        \begin{center}
          \includegraphics{figure-02}
          \caption{System Flowchart}
          \label{fig-2}
        \end{center}
      \end{figure}

      \subsubsection{Grounding Constraint Variables.}

        As the constraints are in the process of being added into the
        constraints table, each variable identifier that occurs within
        every constraint is grounded by replacing that constraint with a set of
        constraints wherein each instance of the variable is replaced by all
        entity identifiers defined in the symbol table. Note that only those
        entity identifiers that are valid for each fact in the current
        constraint are used to replace the variable (e.g. only singular
        subject entity identifiers are used to replace an element variable
        occurring in a subject member fact).

        For example, given that the symbol table contains three singular
        subject entity identifiers: $alice$, $bob$ and $charlie$, and the
        following constraint:

        \begin{vverbatim}
  always holds(SSUB, write, file)
    implied by
      holds(SSUB, read, file) &&
      memb(SSUB, students)
    with absence
      !holds(SSUB, write, file);
        \end{vverbatim}

        Grounding the constraint statement above will yield three new
        constraint rules, each replacing occurrences of the variable $SSUB$
        with $alice$, $bob$ and $charlie$, respectively.

      \subsubsection{Policy Updates.}

        In Section \ref{subsec-semantics}, it is shown that policy updates are
        performed by treating each update as a constraint. This constraint is
        composed of a premise, which is the precondition in the current state
        and a consequent, which is the postcondition of the resulting state
        after the application of the policy update. The resulting state in this
        procedure represents the updated policy.

        The most crucial step in performing a policy update is the translation
        of the policy updates into normal logic program constraints. This step
        involves identifying which policy updates are to be applied from the
        update sequence table and then composing the required constraint from
        the update definition in the policy base. Once the policy update
        constraints are composed, they are then treated as any other
        constraint rules and are translated with the rest of the policy into
        a normal logic program.

      \subsubsection{Translation to Normal Logic Program.}

        The semantics of language ${\cal L}$ shows that any consistent language
        ${\cal L}$ program can be translated into an equivalent extended logic
        program then translated again into an equivalent normal logic program.
        However, the implementation of such translations can be greatly
        simplified by translating language ${\cal L}$ programs directly into
        normal logic programs.

        \paragraph{Removing Classical Negation.}

          In order to remove classical negation from facts of language
          ${\cal L}$, each classically negated fact $\lnot$$\rho$ is replaced
          by a new and unique positive fact $\rho'$ that represents the
          negation of fact $\rho$. To preserve the consistency of the policy
          base for all facts $\rho$ in the domain, the following constraint
          rule must be added:

          \begin{vquote}
            $FALSE$ $\leftarrow$ $\rho$, $\rho'$
          \end{vquote}

          The removal process involves adding a boolean parameter to each fact
          to indicate whether the fact is classically negated or not. For
          example, given the fact:

          \begin{vquote}
            $\lnot$ $holds$($alice$, $exec$, $file$)
          \end{vquote}

          \noindent
          To remove classical negation, it is replaced by:

          \begin{vquote}
            $holds$($alice$, $exec$, $file$, $false$)
          \end{vquote}

          \noindent
          For consistency, the following constraint is added:

          \begin{vquote}
            $FALSE$ $\leftarrow$

            \hspace{1em}
            $holds$($alice$, $exec$, $file$, $true$),

            \hspace{1em}
            $holds$($alice$, $exec$, $file$, $false$)
          \end{vquote}

        \paragraph{Representing Facts in Propositional Form.}

          A fact expressed in normal logic program form is composed of the
          atom relation, the state in which it holds and a boolean flag to
          indicate classical negation. For notational simplicity, this tuple
          may be represented by a unique positive integer $i$, where
          $0$ $\leq$ $i$ $<$ $|{\cal F}|$ ($|{\cal F}|$ is the total
          number of facts in the domain). The process of translating facts of
          language ${\cal L}$ into normal logic program form is summarised by
          the following function:

          \begin{vquote}
            $i$ = $Encode$($\alpha$, $\sigma$, $\tau$)
          \end{vquote}

          As shown above, the $Encode$ function takes a language ${\cal L}$
          atom $\alpha$, the state $\sigma$ in which $\alpha$ holds, and a
          truth value $\tau$ that indicates whether $\alpha$ is classically
          negated or not. $Encode$ returns a unique index $i$ for that
          fact. The steps below outlines how the $Encode$ function computes
          the index $i$.

          \begin{itemize}

            \item
              {\em Enumerate all possible atoms.}
              By using all the entities in the symbol table, all possible
              language ${\cal L}$ atoms may be enumerated by grouping together
              2 to 3 entities together. All possible atoms of type $holds$ are
              generated by enumerating all possible combinations of subject,
              access right and object entities. The set of $member$ atoms is
              generated from all the different combinations of singular and
              group entities of types subject, access right and object.
              Similarly, the set of $subset$ atoms is derived from different
              subject, access right and object group pair combinations.

            \item
              {\em Arrange the atoms in a predefined order.}
              This procedure relies on the assumption that the list of all
              possible atoms derived from the step above is arranged in a
              predefined order. In this step we ensure that the atoms are
              enumerated in the following order: $holds$, $subject$ $member$,
              $access$ $right$ $member$, $object$ $member$, $subject$
              $subset$, $access$ $right$ $subset$ and $object$ $subset$. In
              addition to the ordering of atom types, atoms of each type are
              themselves sorted according to the order in which their entities
              appear in the symbol table.

            \item
              {\em Assign an ordinal index for each enumerated atom.}
              Since the enumerated list of atoms are ordered, consecutive
              positive integers may be assigned to each atom as an ordinal
              index $i$, where $0$ $\leq$ $i$ $<$ $n$ ($n$ is the total number
              of atoms enumerated).

            \item
              {\em Extend indexing procedure to represent facts.} At the
              implementation level, facts are just atoms with truth values.
              As such, we can treat each atom as positive facts. Since negative
              facts are just mirror images of their positive counterparts,
              their indices are calculated by adding $n$ to the indices of the
              corresponding positive facts. Thus, indices $i$, where $n$ $\leq$
              $i$ $<$ $2n$ are negative facts while indices $i$, where $0$
              $\leq$ $i$ $<$ $n$ are positive facts. Furthermore, this
              procedure is again extended to represent the states of the
              facts. The process is similar: indices $i$, where $0$ $\leq$ $i$
              $<$ $2n$ represent facts of state $S_{0}$, indices $i$, where
              $2n$ $\leq$ $i$ $<$ $4n$ represent facts of state $S_{1}$, and
              so on.
          \end{itemize}

        \paragraph{Generating the Normal Logic Program from the Policy Base.}

          With the language ${\cal L}$ policy elements stored into the storage
          structures (see Appendix \ref{app-store}), a normal logic program can
          then be generated for evaluation. The following algorithm generates
          a normal logic program, given the Symbol Table $Ts$, Initial
          State Facts Table $Ti$, Constraint Rules Table $Tc$, Policy
          Update Definition Table $Tu$, and Policy Update Sequence Table
          $Tq$:

          \begin{vverbatim}
FUNCTION GenNLP(Ts, Ti, Tc, Tu, Tq)
  TransInitStateRules(Ti)
  TransConstRules(Tc, Tq)
  TransUpdateRules(Tu, Tq)
  GenInherRules(Ts, Tq)
  GenTransRules(Ts, Tq)
  GenInertRules(Ts, Tq)
  GenIdentRules(Ts, Tq)
  GenConsiRules(Ts, Tq)
ENDFUNCTION
          \end{vverbatim}

          The first three $Trans*()$ functions above perform a direct
          translation of language ${\cal L}$ statements to normal logic
          program. The remaining five $Gen*()$ functions generate additional
          constraint rules. In the following algorithms, we use the following
          rule constructor functions to generate normal logic program rules:

          \begin{itemize}
            \item
              $RuleBegin()$ marks the beginning of a new rule.
            \item
              $RuleHead(\alpha, \tau)$ generates the consequent of the rule.
              $\alpha$ is a numeric representation of an atom (e.g. returned by
              the $Encode()$ function) and $\tau$ is either $T$ or $F$,
              indicating whether the atom is positive or negative
              (negation-as-failure).
            \item
              $RuleBody(\alpha, \tau)$ generates the premise of the rule. The
              parameters of this function is the same as that of the
              function $RuleHead()$.
            \item
              $RuleEnd()$ marks the end of a rule.
          \end{itemize}

The algorithm below illustrates how initial state rules are generated from the
storage structures. The process itself is straightforward: each fact in the
initial state facts table is translated by the $Encode()$ function and is made
the head of a new rule whose body is the literal $True$ fact.

          \begin{vverbatim}
FUNCTION TransInitStateRules(Ti)
  FOR i = 0 TO Len(Ti) DO
    a = Encode(Ti[i].atm, 0, Ti[i].tr)
    RuleBegin()
    RuleHead(a, T)
    RuleBody(T, T)
    RuleEnd()
  ENDDO
ENDFUNCTION
          \end{vverbatim}

The constraint rules generating algorithm below works by creating a new rule
that is composed of facts from the constraints table translated by the
$Encode()$ function. The outer loop ensures that a rule is generated for every
policy update state.

          \begin{vverbatim}
FUNCTION TransConstRules(Tc, Tq)
  FOR i = 0 TO Len(Tq) DO
    FOR j = 0 TO Len(Tc) DO
      RuleBegin()
      FOR k = 0 TO Len(Tc[j].exp) DO
        a = Encode(Tc[j].exp[k].atm,
                   i,
                   Tc[j].exp[k].tr)
        RuleHead(a)
      ENDDO
      FOR k = 0 TO Len(Tc[j].pcond) DO
        a = Encode(Tc[j].pcond[k].atm,
                   i,
                   Tc[j].pcond[k].tr)
        RuleHead(a, T)
      ENDDO
      FOR k = 0 TO Len(Tc[j].ncond) DO
        a = Encode(Tc[j].ncond[k].atm,
                   i,
                   Tc[j].ncond[k].tr)
        RuleHead(a, F)
      ENDDO
      RuleEnd()
    ENDDO
  ENDDO
ENDFUNCTION
          \end{vverbatim}

          The algorithm below generates the policy update rules from the given
          policy update definition table. Note that only those policy updates
          that also appear in the policy update sequence list are actually
          translated. The actual translation process is similar to that of
          constraint rules, except each variable that may occur within the
          expressions is first grounded and the policy update state of each
          fact in the rule head is one more than that of each fact in the rule
          body.

          \begin{vverbatim}
FUNCTION TransUpdateRules(Tu, Tq)
  FOR i = 0 TO Len(Tq) DO
    FOR j = 0 TO Len(Tu) DO
      IF Tq[i].name == Tu[j].name THEN
        e =
          GndUpdate(Tu[j], Tq[i].ilist)
        RuleBegin()
        FOR k = 0 TO Len(e.post) DO
          a = Encode(e.post[k].atm,
                     i + 1,
                     e.post[k].tr)
          RuleHead(a, T)
        ENDDO
        FOR k = 0 TO Len(e.pre) DO
          a = Encode(e.pre[k].atm,
                     i,
                     e.pre[k].tr)
          RuleBody(a, T)
        ENDDO
        RuleEnd()
      ENDIF
    ENDDO
  ENDDO
ENDFUNCTION
          \end{vverbatim}

          The function $GndUpdate(U, IL)$ used in the algorithm above returns
          a structure composed of two expressions $pre$ and $post$, which
          corresponds with the $pre$ and $post$ fields of the given policy
          update definition $U$. All variables occurring in the facts of these
          expressions are replaced with the corresponding entities from the
          given entity identifier list $IL$.

          The function below generates 6 types of inheritance rules: subset
          subject, subset access right, subset object, membership subject,
          membership access right and membership object. Each of these 6
          algorithms work in a similar way: a rule is generated by composing
          every possible combination of either subject, access right and object
          entities to form either a subset or membership fact. As with the
          contraint rule generating algorithm, each new rule generated is
          replicated for each policy update state.

          \begin{vverbatim}
FUNCTION GenInherRules(Ts, Tq)
  GenSubSubstInherRules(Ts, Tq)
  GenAccSubstInherRules(Ts, Tq)
  GenObjSubstInherRules(Ts, Tq)
  GenSubMembInherRules(Ts, Tq)
  GenAccMembInherRules(Ts, Tq)
  GenObjMembInherRules(Ts, Tq)
ENDFUNCTION
          \end{vverbatim}

          The function below generates all the transitivity rules. Each
          subject, access right and object transitivity rule generation
          algorithm follows a similar procedure: every possible combination of
          subject, access right or object group entities are used to form
          subset facts, then each of these facts are used to form a
          transitivity rule. As with inheritance rules, each transitivity rule
          is replicated for each policy update state.

          \begin{vverbatim}
FUNCTION GenTransRules(Ts, Tq)
  GenSubTransRules(Ts, Tq)
  GenAccTransRules(Ts, Tq)
  GenObjTransRules(Ts, Tq)
ENDFUNCTION
          \end{vverbatim}

          The inertial rules generation function below is composed of 3
          functions that generate inertial rules for each atom type: holds,
          membership and subset. Each type of rule is generated by composing
          different combinations of entity identifiers together to form a fact.
          Each rule is then formed by stating that for each policy update
          state, a fact holds in the current state if it also holds in the
          previous state and its negation does not hold in the current state.

          \begin{vverbatim}
FUNCTION GenInertRules(Ts, Tq)
  GenHldsInertRules(Ts, Tq)
  GenMembInertRules(Ts, Tq)
  GenSubsInertRules(Ts, Tq)
ENDFUNCTION
          \end{vverbatim}

          The function $GenIdentRules()$ shown below generates the identity
          rules for each atom type: subject, access right and object. A simple
          procedure is followed by each of the 3 functions: for every subject,
          access right and object group entities, a subset rule is formed to
          show that a group is a subset of itself. As with the other rules,
          each rule generated by these functions is replicated for each policy
          update state.

          \begin{vverbatim}
FUNCTION GenIdentRules(Ts, Tq)
  GenSubIdentRules(Ts, Tq)
  GenAccIdentRules(Ts, Tq)
  GenObjIdentRules(Ts, Tq)
ENDFUNCTION
          \end{vverbatim}

          The last two functions below shows the algorithm to generate
          consistency rules for each atom type: holds, membership and subset.
          As these rules use a similar process to generate rules, only the
          holds consistency rule generation algorithm is shown. The rules that
          are generated ensure that only a fact or its negation, but never
          both, holds in the same policy update state.

          \begin{vverbatim}
FUNCTION GenConsiRules(Ts, Tq)
  GenHldsConsiRules(Ts, Tq)
  GenMembConsiRules(Ts, Tq)
  GenSubsConsiRules(Ts, Tq)
ENDFUNCTION
          \end{vverbatim}

          \begin{vverbatim}
FUNCTION GenHldsConsiRules(Ts, Tq)
  FOR i = 0 TO Len(Tq) DO
    FOR j = 0 TO Len(Ts.s) DO
      FOR k = 0 TO Len(Ts.a) DO
        FOR l = 0 TO Len(Ts.o) DO
          ahlds.sub = Ts.s[j]
          ahlds.acc = Ts.a[k]
          ahlds.obj = Ts.o[l]
          RuleBegin()
          RuleHead(F, T)
          a = Encode(ahlds, i, T)
          RuleBody(a, T)
          a = Encode(ahlds, i, F)
          RuleBody(a, T)
          RuleEnd()
        ENDDO
      ENDDO
    ENDDO
  ENDDO
ENDFUNCTION
          \end{vverbatim}

      \subsubsection{Query Evaluation.}

        Once a normal logic program has been generated from the policy stored
        in the storage structure, a set of answer sets may then be generated
        by using the stable model semantics \cite{SIM} with the
        {\em smodels}\footnotemark program. Query evaluation then becomes
        possible by checking whether each fact of a given query expression
        holds in each generated answer set of the normal logic program.

        \footnotetext{
          Smodels ({\tt \scriptsize http://www.tcs.hut.fi/Software/smodels})
        }

        If a given fact indeed holds in all the answer sets, it is then
        evaluated to be true. On the other hand, if the negation of a fact
        holds in every answer set, then it is evaluated to be false. A fact
        or its negation that does not hold in every answer set is neither true
        nor false, in which case the system concludes that the truth value of
        the fact is unknown. The algorithm below shows how, given a state $S$,
        a query expression $Qe$ can be evaluated against a list of stable
        models $SM$, where each element in $SM$ is a list of facts.

        \begin{vverbatim}
FUNCTION EvaluateExp(Qe, SM, S)
  result = T
  FOR i = 0 TO Len(Qe) DO
    rv = EvaluateFact(Qe[i], SM, S)
    IF rv == F THEN
      RETURN F
    ELSE IF rv == U THEN
      result = U
    ENDIF
  ENDDO
  RETURN result
ENDFUNCTION
        \end{vverbatim}

        The algorithm above attempts to evaluate each fact in the query
        expression. The function $EvaluateFact()$ shown below evaluates a
        single fact $Qf$ in state $S$, against a list of stable models $SM$.

        \begin{vverbatim}
FUNCTION EvaluateFact(Qf, SM, S)
  a = Encode(Qf.atm, S, Qf.tr)
  IF IsFactIn(SM, a) THEN
    RETURN T
  ELSE
    a = Encode(Qf.atm, S, NOT Qf.tr)
    IF IsFactIn(a, SM) THEN
      RETURN F
    ELSE
      RETURN U
    ENDIF
  ENDIF
ENDFUNCTION
        \end{vverbatim}

        The function $IsFactIn()$ below simply returns a boolean value to
        indicate whether the given fact index $Fi$ (as returned by the
        $Encode()$ function) is present in every stable model in $SM$.

        \begin{vverbatim}
FUNCTION IsFactIn(Fi, SM)
  FOR i = 0 TO Len(SM) DO
    IF NOT IsIn(SM[i], Fi) THEN
      RETURN F
    ENDIF
  ENDDO
  RETURN T
ENDFUNCTION
        \end{vverbatim}

    \subsection{Experimental Results}

      In this subsection, we investigate the effects of domain size over
      computation time. The following tests were conducted with the latest
      version of PolicyUpdater\footnotemark running on an AMD Athlon
      XP 2000+ machine with 512 MB of RAM, running the Debian GNU/Linux 3.0r5
      operating system with a plain Linux 2.4.30 kernel.

      \footnotetext{\
        At the time of writing, the latest version of PolicyUpdater is vlad\
        1.0.4.\
      }

      Table \ref{tab-1} shows the domain size for each test case. $S_{E_{s}}$
      and $S_{E_{g}}$ are the numbers of singular and group entities,
      respectively; $S_{I}$ is the number of initial state facts; $S_{C}$ is
      the number of constraint rules; $S_{U}$ is the number of policy update
      definitions; $S_{S}$ is the number of policy updates in the sequence
      list; and $S_{Q}$ is the number of facts to be queried.

      \begin{table}[ht]
        \begin{center}
          \begin{tabular}[t]{|r|r|r|r|r|r|r|r|}
            \hline
            &
            \textbf{$S_{E_{s}}$} &
            \textbf{$S_{E_{g}}$} &
            \textbf{$S_{I}$} &
            \textbf{$S_{C}$} &
            \textbf{$S_{U}$} &
            \textbf{$S_{S}$} &
            \textbf{$S_{Q}$} \\
            \hline
            1 & 4 & 3 & 3 & 1 & 1 & 1 & 4 \\
            \hline
            2 & 24 & 23 & 3 & 1 & 1 & 1 & 4 \\
            \hline
            3 & 104 & 3 & 3 & 1 & 1 & 1 & 4 \\
            \hline
            4 & 4 & 103 & 3 & 1 & 1 & 1 & 4 \\
            \hline
            5 & 24 & 23 & 103 & 1 & 1 & 1 & 4 \\
            \hline
            6 & 24 & 23 & 3 & 101 & 1 & 1 & 4 \\
            \hline
            7 & 24 & 23 & 3 & 1 & 101 & 1 & 4 \\
            \hline
            8 & 24 & 23 & 3 & 1 & 101 & 101 & 4 \\
            \hline
            9 & 24 & 23 & 3 & 1 & 1 & 1 & 104 \\
            \hline
            10 & 24 & 23 & 103 & 1 & 101 & 101 & 4 \\
            \hline
            11 & 24 & 23 & 3 & 101 & 101 & 101 & 4 \\
            \hline
            12 & 24 & 23 & 103 & 101 & 101 & 101 & 104 \\
            \hline
            13 & 104 & 103 & 103 & 101 & 101 & 101 & 104 \\
            \hline
          \end{tabular}
        \end{center}
        \caption[]{Thirteen test cases with different domain sizes}
        \label{tab-1}
      \end{table}

      The language {\cal L} code listing in Example \ref{ex-1} is used in the
      first test case. In the second test case, the same code is used with 20
      new singular entities and 20 new group entities. Test cases 3 and 4 are
      similar to test case 1, except 100 new singular and group entities were
      added, respectively. Test cases 5 and 6 are similar to test case 2,
      except 100 new initial state facts and constraint rules were added,
      respectively. In test case 7, 100 new policy update definitions were
      added, and in test case 8, these policy update defintions were applied.
      Test case 9 is similar to test case 2, but this one tries to evaluate 100
      additional query facts. Test case 11 is a combination of test cases 6 and
      8. Test case 12 is a combination of test cases 5, 9 and 11. Finally, test
      case 13 is a combination of test cases 3 to 9, where the number of each
      domain component is over 100.

      Table \ref{tab-2} shows the execution times of each test case. $T_{C}$ is
      the total time (in seconds) spent by the system to translate the language
      {\cal L} statements to a normal logic program and to generate the answer
      sets. $T_{Q}$ is the total time (in seconds) used by the system to
      evaluate all the queries. To increase result accuracy, each test was
      conducted 10 times. The figures in Table \ref{tab-2} are the averages.

      \begin{table}[ht]
        \begin{center}
          \begin{tabular}[t]{|r|r|r|}
            \hline
            &
            \textbf{$T_{C}$} &
            \textbf{$T_{Q}$} \\
            \hline
            1 & 0.000794 & 0.000472 \\
            \hline
            2 & 0.261828  & 0.600932 \\
            \hline
            3 & 0.072069 & 0.157254 \\
            \hline
            4 & 14.017335 & 32.109291 \\
            \hline
            5 & 0.309517 & 0.698068 \\
            \hline
            6 & 0.306517 & 0.694729 \\
            \hline
            7 & 0.304570 & 0.696636 \\
            \hline
            8 & 15.315347 & 32.111353 \\
            \hline
            9 & 0.301429 & 25.147113 \\
            \hline
            10 & 15.375953 & 32.537575 \\
            \hline
            11 & 15.889154 & 33.246048 \\
            \hline
            12 &  &  \\
            \hline
            13 & ? & ? \\
            \hline
          \end{tabular}
        \end{center}
        \caption[]{Average computation times in seconds for each test case}
        \label{tab-2}
      \end{table}

      As shown in Table \ref{tab-2}, the first two execution times are minimal
      when the domain size is small. Test 3 shows that having a large number of
      singular entities have a measurable, but insignificant effect on
      computation time. However, test 4 shows that an increase in the number of
      group entities have a great impact on computation speed. This is to be
      expected, as Section \ref{subsec-semantics} shows that the number of group
      entities directly affect the number of transitivity, inheritance and
      identity rules generated in the translation.

      Comparing test 2 with tests 5 and 6, where the number of initial state
      facts and constraint rules are increased by 100, respectively, we observe
      that that there is a slight increase in the times required to perform the
      computation and query evaluation. One would expect that an increase in
      the number of constraint rules will have more impact in execution times
      than an increase in initial state facts. However, in test 6, the
      computation times were low because only one policy update was actually
      applied.

      Test 7 shows that increasing the number of policy update definitions
      has little impact on the computation times. However, as test 8 shows,
      if these policy updates are actually applied to the policy base,
      computation time increases dramatically.

      Test case 9 shows that evaluating 100 additional queries has little
      effect on translation and computation time, but obviously affects
      evaluation time.

      Test case 10 shows the combined effects of an increased number of policy
      updates and initial state facts. As expected, the times are only slightly
      larger than the times in test case 8, where only the number of policy
      updates were increased. This is due to the fact that initial state facts
      are translated directly into normal logic program rules. On the other
      hand, test case 11 shows a significant increase in both computation and
      evaluation times. This is expected, as the translation of a single
      constraint rule results in a constraint rule in every policy update
      state.

      Test case 12 shows that although large numbers of initial state facts and
      query requests by themselves have little effect on performance, if
      combined together with the effects of a large number of policy updates,
      computation time is significantly increased, paticularly the query
      evaluation time.

      Unfortunately, the test system used in this experiment ran out of memory
      while performing test case 13. Again, this is expected, as the combined
      effects of having a large number of entities, constraint rules, policy
      updates and queries will result in approximately 5.7 billion rules, using
      the formula given in Theorem \ref{the-size}.

  \section{Case Study: Web Server Application}
    \label{sec-case}

    \begin{figure}[ht]
      \begin{center}
        \includegraphics{figure-03}
        \caption{PolicyUpdater module for Apache}
        \label{fig-3}
      \end{center}
    \end{figure}

    The expressiveness of language $\cal{L}$ and the effectiveness of the
    PolicyUpdater system can be demonstrated by a web server authorisation
    application. In this application, the core PolicyUpdater system serves as
    an authorisation module for the {\em Apache}\footnotemark web server.

    \footnotetext{Apache Web Server ({\tt \scriptsize http://www.apache.org})}

    The Apache web server provides a generic access control system as provided
    by its {\em mod\_auth} and {\em mod\_access} modules \cite{AP,LAU}. With
    this built-in access control system, Apache provides the standard HTTP
    {\em Basic} and {\em Digest} authentication schemes \cite{HTTP2}, as well
    as an authorisation system to enforce access control policies. Although the
    PolicyUpdater module do not provide the full functionality of Apache's
    built-in authorisation module {\em mod\_auth}, it does provide a flexible
    logic-based authorisation mechanism.

    As shown in Figure \ref{fig-3}, Apache's Access Control module, together
    with its policy base, is replaced by the PolicyUpdater module and its own
    policy base. The sole purpose of the PolicyUpdater module is to act as an
    interface between the web server and the core PolicyUpdater system. The
    system works as follows: as the server is started, the PolicyUpdater
    module initialises the core PolicyUpdater system by sending the policy
    base. When a client makes an arbitrary HTTP request for a resource from
    the server (1), the client (user) is authenticated against the password
    table by the built-in authentication module; once the client is properly
    authenticated (2) the request is transferred to the PolicyUpdater module,
    which in turn generates a language ${\cal L}$ query (3) from the request
    details, then sends the query to the core PolicyUpdater system for
    evaluation; if the query is successful and access control is granted,
    the original request is sent to the other request handlers of the web
    server (4) where the request is eventually honoured; then finally (5),
    the resource (or acknowledgement for HTTP requests other than GET) is sent
    back to the client. Optionally, client can be an administrator who,
    after being authenticated, is presented with a special administrator
    interface by the module to allow the policy base to be updated.

    \subsection{Policy Description in Language ${\cal L'}$}

      The policy description in the policy base is written in language
      ${\cal L'}$, which is syntactically and semantically similar to
      language ${\cal L}$ except for the lack of entity identifier
      definitions. Entity identifiers need not be explicitly defined in
      the policy definition:

      \begin{itemize}
        \item
          {\em Subjects} of the access control policies are the users. Since
          all users must first be authenticated, the password table used
          in authentication may also be used to extract the list of subjects.
        \item
          {\em Access Rights} are built in: they are the HTTP request methods
          as defined by the HTTP 1.1 standard \cite{HTTP1} (i.e. OPTIONS, GET,
          HEAD, POST, PUT, DELETE, TRACE and CONNECT).
        \item
          {\em Objects} are the resources available in the server themselves.
          Assuming that the document root is a hierarchy of directories and
          files, each of these are mapped as a unique object of language
          ${\cal L'}$.
      \end{itemize}

      Like language ${\cal L}$, language ${\cal L'}$ allows the definition of
      initial state facts, constraint rules and policy update definitions.

    \subsection{Mapping the Policies to Language ${\cal L}$}

      As mentioned above, one task of the PolicyUpdater module is to generate
      a language ${\cal L}$ policy from the given language ${\cal L'}$ to be
      evaluated by the core PolicyUpdater system. This process is outlined
      below:

      \begin{itemize}
        \item
          {\em Generating entity identifier definitions.} Subject entities are
          taken from the authentication (password) table; access rights are
          hard-coded built-ins; and the list of objects are generated by
          traversing the document root for files and directories.
        \item
          {\em Generating additional constraints.} Additional constraints are
          generated to preserve the relationship between groups and elements.
          This is useful to model the assertion that unless explicitly stated,
          users holding particular access rights to a directory automatically
          hold those access rights to every file in that directory
          (recursively, if with subdirectories). The module makes this
          assertion by generating non-conditional constraint rules that state
          that each file (object) is a member of the directory (object group)
          in which it is contained.
      \end{itemize}

      All other language ${\cal L'}$ statements (initial state definitions,
      constraint definitions and policy update definitions) are already in
      language ${\cal L}$ form.

    \subsection{Evaluation of HTTP Requests}

      A HTTP request may be represented as a simplified tuple:

      \begin{quote}
        $<$$usr$, $req\_meth$, $req\_res$$>$
      \end{quote}

      $usr$ is the authenticated username that made the request (subject);
      $req\_meth$ is a standard HTTP request method (access right); and
      $req\_res$ is the resource associated with the request (object).
      Intuitively, such a tuple may be expressed as a language ${\cal L}$ atom:

      \begin{vverbatim}
  holds(usr, req\_meth, req\_res)
      \end{vverbatim}

      With each request expressed as language ${\cal L}$ atoms, a language
      ${\cal L}$ query statement can be composed to check if the request is
      to be honoured:

      \begin{vverbatim}
  query holds(usr, req\_meth, req\_res);
      \end{vverbatim}

      Once the query statement is composed, it is then sent by the
      PolicyUpdater module to the core PolicyUpdater system for evaluation
      against the policy base.

    \subsection{Policy Updates by Administrators}

      After being properly authenticated, an administrator can perform policy
      updates through the use of a special interface generated by the
      PolicyUpdater module. This interface lists all the predefined policy
      updates that are allowed, as defined in the policy description in
      language ${\cal L'}$, as well as all the policy updates that have been
      previously applied and are in effect. As with the core PolicyUpdater
      system, administrators are allowed only the following operations:

      \begin{itemize}
        \item
          Apply a policy update or a sequence of policy updates to the policy
          base. Note that like language ${\cal L}$, in language ${\cal L'}$
          policy updates are predefined within the policy base themselves.
        \item
          Revert to a previous state of the policy base by removing a
          previously applied policy update from the policy base.
      \end{itemize}

  \section{Future Research and Extension}
    \label{sec-future}

    An obvious limitation of language ${\cal L}$, and therefore of the
    PolicyUpdater system is its inability to express time-dependent
    authorisations. Consider the following authorisation rule:

    \begin{vquote}
      $Bob$ holds $read$ access to file $f1$ between $9:00$ $AM$ and $5:00$
      $PM$.
    \end{vquote}

    The authorisation information above can be broken down into two parts: an
    authorisation part, i.e. "Bob holds read access to file $f1$", and a
    temporal part, i.e. "between 9:00 AM and 5:00 PM". As language ${\cal L}$
    can already express authorisations, we focus our attention to the temporal
    part. A naive attempt to extend language ${\cal L}$ to express time may
    involve adding two extra parameters to each authorisation atom to represent
    the starting and ending time points of the interval. For example, the
    authorisation rule above can be represented as:

    \begin{vquote}
      $holds$($bob$, $read$, $f1$, $900$, $1700$)
    \end{vquote}

    The atom above may be interpreted to mean that the authorisation holds for
    all times between 9:00 AM and 5:00 PM, inclusive. In this example, time
    granularity, or the smallest unit of time that can be expressed is one
    minute. Of course, a more general approach is to use the domain of positive
    integers. With this approach, the system can handle different granularities
    of time, where the choice of what time unit each discrete value is
    interpreted as is left to the application. For example, if the temporal
    values are defined to be the number of seconds since 12 midnight, 01 Jan
    1970 (i.e. the UNIX epoch), then the atom below states that the
    authorisation holds at an interval starting at 9:00 AM, 18 March 1976 and
    ending at 5:00 PM, 18 March 1976:

    \begin{vquote}
      $holds$($bob$, $read$, $f1$, $195951600$, $195980400$)
    \end{vquote}

    While this approach gives the language enough expressive power to represent
    authorisations bound by literal time values, it is by no means expressive
    enough to model relationships between the temporal intervals themselves.
    This deficiency is shown in the example below:

    \begin{vquote}
      $Alice$ holds a $write$ access right to file $f1$ after $Bob$ holds a
      $read$ access right to file $f2$.
    \end{vquote}

    Such authorisation rule might arise in a scenario where the access right
    $write$ to file $f1$ can only be granted in some time after the $read$
    access right to file $f2$ has been granted and revoked. This example shows
    that the specific times at which authorisations hold are not as important
    as the relationship between the times themselves. This authorisation rule
    may be represented as follows:

    \begin{vquote}
      $holds$($alice$, $write$, $f1$, $i_{1}$)

      $holds$($bob$, $read$, $f2$, $i_{2}$)

      $after$($i_{1}$, $i_{2}$)
    \end{vquote}

    The example above states that $alice$ holds a $write$ access right to file
    $f1$ at some time interval $i_{1}$, $bob$ holds a $read$ access right to
    file $f2$ at some time interval $i_{2}$, and that the interval $i_{1}$
    occurs at some time after the interval $i_{1}$. As mentioned earlier, the
    actual values of the time interval variables $i_{1}$ and $i_{2}$ is not as
    important as the fact that the interval $i_{1}$ occurs after interval
    $i_{2}$.

    Allen \cite{AL} found that a total of 13 possible disjoint relations may
    exist between any two temporal intervals: $before$, $after$, $during$,
    $contains$, $meets$, $met$ $by$, $starts$, $started$ $by$, $finishes$,
    $finished$ $by$ and $equals$. Furthermore, as each of these temporal
    interval relations are disjoint, he proposed an algebra to represent a
    network of interval relations, which may be composed of partial or
    disjunctive interval relation information.

    At the time of writing, the authors of this paper are working on a new
    authorisation language ${\cal L}^{T}$, an extension of language ${\cal L}$
    with provisions to (1) express authorisation rules that hold only on
    specified time intervals, and to (2) allow the representation of temporal
    interval relations either under Allen's full interval algebra or one of its
    subalgebras \cite{KRO}.

  \section{Conclusion}
    \label{sec-conclusion}

    In this paper, we have presented the PolicyUpdater system, a logic-based
    authorisation system that features query evaluation and dynamic policy
    updates. This is made possible by the use of a first-order logic
    authorisation language, language ${\cal L}$, for the definition, updating
    and querying of access control policies. As we have shown, language
    ${\cal L}$ is expressive enough to represent constraints and default rules.

    The case study in Section \ref{sec-case} demonstrated how the PolicyUpdater
    system can be adapted to be used in a real-world web server authorisation
    application. As mentioned earlier, while other logic based access control
    approaches have been proposed recently, most of these cannot deal with
    dynamic policy updates. Furthermore, most of these approaches rarely
    address implementation issues. To the best of our knowledge, the
    PolicyUpdater system is the first fully implemented logic based access
    control system used in a web server security application.

    Finally, as discussed in Section \ref{sec-future}, we are currently working
    on extending language {\cal L}, and therefore the PolicyUpdater system, to
    support time-bound authorisation policies.

  \begin{thebibliography}{5}
    \bibitem{AL}
      Allen, J. F.
      Maintaining Knowledge about Temporal Intervals.
      {\em Communications of the ACM},
      Vol. 26, No. 11, pp. 832-843, ACM, 1983.
    \bibitem{AP}
      Apache Software Foundation,
      Authentication, Authorization and Access Control.
      {\em Apache HTTP Server Version 2.1 Documentation},
      {\tt \scriptsize http://httpd.apache.org/docs-2.1/}, 2004.

    \bibitem{BA1}
      Bai, Y., Varadharajan, V.,
      On Formal Languages for Sequences of Authorization Transformations.
      In {\em Proceedings of Safety, Reliability and Security of Computer
      Systems}. Also in {\em Lecture Notes in Computer Science},
      Vol. 1698, pp. 375-384. Springer-Verlag, 1999.

    \bibitem{BA2}
      Bai, Y., Varadharajan, V.,
      On Transformation of Authorization Policies.
      In {\em Data and Knowledge Engineering},
      Vol. 45, No. 3, pp. 333-357, 2003.

    \bibitem{BAR}
      Baral, C.,
      {\em Knowledge, Representation, Reasoning and Declarative Problem
      Solving}.
      pp. 99-100, Cambridge University Press, UK, 2003.

    \bibitem{BE1}
      Bertino, E., Buccafurri, F., Ferrari, E., Rullo, P.,
      A Logic-based Approach for Enforcing Access Control.
      {\em Journal of Computer Security},
      Vol. 8, No. 2-3, pp. 109-140, IOS Press, 2000.

    \bibitem{BE2}
      Bertino, E., Mileo, A., Provetti, A.
      Policy Monitoring with User-Preferences in PDL.
      In {\em Proceedings of IJCAI-03 Workshop for Nonmonotonic Reasoning,
      Action and Change},
      pp. 37-44, 2003.

    \bibitem{CHO}
      Chomicki, J., Lobo, J., Naqvi, S.,
      A Logic Programming Approach to Conflict Resolution in Policy Management.
      In {\em Proceedings of KR2000, 7th International Conference on Principles
      of Knowledge Representation and Reasoning},
      pp. 121-132, Kaufmann, 2000.

    \bibitem{CR1}
      Crescini, V. F., Zhang, Y.,
      A Logic Based Approach for Dynamic Access Control.
      In {\em Proceedings of the 17th Australian Joint Conference on Artificial
      Intelligence},
      2004. (to be published)

    \bibitem{CR2}
      Crescini, V. F., Zhang, Y., Wang, W.,
      Web Server Authorisation with the PolicyUpdater Access Control System.
      In {\em Proceedings of the 2004 IADIS WWW/Internet Conference},
      2004. (to be published)

    \bibitem{GEL}
      Gelfond, M., Lifschitz, V.,
      The Stable Model Semantics for Logic Programming.
      In {\em Proceedings of the 5th Joint International Conference and
      Symposium},
      pp. 1070-1080, MIT Press, 1998.

    \bibitem{HAL}
      Halpern, J. Y., Weissman, V.,
      Using First-Order Logic to Reason About Policies.
      In {\em Proceedings of the 16th IEEE Computer Security Foundations
      Workshop}, pp. 187-201, 2003.

    \bibitem{JAJ}
      Jajodia, S., Samarati, P., Sapino, M. L., Subrahmanian, V. S.,
      Flexible Support for Multiple Access Control Policies.
      {\em ACM Transactions on Database Systems},
      Vol. 29, No. 2, pp. 214-260, ACM, 2001.

    \bibitem{KRO}
      Krokhin, A., Jeavons, P., Jonsson, P.,
      Reasoning about Temporal Relations: The Tractable Subalgebras of
      Allen's Interval Algebra.
      {\em Journal of the ACM},
      Vol. 50, No. 5, pp. 591-640, ACM, 2003.

    \bibitem{LAU}
      Laurie, B., Laurie, P.,
      {\em Apache: The Definitive Guide} (3rd Edition).
      O'Reilly \& Associates Inc., 2003.

    \bibitem{LI}
      Li, N., Grosof, B. N., Feigenbaum, J.,
      Delegation Logic: A Logic-Based Approach to Distributed Authorization.
      {\em ACM Transactions on Information and System Security (TISSEC)},
      Vol. 6, No. 1, pp. 128-171, 2003.

    \bibitem{LIN}
      Lin, F., Zhao, X.,
      On Odd and Even Cycles in Normal Logic Programs.
      In {\em Proceedings of AAAI 19th National Conference on Artificial
      Intelligence and 16th Conference on Innovative Applications of Artificial
      Intelligence},
      pp. 80, AAAI Press, 2004.

    \bibitem{LOB}
      Lobo, J., Bhatia, R., Naqvi, S.,
      A Policy Description Language.
      In {\em Proceedings of AAAI 16th National Conference on Artificial
      Intelligence and 11th Conference on Innovative Applications of Artificial
      Intelligence},
      pp. 291-298, AAAI Press, 1999.

    \bibitem{HTTP1}
      Network Working Group,
      {\em HTTP 1.1 (RFC 2616)}.
      The Internet Society,
      {\tt \scriptsize ftp://ftp.isi.edu/in-notes/rfc2616.txt},
      1999.

    \bibitem{HTTP2}
      Network Working Group,
      {\em HTTP Authentication: Basic and Digest Access Authentication (RFC 2617)}.
      The Internet Society,
      {\tt \scriptsize ftp://ftp.isi.edu/in-notes/rfc2617.txt},
      1999.

    \bibitem{SIM}
      Simons., P.,
      Efficient Implementation of the Stable Model Semantics for Normal Logic
      Programs.
      {\em Research Reports, Helsinki University of Technology},
      No. 35, 1995.
  \end{thebibliography}

  \appendix

  \vappsection{Translation to Language $\cal{L^{*}}$}

     \label{app-trans}
     The following shows the language ${\cal L^{*}}$ translation of the
     language ${\cal L}$ program listing shown in Example \ref{ex-1}.

     \begin{enumerate}
       \item
         Initial Fact Rules

         \begin{vquote}
           $\hat{memb}(alice, grp2, S_{0}) \leftarrow$

           $\hat{holds}(grp1, read, file,S_{0}) \leftarrow$

           $\hat{subst}(grp2, grp1, S_{0}) \leftarrow$
         \end{vquote}

       \item
         Constraint Rules

         \begin{vquote}
           % constraints (S0)
           $\hat{holds}(grp1, write, file, S_{0})$ $\leftarrow$

           \hspace{1em}
           $\hat{holds}(grp1, read, file, S_{0})$,

           \hspace{1em}
           $not$ $\lnot\hat{holds}(grp3, write, file, S_{0})$
         \end{vquote}

         \begin{vquote}
           % constraints (S1)
           $\hat{holds}(grp1, write, file, S_{1})$ $\leftarrow$

           \hspace{1em}
           $\hat{holds}(grp1, read, file, S_{1})$,

           \hspace{1em}
           $not$ $\lnot\hat{holds}(grp3, write, file, S_{1})$
         \end{vquote}

       \item
         Policy Update Rules

         \begin{vquote}
           $\lnot\hat{holds}(grp1, read, file, S_{1})$ $\leftarrow$
         \end{vquote}

       \item
         Inheritance Rules

         \begin{vquote}
           % inheritance rules (positive, read, S0)
           $\hat{holds}(alice, read, file, S_{0})$ $\leftarrow$

           \hspace{1em}
           $\hat{holds}(grp1, read, file, S_{0})$,

           \hspace{1em}
           $\hat{memb}(alice, grp1, S_{0})$,

           \hspace{1em}
           $not$ $\lnot\hat{holds}(alice, read, file, S_{0})$
         \end{vquote}

         \begin{vquote}
           % inheritance rules (negative, read, S0)
           $\lnot\hat{holds}(alice, read, file, S_{0})$ $\leftarrow$

           \hspace{1em}
           $\lnot\hat{holds}(grp1, read, file, S_{0})$,

           \hspace{1em}
           $\hat{memb}(alice, grp1, S_{0})$
         \end{vquote}

         \begin{vquote}
           \hspace{2em}$\vdots$
         \end{vquote}

         \begin{vquote}
           % inheritance rules (positive, write, S1)
           $\hat{holds}(alice, write, file, S_{1})$ $\leftarrow$

           \hspace{1em}
           $\hat{holds}(grp3, write, file, S_{1})$,

           \hspace{1em}
           $\hat{memb}(alice, grp3, S_{1})$,

           \hspace{1em}
           $not$ $\lnot\hat{holds}(alice, write, file, S_{1})$
         \end{vquote}

         \begin{vquote}
           % inheritance rules (negative, write, S1)
           $\lnot\hat{holds}(alice, write, file, S_{1})$ $\leftarrow$

           \hspace{1em}
           $\lnot \hat{holds}(grp3, write, file, S_{1})$,

           \hspace{1em}
           $\hat{memb}(alice, grp3, S_{1})$
         \end{vquote}

         \begin{vquote}
           % inheritance rules (subset positive, read, S0)
           $\hat{holds}(grp1, read, file, S_{0})$ $\leftarrow$

           \hspace{1em}
           $\hat{holds}(grp2, read, file, S_{0})$,

           \hspace{1em}
           $\hat{subst}(grp1, grp2, S_{0})$
         \end{vquote}

         \begin{vquote}
           % inheritance rules (subset negative, read, S0)
           $\lnot \hat{holds}(grp1, read, file, S_{0})$ $\leftarrow$

           \hspace{1em}
           $\lnot \hat{holds}(grp2, read, file, S_{0})$,

           \hspace{1em}
           $\hat{subst}(grp1, grp2, S_{0})$
         \end{vquote}

         \begin{vquote}
           \hspace{2em}$\vdots$
         \end{vquote}

         \begin{vquote}
           % inheritance rules (subset positive, write, S1)
           $\hat{holds}(grp3, write, file, S_{1})$ $\leftarrow$

           \hspace{1em}
           $\hat{holds}(grp2, write, file, S_{1})$,

           \hspace{1em}
           $\hat{subst}(grp3, grp2, S_{1})$
         \end{vquote}

         \begin{vquote}
           % inheritance rules (subset negative, write, S1)
           $\lnot \hat{holds}(grp3, write, file, S_{1})$ $\leftarrow$

           \hspace{1em}
           $\lnot \hat{holds}(grp2, write, file, S_{1})$,

           \hspace{1em}
           $\hat{subst}(grp3, grp2, S_{1})$
         \end{vquote}

       \item
         Transitivity Rules

         \begin{vquote}
           $\hat{subst}(grp1, grp3, S_{0})$ $\leftarrow$

           \hspace{1em}
           $\hat{subst}(grp1, grp2, S_{0})$,
           $\hat{subst}(grp2, grp3, S_{0})$
         \end{vquote}

         \begin{vquote}
           \hspace{2em}$\vdots$
         \end{vquote}

         \begin{vquote}
           $\hat{subst}(grp3, grp1, S_{0})$ $\leftarrow$

           \hspace{1em}
           $\hat{subst}(grp3, grp2, S_{0})$,
           $\hat{subst}(grp2, grp1, S_{0})$
         \end{vquote}

         \begin{vquote}
           $\hat{subst}(grp1, grp3, S_{1})$ $\leftarrow$

           \hspace{1em}
           $\hat{subst}(grp1, grp2, S_{1})$,
           $\hat{subst}(grp2, grp3, S_{1})$
         \end{vquote}

         \begin{vquote}
           \hspace{2em}$\vdots$
         \end{vquote}

         \begin{vquote}
           $\hat{subst}(grp3, grp1, S_{1})$ $\leftarrow$

           \hspace{1em}
           $\hat{subst}(grp3, grp2, S_{1})$,
           $\hat{subst}(grp2, grp1, S_{1})$
         \end{vquote}

       \item
         Inertial Rules

         \begin{vquote}
           % inertial rule holds(alice, read, file)
           $\hat{holds}(alice, read, file, S_{1})$ $\leftarrow$

           \hspace{1em}
           $\hat{holds}(alice, read, file, S_{0})$,

           \hspace{1em}
           $not$ $\lnot\hat{holds}(alice, read, file, S_{1})$
         \end{vquote}

         \begin{vquote}
           % inertial rule !holds(alice, read, file)
           $\lnot\hat{holds}(alice, read, file, S_{1})$ $\leftarrow$

           \hspace{1em}
           $\lnot\hat{holds}(alice, read, file, S_{0})$,

           \hspace{1em}
           $not$ $\lnot\hat{holds}(alice, read, file, S_{1})$
         \end{vquote}

         \begin{vquote}
           % inertial rule holds(alice, write, file)
           $\hat{holds}(alice, write, file, S_{1})$ $\leftarrow$

           \hspace{1em}
           $\hat{holds}(alice, write, file, S_{0})$,

           \hspace{1em}
           $not$ $\lnot\hat{holds}(alice, write, file, S_{1})$
         \end{vquote}

         \begin{vquote}
           % inertial rule !holds(alice, write, file)
           $\lnot\hat{holds}(alice, write, file, S_{1})$ $\leftarrow$

           \hspace{1em}
           $\lnot\hat{holds}(alice, write, file, S_{0})$,

           \hspace{1em}
           $not$ $\lnot\hat{holds}(alice, write, file, S_{1})$
         \end{vquote}

         \begin{vquote}
           % inertial rule holds(grp1, read, file)
           $\hat{holds}(grp1, read, file, S_{1})$ $\leftarrow$

           \hspace{1em}
           $\hat{holds}(grp1, read, file, S_{0})$,

           \hspace{1em}
           $not$ $\lnot\hat{holds}(grp1, read, file, S_{1})$
         \end{vquote}

         \begin{vquote}
           % inertial rule !holds(grp1, read, file)
           $\lnot\hat{holds}(grp1, read, file, S_{1})$ $\leftarrow$

           \hspace{1em}
           $\lnot\hat{holds}(grp1, read, file, S_{0})$,

           \hspace{1em}
           $not$ $\lnot\hat{holds}(grp1, read, file, S_{1})$
         \end{vquote}

         \begin{vquote}
           \hspace{2em}$\vdots$
         \end{vquote}

         \begin{vquote}
           % inertial rule holds(grp3, read, file)
           $\hat{holds}(grp3, read, file, S_{1})$ $\leftarrow$

           \hspace{1em}
           $\hat{holds}(grp3, read, file, S_{0})$,

           \hspace{1em}
           $not$ $\lnot\hat{holds}(grp3, read, file, S_{1})$
         \end{vquote}

         \begin{vquote}
           % inertial rule !holds(grp3, read, file)
           $\lnot\hat{holds}(grp3, read, file, S_{1})$ $\leftarrow$

           \hspace{1em}
           $\lnot\hat{holds}(grp3, read, file, S_{0})$,

           \hspace{1em}
           $not$ $\lnot\hat{holds}(grp3, read, file, S_{1})$
         \end{vquote}

         \begin{vquote}
           % inertial rule holds(grp1, write, file)
           $\hat{holds}(grp1, write, file, S_{1})$ $\leftarrow$

           \hspace{1em}
           $\hat{holds}(grp1, write, file, S_{0})$,

           \hspace{1em}
           $not$ $\lnot\hat{holds}(grp1, write, file, S_{1})$
         \end{vquote}

         \begin{vquote}
           % inertial rule !holds(grp1, write, file)
           $\lnot\hat{holds}(grp1, write, file, S_{1})$ $\leftarrow$

           \hspace{1em}
           $\lnot\hat{holds}(grp1, write, file, S_{0})$,

           \hspace{1em}
           $not$ $\lnot\hat{holds}(grp1, write, file, S_{1})$
         \end{vquote}

         \begin{vquote}
           \hspace{2em}$\vdots$
         \end{vquote}

         \begin{vquote}
           % inertial rule holds(grp3, write, file)
           $\hat{holds}(grp3, write, file, S_{1})$ $\leftarrow$

           \hspace{1em}
           $\hat{holds}(grp3, write, file, S_{0})$,

           \hspace{1em}
           $not$ $\lnot\hat{holds}(grp3, write, file, S_{1})$
         \end{vquote}

         \begin{vquote}
           % inertial rule !holds(grp3, write, file)
           $\lnot\hat{holds}(grp3, write, file, S_{1})$ $\leftarrow$

           \hspace{1em}
           $\lnot\hat{holds}(grp3, write, file, S_{0})$,

           \hspace{1em}
           $not$ $\lnot\hat{holds}(grp3, write, file, S_{1})$
         \end{vquote}

         \begin{vquote}
           % inertial rule memb(alice, grp1)
           $\hat{memb}(alice, grp1, S_{1})$ $\leftarrow$

           \hspace{1em}
           $\hat{memb}(alice, grp1, S_{0})$,

           \hspace{1em}
           $not$ $\lnot\hat{memb}(alice, grp1, S_{1})$
         \end{vquote}

         \begin{vquote}
           % inertial rule !memb(alice, grp1)
           $\lnot \hat{memb}(alice, grp1, S_{1})$ $\leftarrow$

           \hspace{1em}
           $\lnot\hat{memb}(alice, grp1, S_{0})$,

           \hspace{1em}
           $not$ $\hat{memb}(alice, grp1, S_{1})$
         \end{vquote}

         \begin{vquote}
           \hspace{2em}$\vdots$
         \end{vquote}

         \begin{vquote}
           % inertial rule memb(alice, grp3)
           $\hat{memb}(alice, grp3, S_{1})$ $\leftarrow$

           \hspace{1em}
           $\hat{memb}(alice, grp3, S_{0})$,

           \hspace{1em}
           $not$ $\lnot\hat{memb}(alice, grp3, S_{1})$
         \end{vquote}

         \begin{vquote}
           % inertial rule !memb(alice, grp3)
           $\lnot\hat{memb}(alice, grp3, S_{1})$ $\leftarrow$

           \hspace{1em}
           $\lnot\hat{memb}(alice, grp3, S_{0})$,

           \hspace{1em}
           $not$ $\hat{memb}(alice, grp3, S_{1})$
         \end{vquote}

         \begin{vquote}
           % inertial rule subst(grp1, grp1)
           $\hat{subst}(grp1, grp1, S_{1})$ $\leftarrow$

           \hspace{1em}
           $\hat{subst}(grp1, grp1, S_{0})$,

           \hspace{1em}
           $not$ $\lnot\hat{subst}(grp1, grp1, S_{1})$
         \end{vquote}

         \begin{vquote}
           % inertial rule !subst(grp1, grp1)
           $\lnot\hat{subst}(grp1, grp1, S_{1})$ $\leftarrow$

           \hspace{1em}
           $\lnot\hat{memb}(grp1, grp1, S_{0})$,

           \hspace{1em}
           $not$ $\hat{memb}(grp1, grp1, S_{1})$
         \end{vquote}

         \begin{vquote}
           \hspace{2em}$\vdots$
         \end{vquote}

         \begin{vquote}
           % inertial rule subst(grp3, grp3)
           $\hat{subst}(grp3, grp3, S_{1})$ $\leftarrow$

           \hspace{1em}
           $\hat{subst}(grp3, grp3, S_{0})$,

           \hspace{1em}
           $not$ $\lnot\hat{subst}(grp3, grp3, S_{1})$
         \end{vquote}

         \begin{vquote}
           % inertial rule !subst(grp3, grp3)
           $\lnot\hat{subst}(grp3, grp3, S_{1})$ $\leftarrow$

           \hspace{1em}
           $\lnot\hat{memb}(grp3, grp3, S_{0})$,

           \hspace{1em}
           $not$ $\hat{memb}(grp3, grp3, S_{1})$
         \end{vquote}
       \item
         Identity Rules

         \begin{vquote}
           $\hat{subset}(grp1, grp1, S_{0})$ $\leftarrow$

           $\hat{subset}(grp2, grp2, S_{0})$ $\leftarrow$

           $\hat{subset}(grp3, grp3, S_{0})$ $\leftarrow$

           $\hat{subset}(grp1, grp1, S_{1})$ $\leftarrow$

           $\hat{subset}(grp2, grp2, S_{1})$ $\leftarrow$

           $\hat{subset}(grp3, grp3, S_{1})$ $\leftarrow$
         \end{vquote}
     \end{enumerate}

  \vappsection{Storage Structures}

    \label{app-store}

    The data structures outlined in this section are used as a storage
    structure to hold the elements of language ${\cal L}$ before any
    operations are performed.

    Each of the tables and lists used in the system inherits from a generic
    ordered and indexed list implementation. Each node in this list holds a
    generic data type that can be used to store strings, an arbitrary data
    type or another list type.

    \subsection{Symbol Table}

      The symbol table is used to store the identifier entities defined in
      the entity identifier declaration section of language ${\cal L}$
      programs. The symbol table is composed of 6 separate string lists:

      \begin{vquote}
        \begin{tabular}[t]{|l|l|l|}
          \hline
          \textbf{Field} & \textbf{Type} & \textbf{Description} \\
          \hline
          $ss$ & string list & single subject \\
          \hline
          $sg$ & string list & group subject \\
          \hline
          $as$ & string list & single access right \\
          \hline
          $ag$ & string list & group access right \\
          \hline
          $os$ & string list & single object \\
          \hline
          $og$ & string list & group object \\
          \hline
        \end{tabular}
      \end{vquote}

      Each entity identifier are sorted in the above lists according to
      their type, and ordered according to the order in which they are
      declared in the program. Each list is indexed by consecutive
      positive integers starting from zero.

    \subsection{Policy Base}

      When a language ${\cal L}$ program is parsed, each of the facts,
      rules and policy updates must first be stored into the policy base.
      The policy base is composed of 4 tables to store the following:
      initial state facts, constraint rules, policy update definitions and
      the policy update sequence.

      \subsubsection{Atoms.}

        The three types of atoms (holds, membership and subset) are
        represented as structures with 2 to 3 strings, with each string
        matching an entity identifier from the symbol table.

        \begin{vquote}
          \begin{tabular}[t]{|l|l|l|l|}
            \hline
            \textbf{Atom} & \textbf{Field} & \textbf{Type} & \textbf{Description} \\
            \hline
            {\multirow{3}{*}{holds}} & $sub$ & string & subject entity \\
            \cline{2-4}
            & $acc$ & string & access right entity \\
            \cline{2-4}
            & $obj$ & string & object entity \\
            \hline
            \hline
            {\multirow{2}{*}{member}} & $elt$ & string & single entity \\
            \cline{2-4}
            & $grp$ & string & group entity \\
            \hline
            \hline
            {\multirow{2}{*}{subset}} & $grp1$ & string & subgroup entity \\
            \cline{2-4}
            & $grp2$ & string & supergroup entity \\
            \hline
          \end{tabular}
        \end{vquote}

      \subsubsection{Facts.}

        Facts are stored in a three-element structure composed of the
        following: polymorphic type which can be any of the three atom
        structures above; a type indicator to specify whether the fact is
        $holds$, $member$ or $subset$; and a truth flag, to indicate whether the
        atom is classically negated or not ($true$ if the fact holds
        and $false$ if the classical negation of the fact holds).

        \begin{vquote}
          \begin{tabular}[t]{|l|l|l|}
            \hline
            \textbf{Field} & \textbf{Type} & \textbf{Description} \\
            \hline
            $atom$ & atom type & polymorphic structure \\
            \hline
            $type$ & \{h$|$m$|$s\} & holds, member or subset \\
            \hline
            $truth$ & boolean & negation indicator \\
            \hline
          \end{tabular}
        \end{vquote}

      \subsubsection{Expressions.}

        Since expressions are simply conjunctions of facts, they are
        represented as a list of fact structures.

      \subsubsection{Initial State Facts Table.}

        The initial state facts table is represented as a single list of
        fact structures, or an expression. Each fact in all {\em initially}
        statements are added into the initial state facts table.

      \subsubsection{Constraint Table.}

        The constraint table is represented as a list of constraint
        structures, with each structure composed of the following:

        \begin{vquote}
          \begin{tabular}[t]{|l|l|l|}
            \hline
            \textbf{Field} & \textbf{Type} & \textbf{Description} \\
            \hline
            $exp$ & expression type & consequent \\
            \hline
            $pcond$ & expression type & positive premise \\
            \hline
            $ncond$ & expression type & negative premise \\
            \hline
          \end{tabular}
        \end{vquote}

      \subsubsection{Policy Update Definition Table.}

        Another list of structures is the policy update table. Each element
        structure of this table is composed of the following 4 fields:

        \begin{vquote}
          \begin{tabular}[t]{|l|l|l|}
            \hline
            \textbf{Field} & \textbf{Type} & \textbf{Description} \\
            \hline
            $name$ & string & update identifier \\
            \hline
            $vlist$ & ordered string list & variables \\
            \hline
            $pre$ & expression type & precondition \\
            \hline
            $post$ & expression type & postcondition \\
            \hline
          \end{tabular}
        \end{vquote}

    \subsection{Policy Update Sequence Table}

      The policy update sequence table is an ordered list of sequence
      structures, each with the following elements:

      \begin{vquote}
        \begin{tabular}[t]{|l|l|l|}
          \hline
          \textbf{Field} & \textbf{Type} & \textbf{Description} \\
          \hline
          $name$ & string & update identifier \\
          \hline
          $ilist$ & ordered string list & identifiers \\
          \hline
        \end{tabular}
      \end{vquote}

\end{document}
