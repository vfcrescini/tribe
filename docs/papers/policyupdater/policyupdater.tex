\documentclass[10pt, twocolumn]{article}
\usepackage{latex8}
\usepackage{times}
\usepackage{amsmath}

\newtheorem{definition}{Definition}
\newtheorem{examp}{Example}
\newenvironment{example}{\begin{examp}\rm}{\rule{2mm}{2mm}\end{examp}}

\begin{document}
  \title{
    PolicyUpdater -- A System for Dynamic Access Control:  \\
    Formalization, Implementation and a Case Study
  }

  \author{
    Vino Fernando Crescini and Yan Zhang                   \\
    School of Computing and Information Technology         \\
    University of Western Sydney                           \\
    Penrith South DC, NSW 1797, Australia                  \\
    E-mail: \{jcrescin,yan\}@cit.uws.edu.au
  }

  \date{}

  \maketitle

  \begin{abstract}
    [text]
  \end{abstract}

  \section{Introduction}

    \subsection{Overview of Logic-Based Approaches}

    \subsection{Contribution}

    \subsection{Structure of this Paper}

  \section{Language $\cal{L}$}

    \subsection{Syntax}

      [language statements, statement terminators, and comments]

      \subsubsection{Components of Language $\cal{L}$}

        \paragraph{Identifiers.}
          The most basic unit of language $\cal{L}$ is an identifier.
          Identifiers are used to represent the different components of the
          language. 
    
          The first character of an identifier must be an alpha character
          followed by 0 to 127 characters of alpha, digit or underscore
          characters.

          \begin{verbatim}[a-zA-Z]([a_zA-Z0-9_]){0,127}\end{verbatim}

          Identifiers are divided into three main classes:

          \begin{itemize}
            \item
              $Entity$ $Identifiers$ represent entities that make up a logical
              atom. They are further divided into three types, with each type
              again divided into the $singular$ $entity$ and $group$ $entity$
              categories:
              \begin{itemize}
                \item
                  $Subjects$: privilege holders (e.g. alice, lecturers,
                  students).
                \item
                  $Access Rights$: privilege (e.g. read, write, own).
                \item
                  $Objects$: privilege entities (e.g. file, database,
                  directory).
              \end{itemize}
            \item
              $Policy$ $Update$ $Identifiers$ are used to name a policy
              update definition or directive.
            \item
              $Variable$ $Identifiers$ are used as entity identifier
              place-holders in policy updates.
          \end{itemize}

        \paragraph{Atoms.}
          An atom is composed of a relation with 2 to 3 entity identifiers
          that represent a logical relationship between the entities. There are
          three types of atoms:

          \begin{itemize}
            \item
              $Holds$. An atom of this type states that the subject identifier
              $sub$ holds the access right identifier $acc$ for the object
              identifier $obj$.
         
              \begin{verbatim}holds(<sub>, <acc>, <obj>)\end{verbatim}
            \item
              $Membership$. This type of atom states that the singular
              identifier $elt$ is a member or element of the group identifier
              $grp$. It is important to note that identifiers $elt$ and $grp$
              must be of the same base type (e.g. subject and subject group).
         
              \begin{verbatim}memb(<elt>, <grp>)\end{verbatim}
            \item
              $Subset$. The subset atom states that the groups identifiers
              $grp1$ and $grp2$ are of the same types and that group $grp1$
              is a subset of the group $grp2$.

              \begin{verbatim}subst(<grp1>, <grp2>)\end{verbatim}
          \end{itemize}

        \paragraph{Facts.}
          A fact makes a claim that the relationship represented by an atom or
          its negation holds in the current context. Facts are negated by the
          use of the negation operator ($!$). The following shows the formal
          syntax of a fact:
 
          \begin{verbatim}[!]<holds_atm>|<memb_atm>|<subst_atm>\end{verbatim}
 
        \paragraph{Expressions.}
          An expression is either a fact, or a logical conjunction of facts,
          separated by the double-ampersand characters $\&\&$.

          \begin{verbatim}<fact1> [&& <fact2> [&& ...]]\end{verbatim}

      \subsubsection{Identifier Entity Declaration}

        All entity identifiers (subjects, access rights, objects and groups)
        must first be declared before any other statements to define the
        entity domain of the policy base. The following entity declaration
        syntax illustrates how to define one or more entity identifiers of a
        particular type.

        \begin{verbatim}ident sub|acc|obj[-grp] <name>[, ...]];\end{verbatim}

      \subsubsection{Initial State Definition}

        The initial facts in the policy base can be defined by the following
        syntax:

        \begin{verbatim}initially <expression>;\end{verbatim}

      \subsubsection{Constraint Definition}

        Constraints are logical rules that holds regardless of any changes
        that may occur when the policy base is updated. The constraint rules
        are true in the initial state and remains true after any policy update.

        The constraint syntax below shows that for any state of the policy
        base, expression $exp1$ holds if expression $exp2$ is true and there
        is no evidence that $exp3$ is true.

        Note that the conditional clauses $implied$ $by$ and $with$ $absence$
        are optional.

\begin{verbatim}
always <exp1>
  [implied by <exp2>
  [with absence <exp3>]];
\end{verbatim}

      \subsubsection{Policy Update Definition}

        Before a policy update can be applied, it must first be defined by
        by using the following syntax:

\begin{verbatim}
<up_ident>([<var_ident>[, ...]])
  causes <exp1>
  if <exp2>;
\end{verbatim}

        $up\_ident$ is the policy update identifier to be used in referencing
        this identifier. The optional $var\_ident$ list are the variable
        identifiers occurring the expressions and will eventually be replaced
        by entity identifiers when the update is referenced. The postcondition
        expression $exp1$ is an expression that will hold in the state after
        this update is applied. The expression $exp2$ is a precondition
        expression that must hold in the current state before this update is
        applied.

        It is important to note that a defined policy update has no effect
        whatsoever on the policy base until it is applied by one of the
        directives described in the next section.

      \subsubsection{Policy Update Directives}

        The policy update sequence list contains a list of references to
        defined policy updates in the domain. The policy updates in the
        sequence list are applied to the current state of the policy base one
        at a time to produce a policy base state upon which queries can be
        evaluated. The order in which the policy update references appear in
        sequence list is the same order in which they were added.

        The following four directives are the policy sequence manipulation
        features of language $\cal{L}$.

        \paragraph{Adding an Update into the Sequence.}
          Defined policy updates are added into the sequence list through the
          syntax below.

          \begin{verbatim}seq add <up_ident>([<e_ident>[, ...]]);\end{verbatim}

          \noindent where $up\_ident$ is the identifier of a defined policy
          update and the $e\_ident$ list is a comma-separated list of entity
          identifiers that will replace the variable identifiers that occur in
          the definition of the policy update. It is important to note that the
          replacement of variable identifiers with entity identifiers is
          dependent on the order in which they appear in the list.

        \paragraph{Listing the Updates in the Sequence.}
          The following directive may be used to list the current contents of
          the policy update sequence list.

          \begin{verbatim}seq list;\end{verbatim}

          This directive is answered with an ordinal list of policy updates in
          the form:

          \begin{verbatim}<n> <up_ident>([e_ident[, ...]])\end{verbatim}

          \noindent where $n$ is the ordinal index of the policy update within
          the sequence list starting at 0. $up_ident$ is the policy update
          identifier and the $e_ident$ list is the list of entity identifiers
          used to replace the variable identifier place-holders.

        \paragraph{Removing an Update from the Sequence.}
          The syntax below shows the directive to remove a policy update
          reference from the list. $n$ is the ordinal index of the policy
          update to be removed. Note that removing a policy update reference
          from the sequence list may change the ordinal index of the other
          update references.

          \begin{verbatim}seq del <n>;\end{verbatim}

        \paragraph{Computing after a Policy Update.}

          The policy updates in the sequence list does not get applied until
          the compute directive is issued. The directive causes the policy
          update references in the sequence list to be applied one at a time in
          the same order that they appear in the list. The directive also
          causes the system to generate the policy base models against which
          query requests can be evaluated.

          \begin{verbatim}compute;\end{verbatim}

      \subsubsection{Query Directive}

        A query expression may be issued against the current state of the
        policy base. This current state is derived after all the updates in
        the update sequence has been applied, one at a time, onto the initial
        state. Query expressions are answered with a \emph{true}, \emph{false}
        or an \emph{unknown}, depending on whether the queried expression
        holds, its negation holds, or neither, respectively. Syntax is as
        follows:

        \begin{verbatim}query <exp>;\end{verbatim} 

      \begin{example}
        The following language L program code listing shows the chinese-wall
        example.  The main function of this program is to ensure that the
        subject $alice$ is denied $read$ access to any object in one group if
        she already holds $read$ to the other group.

        In this example, the two objects $db1$ and $db2$ are members of the
        groups $comp1$ and $comp2$, respectively. The policy updates
        $req\_read\_comp1$ and $req\_read\_comp2$ both grant $read$ access
        to the given subject for the given object and automatically revokes
        the subject's right to read objects from the other group. The policy
        updates can only be applied if their respective preconditions have
        been met, i.e. that the given subject has $readable$ access for the
        given object and that the given object is a member of the group
        concerned.
        \begin{verbatim}

/* declare identifiers */

ident sub alice;
ident acc read, readable;
ident obj db1, db2;
ident obj-grp comp1, comp2;

/* initial fact definitions */

initially
  holds(alice, readable, comp1) &&
  holds(alice, readable, comp2);

/* constraint definitions */

always memb(db1, comp1);
always memb(db2, comp2);

/* policy update definitions */

req_read_comp1(SUB, OBJ)
  causes
    holds(SUB, read, OBJ) &&
    !holds(SUB, readable, comp2)
  if
    memb(OBJ, comp1) &&
    holds(SUB, readable, comp1);

req_read_comp2(SUB, OBJ)
  causes
    holds(SUB, read, OBJ) &&
    !holds(SUB, readable, comp1)
  if
    memb(OBJ, comp2) &&
    holds(SUB, readable, comp2);

/* sequence manipluation */

seq add req_read_comp1(alice, db1);
seq add req_read_comp2(alice, db2);
compute;

/* query */

query holds(alice, read, db1);
query holds(alice, read, db2);
query holds(alice, read, comp1);
query holds(alice, read, comp2);
        \end{verbatim}

        The result should be:

        \begin{verbatim}
TRUE
?
TRUE
FALSE
        \end{verbatim}
      \end{example}

    \subsection{Semantics}

      \subsubsection{Domain Definition of Language ${\cal L}$}

        \begin{definition}
          The domain of language ${\cal L}$ is as follows:

          \begin{quote}
            \begin{math}
              {\cal D}_{\cal L} = 
              \begin{cases}
                \mbox{set of Initial State Facts $\alpha$} \\
                \mbox{set of Constraint Rules $\beta$} \\
                \mbox{set of Policy Update Definitions $\gamma$}
              \end{cases}
            \end{math}
          \end{quote}

          \begin{itemize}
            \item
              The initial state facts $\alpha$ is an expression, or a
              conjunction of facts that hold in the initial state.
            \item
              The constraint rules $\beta$ are the rules that holds in all
              states.
            \item
              The policy update definitions $\gamma$ contains the name,
              preconditions and postconditions of all policy updates that
              can be applied in the policy base.
          \end{itemize}
        \end{definition}

        In addition to the domain ${\cal D}_{\cal L}$, language ${\cal L}$
        includes one additional ordered set: the sequence list $\psi$.

        \begin{definition}
          The sequence list $\psi$ is an ordered set that contains a sequence 
          of references to policy update definitions from set $\gamma$. Each
          policy update reference consists of the policy update identifier and
          a set of zero or more entities from set ${\cal E}$ to replace the
          variable place-holders in the policy update definitions.
        \end{definition}

        Given the domain ${\cal D}_{\cal L}$, a query expression may be
        evaluated against the state of the policy base derived after a sequence
        of policy updates from the sequence list $\psi$ are applied.

        \begin{definition}
          A query expression $\omega$ is said to hold in the policy base
          $PB_{{\cal D}_{\cal L}}$ with the sequence list $\psi$ if and only
          if:

          \begin{quote}
            $PB_{{\cal D}_{\cal L}}$ $\models$ $\omega$ after $\psi$
          \end{quote}
          
        \end{definition}

      \subsubsection{Language ${\cal L}^{*}$}

        The elements of language ${\cal L}$ must first be expressed as a Normal
        Logic Program before logical operations such as querying and updating
        can be performed. The language ${\cal L}^{*}$ is a direct translation
        of language ${\cal L}$, and as such, it contains the same elements and
        has an equivalent expressive power.

      \subsubsection{Entity Definition of Language ${\cal L}^{*}$}

        \begin{definition}

          The entity set ${\cal E}$ is the union of six disjoint entity sets:
          single subject ${\cal E}_{SS}$, group subject ${\cal E}_{SG}$,
          single access right ${\cal E}_{AS}$, group access right
          ${\cal E}_{AG}$, single object ${\cal E}_{OS}$ and group object
          ${\cal E}_{OG}$.

          \begin{enumerate}
            \item
              ${\cal E} = {\cal E}_{S} \cup {\cal E}_{A} \cup {\cal E}_{O}$
            \item
              ${\cal E}_{S} = {\cal E}_{SS} \cup {\cal E}_{SG}$
            \item
              ${\cal E}_{A} = {\cal E}_{AS} \cup {\cal E}_{AG}$
            \item
              ${\cal E}_{O} = {\cal E}_{OS} \cup {\cal E}_{OG}$
          \end{enumerate}

        \end{definition}

      \subsubsection{State Semantics}

        Conceptually, a state may be thought of as a set of facts and
        constraints of the policy base at a particular instant. The state
        transition notation below shows that the new state $PB'$ is generated
        from the current state $PB$ after the policy update $U$ is applied.

        \begin{quote}
          $PB$ $\overrightarrow{_{U}}$ $PB'$
        \end{quote}

        Using this definition of a state means that for every policy update
        applied to the policy base, a new instance of the policy base or a new 
        set of facts and constraints are generated. To avoid having to deal
        with multiple instances of the policy base, the semantics of language
        ${\cal L}^{*}$ uses a simpler and less tangible definition of a state:

        \begin{definition}
          State is a property of atoms in the policy base used to differentiate
          atoms that hold before the application of a particular policy update,
          from the atoms that hold after the application. Consequently, this
          implies that states are defined only after the sequence list $\psi$
          has been defined.
        \end{definition}

        \begin{definition}
          The total number of states $\Sigma$ is the number of policy updates
          in the sequence list $\psi$ plus the initial state.
          \begin{quote}
            $\Sigma$ = $|\psi|$ + $1$
          \end{quote}
        \end{definition}

        \begin{definition}
          The result function $Res$ takes a policy update reference $U$ and
          the current state $S$ as input and returns the resulting state $S'$
          after $U$ has been applied to $S$.

          \begin{quote}
            $S'$ = $Res$($U$, $S$)
          \end{quote}
        \end{definition}

      \subsubsection{Atoms, Facts and Expressions}

        The principal difference between language ${\cal L}$ and language
        ${\cal L}^{*}$ lies in the definition of atoms. Atoms in language
        ${\cal L}^{*}$ includes an additional state parameter $\sigma$, thereby
        limiting the scope of the atom to a particular state in the policy
        base.

        \begin{definition}
          An atom represents a logical relationship of two to three entities
          is a particular policy base state. The atom set $\cal{A}^{\sigma}$ is
          the set of all possible atoms in state $\sigma$.

          \begin{enumerate}
            \item
              ${\cal A}^{\sigma} = {\cal A}^{\sigma}_{H} \cup {\cal A}^{\sigma}_{M} \cup {\cal A}^{\sigma}_{S}$
            \item
              ${\cal A}^{\sigma}_{H} = \{\forall (s \in {\cal E}_{S}, a \in {\cal E}_{A}, o \in {\cal E}_{O}, holds(s, a, o, \sigma) \}$
            \item
              ${\cal A}^{\sigma}_{M} = {\cal A}^{\sigma}_{MS} \cup {\cal A}^{\sigma}_{MA} \cup {\cal A}^{\sigma}_{MO}$
            \item
              ${\cal A}^{\sigma}_{S} = {\cal A}^{\sigma}_{SS} \cup {\cal A}^{\sigma}_{SA} \cup {\cal A}^{\sigma}_{SO}$
            \item
              ${\cal A}^{\sigma}_{MS} = \{\forall (e \in {\cal E}_{SS}, g \in {\cal E}_{SG}), memb(e, g, \sigma)\}$
            \item
              ${\cal A}^{\sigma}_{MA} = \{\forall (e \in {\cal E}_{AS}, g \in {\cal E}_{AG}), memb(e, g, \sigma)\}$
            \item
              ${\cal A}^{\sigma}_{MO} = \{\forall (e \in {\cal E}_{OS}, g \in {\cal E}_{OG}), memb(e, g, \sigma)\}$
            \item
              ${\cal A}^{\sigma}_{SS} = \{\forall (g1, g2 \in {\cal E}_{SG}), subst(g1, g2, \sigma)\}$
            \item
              ${\cal A}^{\sigma}_{SA} = \{\forall (g1, g2 \in {\cal E}_{AG}), subst(g1, g2, \sigma)\}$
            \item
              ${\cal A}^{\sigma}_{SO} = \{\forall (g1, g2 \in {\cal E}_{OG}), subst(g1, g2, \sigma)\}$
          \end{enumerate}

        \end{definition} 

        \begin{definition}
          A fact is a logical statement that makes a claim that an atom either
          holds or does not hold in a particular state $\sigma$. The following
          is the formal definition of a fact:

          \begin{quote}
            $f^{\sigma}$ = $[\lnot]$$a$, $a$ $\in$ ${\cal A}^{\sigma}$
          \end{quote}
        \end{definition}

        \begin{definition}
          An expression is a fact or a conjunction of facts of a particular
          state $\sigma$.

          \begin{quote}
            $e^{\sigma}$ = $f^{\sigma}_{0}$ $\land$ ... $\land$ $f^{\sigma}_{n}$, $n$ $\geq$ $0$
          \end{quote}
        \end{definition}

      \subsubsection{Translating Language ${\cal L}$ to Language ${\cal L^{*}}$}

        \noindent \textbf{\emph{Initial State Expression}}

          To translate the initial state expressions of language ${\cal L}$ to
          language ${\cal L}^{*}$, the state parameter $\sigma$ of every atom
          in the expression is replaced by the initial state $S_{0}$.

          \begin{definition}
            An initial state expression is of the form:

            \begin{quote}
              $f_{0}$ $\land$ .. $\land$ $f_{n}$, $n$ $\geq$ $0$

              where $f_{i}$ = $[\lnot]$ $a$, $a$ $\in$ ${\cal A}^{S_{0}}$, $0$ $\leq$ $i$ $\leq$ $n$
            \end{quote}

          \end{definition}

        \noindent \textbf{\emph{Constraints}}

          Constraints in ${\cal L}$ is in the form:

          \begin{quote}
            always $a_{0}$ $\land$ ... $\land$ $a_{m}$
            implied by $b_{0}$ $\land$ ... $\land$ $b_{n}$
            with absence $not$ $c_{0}$ $\land$ ... $\land$ $not$ $c_{o}$

            where $m$, $n$, $o$ $\geq$ $0$
          \end{quote}

          To translate a constraint from language ${\cal L}$ to language
          ${\cal L}^{*}$, an instance of the constraint rule must be defined
          for each state $S_{0}$ to $S_{\Sigma}$.

          \begin{definition}
            A constraint is of the form:

            \begin{quote}
              $a_{0}$ $\land$ ... $\land$ $a_{m}$ $\leftarrow$ $b_{0}$ $\land$ ... $\land$ $b_{n}$ $\land$ $not$ $c_{0}$ $\land$ ... $\land$ $not$ $c_{o}$

              where:

              \begin{itemize}
                \item
                  $a_{i}$ = $[\lnot]$ $a$, $a$ $\in$ ${\cal A}^{\sigma}$, $0$ $\leq$ $i$ $\leq$ $m$
                \item
                  $b_{j}$ = $[\lnot]$ $b$, $b$ $\in$ ${\cal A}^{\sigma}$, $0$ $\leq$ $j$ $\leq$ $n$
                \item
                  $c_{k}$ = $[\lnot]$ $c$, $c$ $\in$ ${\cal A}^{\sigma}$, $0$ $\leq$ $k$ $\leq$ $o$
                \item
                  $\sigma$ = $S_{l}$, $0$ $\leq$ $l$ $\leq$ $\Sigma$
              \end{itemize}

            \end{quote}

          \end{definition}

        \noindent \textbf{\emph{Policy Updates}}

          With all variable occurences grounded to entity identifiers, a
          policy update $U$ is of the form:

          \begin{quote}
            $U$: $a_{0}$ $\land$ ... $\land$ $a_{m}$ $\leftarrow$ $b_{0}$ $\land$ ... $\land$ $b_{n}$

            where $m$, $n$ $\geq$ $0$
          \end{quote}

          In language ${\cal L}^{*}$, a distinction must be made between the
          atoms before $U$ is applied and the atoms after $U$ has been applied.
          Furthermore, such distinction is also needed in the above rule to
          state that all facts $a_{i}$ ($0$ $\leq$ $i$ $\leq$ $m$) holds in
          the resulting state if all facts $b_{j}$ ($0$ $\leq$ $j$ $\leq$ $n$)
          hold in the current state.

          \begin{definition}
            A policy update rule $U$ applied to the current state $\sigma$ is
            expressed as:

            \begin{quote}
              $a_{0}$ $\land$ ... $\land$ $a_{m}$ $\leftarrow$ $b_{0}$ $\land$ ... $\land$ $b_{n}$

              where:

              \begin{itemize}
                \item
                  $a_{i}$ = $[\lnot]$ $a$, $a$ $\in$ ${\cal A}^{\sigma'}$, $0$ $\leq$ $i$ $\leq$ $m$
                \item
                  $b_{j}$ = $[\lnot]$ $b$, $b$ $\in$ ${\cal A}^{\sigma}$, $0$ $\leq$ $j$ $\leq$ $n$
                \item
                  $\sigma'$ = $Res$($U$, $\sigma$)
              \end{itemize}
            \end{quote}
          \end{definition}

      \subsubsection{Additional Constraints}

        In addition to the above translations, there are a few other implicit
        constraints implied by language ${\cal L}$ that needs to be explicitly
        stated in language ${\cal L}^{*}$.

        \noindent \textbf{\emph{Inheritance Rules}}

          All properties held by a group is inherited by all the members and
          subsets of that group. This rule is easy to apply for subject group
          entities. However, careful attention must be given to access right
          and object groups. Subjects holding an access right for an object
          group means that subject also holds that access right for all objects
          in the object group. Similarly, a subject holding an access right
          group for a particular object implies that the subject holds
          all access rights contained in the access right group for that
          object.

          \begin{definition}
            The following are the inheritance constrant rules to allow the
            properties held by a group to propagate to its members and
            subsets.

            \begin{enumerate}
              \item
                \begin{math}
                  \begin{aligned}[t]
                    holds(s_{s}, a, o, \sigma) \leftarrow &
                    holds(s_{g}, a, o, \sigma) \land \\
                    & memb(s_{s}, s_{g}, \sigma)
                  \end{aligned}
                \end{math}

                $\forall (s_{s}, s_{g}, a, o, \sigma)$ where:

                $s_{s} \in {\cal A}^{\sigma}_{SS}$,
                $s_{g} \in {\cal A}^{\sigma}_{SG}$,
                $a \in {\cal A}^{\sigma}_{A}$,
                $o \in {\cal A}^{\sigma}_{O}$,
                $0 \leq \sigma \leq \Sigma$
              \item
                \begin{math}
                  \begin{aligned}[t]
                    holds(s, a_{s}, o, \sigma) \leftarrow &
                    holds(s, a_{g}, o, \sigma) \land \\
                    & memb(a_{s}, a_{g}, \sigma)
                  \end{aligned}
                \end{math}

                $\forall (s, a_{s}, a_{g}, o, \sigma)$ where:

                $s \in {\cal A}^{\sigma}_{S}$,
                $a_{s} \in {\cal A}^{\sigma}_{AS}$,
                $a_{g} \in {\cal A}^{\sigma}_{AG}$,
                $o \in {\cal A}^{\sigma}_{O}$,
                $0 \leq \sigma \leq \Sigma$
              \item
                \begin{math}
                  \begin{aligned}[t]
                    holds(s, a, o_{s}, \sigma) \leftarrow &
                    holds(s, a, o_{g}, \sigma) \land \\
                    & memb(o_{s}, o_{g}, \sigma)
                  \end{aligned}
                \end{math}

                $\forall (s, a, o_{s}, o_{g}, \sigma)$ where:

                $s \in {\cal A}^{\sigma}_{S}$,
                $a \in {\cal A}^{\sigma}_{A}$,
                $o_{s} \in {\cal A}^{\sigma}_{OS}$,
                $o_{g} \in {\cal A}^{\sigma}_{OG}$,
                $0 \leq \sigma \leq \Sigma$
              \item
                \begin{math}
                  \begin{aligned}[t]
                    holds(s_{g1}, a, o, \sigma) \leftarrow &
                    holds(s_{g2}, a, o, \sigma) \land \\
                    & subst(s_{g1}, s_{g2}, \sigma)
                  \end{aligned}
                \end{math}

                $\forall (s_{g1}, s_{g2}, a, o, \sigma)$ where:

                $s_{g1}, s_{g2} \in {\cal A}^{\sigma}_{SG}$,
                $a \in {\cal A}^{\sigma}_{A}$,
                $o \in {\cal A}^{\sigma}_{O}$,
                $0 \leq \sigma \leq \Sigma$
              \item
                \begin{math}
                  \begin{aligned}[t]
                    holds(s, a_{g1}, o, \sigma) \leftarrow &
                    holds(s, a_{g2}, o, \sigma) \land \\
                    & subst(a_{g1}, a_{g2}, \sigma)
                  \end{aligned}
                \end{math}

                $\forall (s, a_{g1}, a_{g2}, o, \sigma)$ where:

                $s \in {\cal A}^{\sigma}_{S}$,
                $a_{g1}, a_{g2} \in {\cal A}^{\sigma}_{AG}$,
                $o \in {\cal A}^{\sigma}_{O}$,
                $0 \leq \sigma \leq \Sigma$
              \item
                \begin{math}
                  \begin{aligned}[t]
                    holds(s, a, o_{g1}, \sigma) \leftarrow &
                    holds(s, a, o_{g2}, \sigma) \land \\
                    & subst(o_{g1}, o_{g2}, \sigma)
                  \end{aligned}
                \end{math}

                $\forall (s, a, o_{g1}, o_{g2}, \sigma)$ where:

                $s \in {\cal A}^{\sigma}_{S}$,
                $a \in {\cal A}^{\sigma}_{A}$,
                $o_{g1}, o_{g2} \in {\cal A}^{\sigma}_{OG}$,
                $0 \leq \sigma \leq \Sigma$
            \end{enumerate}
          \end{definition}

        \noindent \textbf{\emph{Transitivity Rules}}

          Given three groups $G$, $G'$ and $G''$. If $G$ is a subset of $G'$
          and $G'$ is a subset of $G''$, then $G$ must also be a subset of
          $G''$.

          \begin{definition}
            A subset of a set is also a subset of its superset's superset.

            \begin{enumerate}
              \item
                \begin{math}
                  \begin{aligned}[t]
                    subst(sg_{1}, sg_{3}, \sigma) \leftarrow &
                    subst(sg_{1}, sg_{2}, \sigma) \land \\
                    & subst(sg_{2}, sg_{3}, \sigma)
                  \end{aligned}
                \end{math}

                $\forall (sg_{1}, sg_{2}, sg_{3}, \sigma)$ where:

                $sg_{1}, sg_{2}, sg_{3} \in {\cal A}^{\sigma}_{SG}$,
                $0 \leq \sigma \leq \Sigma$
              \item
                \begin{math}
                  \begin{aligned}[t]
                    subst(ag_{1}, ag_{3}, \sigma) \leftarrow &
                    subst(ag_{1}, ag_{2}, \sigma) \land \\
                    & subst(ag_{2}, ag_{3}, \sigma)
                  \end{aligned}
                \end{math}

                $\forall (ag_{1}, ag_{2}, ag_{3}, \sigma)$ where:

                $ag_{1}, ag_{2}, ag_{3} \in {\cal A}^{\sigma}_{AG}$,
                $0 \leq \sigma \leq \Sigma$
              \item
                \begin{math}
                  \begin{aligned}[t]
                    subst(og_{1}, og_{3}, \sigma) \leftarrow &
                    subst(og_{1}, og_{2}, \sigma) \land \\
                    & subst(og_{2}, og_{3}, \sigma)
                  \end{aligned}
                \end{math}

                $\forall (og_{1}, og_{2}, og_{3}, \sigma)$ where:

                $og_{1}, og_{2}, og_{3} \in {\cal A}^{\sigma}_{OG}$,
                $0 \leq \sigma \leq \Sigma$

            \end{enumerate}
          \end{definition}

        \noindent \textbf{\emph{Inertial Rules}}

          Intuitively, all facts in the current state that are not affected by
          a policy update should be carried over to the next state after the
          update. In language ${\cal L}^{*}$, this rule must be explicitly
          stated as a constraint.

          \begin{definition}
            A fact $F$ that holds in state $\sigma$ should also hold in state
            $\sigma'$, given $\sigma'$ = $Res$($U$, $\sigma$), if fact $F$ is
            not affected by the application of update $U$.

            \begin{enumerate}
              \item
                \begin{math} 
                  \begin{aligned}
                    x 
                  \end{aligned}
                \end{math}
            \end{enumerate}
          \end{definition}

    \subsection{Example}

  \section{Semantic Properties}

  \section{Implementation}

    \subsection{System Structure}

    \subsection{Algorithms}

  \section{Case Study: Web Server Application}

  \section{Conclusion}

    [summary]

    [effectiveness of system/application]

    [future work]

\end{document}
