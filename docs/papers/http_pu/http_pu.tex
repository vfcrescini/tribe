\documentclass{llncs}

\pagestyle{plain}

\begin{document}
  \title{HTTP Authorisation with PolicyUpdater:}
  \subtitle{A Logic-Based Access-Control System}

  \author{Vino Fernando Crescini \and Yan Zhang}

  \institute{School of Computing and Information Technology\\
             University of Western Sydney\\
             Penrith South DC, NSW 1797, Australia \\
             E-Mail: \{jcrescin,yan\}@cit.uws.edu.au}

  \maketitle

  \begin{abstract}
    [text]
  \end{abstract}

  \section{Introduction}

  \section{PolicyUpdater System}

    \subsection{Language ${\cal L}$}

      Language ${\cal L}$ is a first-order logic language...

      \subsubsection{Syntax}

        \paragraph{\bf Identifiers.}
          As the most basic unit of language ${\cal L}$,
          identifiers are used to represent the different components of the
          language. There are three main classes of identifiers:

          \begin{itemize}
            \item
              {\bf Entity Identifiers} represent constant entities that make
              up an atom. They are divided further into three types, with each
              type again divided into the {\em singular} and {\em group} entity
              catogories:
 
              \begin{itemize}
                \item
                  {\em Subjects}. e.g. bob, administrators, users.
                \item
                  {\em Access Rights}. e.g. read, write, own.
                \item
                  {\em Objects}. e.g. file, directory, database.
              \end{itemize}
            \vspace{1mm}
            \item
              {\bf Policy Update Identifiers} are used as name handles or
              labels for policy update definitions and directives. e.g.
              $grant\_read$, $revoke\_write$.
 
            \vspace{1mm}
            \item
              {\bf Variable Identifiers} are used as placeholders to represent
              defined entity identifiers.
          \end{itemize}
 
          The following regular expression shows the syntax of Entity and
          Policy Update Identifiers:
 
          \begin{verbatim}[a-z]([a-zA-Z0-9_]){0,127}\end{verbatim}
 
          Variable Identifiers in consrast, are prefixed with an upper-case
          character:
 
          \begin{verbatim}[A-Z]([a-zA-Z0-9_]){0,127}\end{verbatim}

        \paragraph{\bf Atoms.}
          An atom is composed of a relation with 2 to 3 entity or variable
          identifiers that represent a logical relationship between the
          entities. There are three types of atoms:

          \begin{itemize}
            \item
              {\bf Holds.} Atoms of this type states that the subject
              identifier $sub$ holds the access right identifier $acc$ for the
              object identifier $obj$.

              \begin{verbatim}holds(<sub>,<acc>.<obj>)\end{verbatim}

            \vspace{1mm}
            \item
              {\bf Membership.} This type of atom states that the singular
              identifier $elt$ is a member or element of the group identifier
              $grp$. It is important to note that identifiers $elt$ and $grp$
              must be of the same base type (e.g. subject singular and subject
              group).

              \begin{verbatim}memb(<elt>, <grp>)\end{verbatim}

            \vspace{1mm}
            \item
              {\bf Subset.} The subset atom states that the group identifiers
              $grp1$ and $grp2$ are of the same types and that group $grp1$ is
              a subset of the group $grp2$.

              \begin{verbatim}subst(<grp1>, <grp2>)\end{verbatim}
          \end{itemize}

          Atoms that contain no variables, i.e. composed entirely of entity
          identifiers, are called {\em ground atoms}.

        \paragraph{\bf Facts.}

          A fact makes a claim that the relationship represented by an atom or
          its negation holds in the current context. Facts are negated by the
          use of the negation operator ($!$). The following shows the formal
          syntax of a fact:

          \begin{verbatim}[!]<holds_atm>|<memb_atm>|<subst_atm>\end{verbatim}

          Note that facts may be made up of atoms that contain variable
          identifiers. Facts with no variable occurrences are called
          {\em ground facts}.

        \paragraph{\bf Expressions.}
          An expression is either a fact, or a logical conjunction of facts,
          separated by the double-ampersand characters $\&\&$.

          \begin{verbatim}<fact1> [&& <fact2> [&& ...]]\end{verbatim}

          Expressions that are made up of only ground facts are called
          {\em ground expressions}.

        \paragraph{\bf Entity Identifier Definition.}

          All entity identifiers (subjects, access rights, objects and groups)
          must first be declared before any other statements to define the
          entity domain of the policy base. The following entity declaration
          syntax illustrates how to define one or more entity identifiers of a
          particular type.

          \begin{verbatim}ident sub|acc|obj[-grp] <e_id>[, ...]];\end{verbatim}

        \paragraph{\bf Initial Fact Definition.}

          The initial facts of the policy base, those that hold before any
          policy updates are performed, are defined by using the following
          definition syntax:

          \begin{verbatim}initially <ground-exp>;\end{verbatim}

        \paragraph{\bf Constraint Definition.}

          Constraints are logical rules that hold regardless of any changes
          that may occur when the policy base is updated. The constraint rules
          are true in the initial state and remains true after any policy
          update.

          The constraint syntax below shows that for any state of the policy
          base, expression $exp1$ holds if expression $exp2$ is true and there
          is no evidence that $exp3$ is true. The $with$ $absence$ clause
          allows constraints to behave like default propositions, where the
          absence of proof of an expression implies that the expression holds.

          It is important to note that the expressions $exp1$, $exp2$ and
          $exp3$ are non-ground expressions, which means identifiers within
          them may be variables.

          \begin{verbatim}always <exp1>
  [implied by <exp2>
  [with absence <exp3>]];\end{verbatim}

        \paragraph{\bf Policy Update Definition.}

          Before a policy update can be applied, it must first be defined by
          using the following syntax:

          \begin{verbatim}<up_id>([<var_id>[, ...]])
  causes <exp1>
  if <exp2>;\end{verbatim}

          $up\_id$ is the policy update identifier to be used in referencing
          this policy update. The optional $var\_id$ list are the variable
          identifiers occurring the expressions $exp1$ and $exp2$ and will
          eventually be replaced by entity identifiers when the update is
          referenced. The postcondition expression $exp1$ is an expression that
          will hold in the state after this update is applied. The expression
          $exp2$ is a precondition expression that must hold in the current
          state before this update is applied.

          It is important to note that a policy update definition will have no
          effect on the policy base until it is applied by one of the
          directives described in the following section.

        \paragraph{\bf Policy Update Directives.}

          The policy update sequence list contains a list of references to
          defined policy updates in the domain. The policy updates in the
          sequence list are applied to the current state of the policy base one
          at a time to produce a policy base state upon which queries can be
          evaluated.

          The following four directives are the policy sequence manipulation
          features of language $\cal{L}$.

          \begin{itemize}
            \item
              {\bf Adding an update into the sequence.} Defined policy updates
              are added into the sequence list through the use of the following
              directive statement:

              \begin{verbatim}seq add <up_id>([<e_id>[, ...]]);\end{verbatim}
                                                                                
              \noindent where $up\_id$ is the identifier of a defined policy
              update and the $e\_id$ list is a comma-separated list of entity
              identifiers that will replace the variable identifiers that occur
              in the definition of the policy update.

            \vspace{1mm}
            \item
              {\bf Listing the updates in the sequence.} The following
              directive may be used to list the current contents of the policy
              update sequence list.
                                                                                
              \begin{verbatim}seq list;\end{verbatim}
                                                                                
              This directive is answered with an ordinal list of policy
              updates in the form:
                                                                                
              \begin{verbatim}<n> <up_id>([e_id[, ...]])\end{verbatim}

              \noindent where $n$ is the ordinal index of the policy update
              within the sequence list starting at 0. $up\_id$ is the policy
              update identifier and the $e\_id$ list is the list of entity
              identifiers used to replace the variable identifier
              place-holders.

            \vspace{1mm}
            \item
              {\bf Removing an update from the sequence.} The syntax below
              shows the directive to remove a policy update reference from the
              list. $n$ is the ordinal index of the policy update to be
              removed. Note that removing a policy update reference from the
              sequence list may change the ordinal index of other update
              references.
                                                                  
              \begin{verbatim}seq del <n>;\end{verbatim}

            \vspace{1mm}
            \item
              {\bf Computing an update sequence.} The policy updates in the
              sequence list does not get applied until the $compute$ directive
              is issued. The directive causes the policy update references in
              the sequence list to be applied one at a time in the same order
              that they appear in the list. The directive also causes the
              system to generate the policy base models against which query
              requests can be evaluated.
                                                               
             \begin{verbatim}compute;\end{verbatim}
          \end{itemize}

        \paragraph{\bf Query Directive.}

          A ground query expression may be issued against the current state of
          the policy base. This current state is derived after all the updates
          in the update sequence have been applied, one at a time, upon the
          initial state. Query expressions are answered with a $true$, $false$
          or an $unknown$, depending on whether the queried expression holds,
          its negation holds, or neither, respectively. Syntax is as follows:

          \begin{verbatim}query <ground-exp>;\end{verbatim}

      \subsubsection{Semantics}

    \subsection{Policy Updates}

    \subsection{Query Evaluation}

  \section{HTTP Authorisation with PolicyUpdater}

  \section{Conclusion}

  \begin{thebibliography}{5}
  \end{thebibliography}

\end{document}
